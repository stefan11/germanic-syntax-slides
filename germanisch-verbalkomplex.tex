%% -*- coding:utf-8 -*-


\subtitle{Verbalkomplexbildung in den SOV-Sprachen}

\section{Verbalkomplexbildung in den SOV-Sprachen}


\huberlintitlepage[22pt]



\frame{
\frametitle{Literaturhinweis}



Zu diesem Abschnitt gibt es das Kapitel~4 in \citew{MuellerGermanic}.

Müller, Stefan, \citeyear{MuellerGermanic}. \emph{Germanic Syntax}. Berlin: Language Science
Press. In Vorbereitung. 



}


\frame{
\frametitle{Verbalkomplexbildung}

\begin{itemize}
\item Deutsch, Niederländisch erlauben Verbalkomplexbildung:
\ea
weil \highlight{es}<1> \highlight{ihr}<2> \highlight{jemand}<3> \highlight{zu lesen}<1> \highlight{versprochen}<2> \highlight{hat}<3> \citep{Haider90b}
\z

\pause
\pause
\pause
Die Verben am Ende verhalten sich wie ein einfaches Verb. 

Umordnung der Argumente ist möglich.


\pause
\item Niederländisch:
\eal
\ex
\gll dat Jan het boek wil lezen\\
     dass Jan das Buch will lesen\\
\glt `dass Jan das Buch lesen will'
% that John the book wants read
% 'that John wants to read the book' 
\ex
\gll dat Jan Marie het boek laat lezen\\
     dass Jan Maria das Buch läßt lesen\\
%that John Mary the book lets read
%'that John lets Mary read the book'
\ex 
\gll dat Jan Marie het boek wil laten lezen\\
     dass Jan Marie das Buch will lassen lesen\\
\glt `dass Jan Maria das Buch lesen lassen will'
%that John Mary the book wants let read 'that John wants to let Mary read the book'
\zl

\pause
\item Englisch, Dänisch, \ldots{} erlauben keine Umordnung von Konstituenten

\end{itemize}

}

\frame{
\frametitle{Variation}


\begin{itemize}
\item Bei den Abfolgen im Verbalkomplex gibt es extreme Variation. 

\pause
\item Standardsprachlich: übergeordnetes Verb steht rechts: V$_3$ V$_2$ V$_1$
\ea
weil es ihm jemand zu lesen versprochen hat \citep{Haider90b}
\z
\pause
\item In den Dialekten gibt es aber die buntesten Abfolgen.


Z.B. folgende \citep[376]{Mueller99a}:
\eal
\ex Ich hätte stapelweise Akten kön\-nen haben.
\ex weil ich mir das nich hab' lassen gefallen
\ex wenn se mir hier würden rausschmeißen, \ldots
\zl
(Interviewpartner in:\\
\emph{Insekten und andere Nachbarn -- ein Haus in Berlin}, ARD 15.11.1995)



\end{itemize}


}

\hfsetfillcolor{green!50!lime!30}
\hfsetbordercolor{green!40!black}

\frame{
\frametitle{Argumentanziehung}


\begin{forest}
sm edges
[V\feattab{
%              \vform \type{fin},\\
              \highlight<4>{\sliste{ NP[\type{nom}], NP[\type{acc}] }} } 
        [{\highlight<1,3>{V}\feattab{
%              \vform \type{bse},\\
              \sliste{ \highlight<2,4>{NP[\type{nom}], NP[\type{acc}]}} }}, name=lesen [lesen] ]
        [V\feattab{
%              \vform \type{fin},\\
              \sliste{ \visible<2->{\highlight<2>{NP[\type{nom}], NP[\type{acc}]},} \highlight<1,3>{V}}}, name=wird [wird] ]
]
\only<2>{\draw[semithick,->] (lesen)..controls +(south east:2) and +(south west:2)..(wird);}
\end{forest}

%% \begin{forest}
%% sm edges
%% [V\feattab{
%% %              \vform \type{fin},\\
%%               \sliste{ NP[\type{nom}], NP[\type{acc}] } } 
%%         [\highlight<1>{V}\feattab{
%% %              \vform \type{bse},\\
%%               \sliste{ NP[\type{nom}], NP[\type{acc}]} }, name=lesen [lesen] ]
%%         [V\feattab{
%% %              \vform \type{fin},\\
%%               \sliste{ NP[\type{nom}], NP[\type{acc}], \highlight<1>{V}}}, name=wird [wird] ]
%% ]
%% \draw[semithick,->] (lesen)..controls +(south east:2) and +(south west:2)..(wird);
%% \end{forest}

\begin{itemize}
\item \emph{wird} verlangt Infinitiv ohne \emph{zu} \visible<2->{und dessen Argumente\\
      \citep{Geach70a,HN94a}}
\pause
\pause
\item Verb wird gesättigt und ist am Mutterknoten nicht mehr in der Valenzliste
\pause
\item Kombination aus \emph{lesen} und \emph{wird} verhält sich wie einfaches Verb und kann mit den
  Argumenten in beliebiger Reihenfolge kombiniert werden.
\end{itemize}

}


\frame{
\frametitle{Verbalkomplexbildung und Scrambling: Normalstellung}

\centerfit{
\begin{forest}
sm edges
[V\feattab{
              \sliste{ }}
        [{NP[\type{nom}]} [keiner] ]
        [V\feattab{
              \sliste{ NP[\type{nom}] }}
          [{NP[\type{acc}]} [das Buch, roof] ]
          [V\feattab{
%              \vform \type{fin},\\
              \sliste{ NP[\type{nom}], NP[\type{acc}]}} 
             [V\feattab{
%              \vform \type{bse},\\
              \sliste{ NP[\type{nom}], NP[\type{acc}]}} [lesen] ]
             [V\feattab{
%              \vform \type{fin},\\
                \sliste{ NP[\type{nom}], NP[\type{acc}], V }} [wird] ] ] ] ]
\end{forest}}


}

\frame{
\frametitle{Verbalkomplexbildung und Scrambling: Acc $<$ Nom}

\centerfit{\begin{forest}
sm edges
[V\feattab{
              \sliste{ }}
        [{NP[\type{acc}]} [das Buch, roof] ]
        [V\feattab{
              \sliste{ NP[\type{acc}] }}
          [{NP[\type{nom}]} [keiner] ]
          [V\feattab{
%              \vform \type{fin},\\
              \sliste{ NP[\type{nom}], NP[\type{acc}] }} 
             [V\feattab{
%              \vform \type{bse},\\
              \sliste{ NP[\type{nom}], NP[\type{acc}]}} [lesen] ]
             [V\feattab{
%              \vform \type{fin},\\
                \sliste{ NP[\type{nom}], NP[\type{acc}], V }} [wird] ] ] ] ]
\end{forest}}


}

\subsection{Argumentanziehung im Detail}

\frame{


\ea
\emph{lesen} infinite Form:\\
\ms{
subj  & \sliste{ NP[\type{nom}] }\\
comps & \sliste{ NP[\type{acc}] }\\
}
\z

Subjekte können nur mit finiten Verben kombiniert werden:
\eal
\ex[]{
Das Buch lesen wollte ein Mann.
}
\ex[*]{
Ein Mann lesen wollte das Buch.
}
\zl

\pause
\ea
\emph{werden} infinite Form:\\
\ms{
subj  & \ibox{1}\\
comps & \ibox{2} $\oplus$ \sliste{ V[ \vform \type{bse}, %\textsc{lex}+, 
                                   \subj \ibox{1}, \comps \ibox{2} ] }\\
}
\z

Das Hilfsverb \emph{werden} verlangt einen Infinitiv ohne zu (\vform \type{bse}).

%\pause
%\textsc{lex}+ sorgt dafür, dass das verbale Komplement ein Wort ist.

\pause

Das Subjekt \iboxb{1} und die anderen Argumente \iboxb{2} werden übernommen.


}


\frame{
\frametitle{Argumentanziehung im Detail}

%% \begin{tikzpicture}
%% \tikzset{level 1+/.style={level distance=4\baselineskip}}
%% \tikzset{level 2/.style={level distance=4\baselineskip}}
%% \tikzset{frontier/.style={distance from root=9\baselineskip}}
%% \Tree[.V\feattab{
%%               \vform \type{fin},\\
%%               \comps \highlight{\ibox{1} $\oplus$ \ibox{2}}<3> } 
%%         [.{\highlight{\ibox{3} V}<1>\feattab{
%%               \highlight{\vform \type{bse}}<1>,\\
%%               \subj  \highlight{\ibox{1} \sliste{ NP[\type{nom}] }}<2,3>, \\ 
%%               \comps \highlight{\ibox{2} \sliste{ NP[\type{acc}] }}<2,3> }} lesen ]
%%         [.V\feattab{
%%               \vform \type{fin},\\
%%               \comps \highlight{\ibox{1} $\oplus$ \ibox{2}}<2> $\oplus$ \sliste{ \highlight{\ibox{3}}<1> } } wird ]
%% ]
%% \end{tikzpicture}

%% \begin{tikzpicture}
%% \tikzset{level 1+/.style={level distance=4\baselineskip}}
%% \tikzset{level 2/.style={level distance=4\baselineskip}}
%% \tikzset{frontier/.style={distance from root=9\baselineskip}}
%% \Tree[.V\feattab{
%%               \vform \type{fin},\\
%%               \comps \highlight{\ibox{1} $\oplus$ \ibox{2}}<3> } 
%%         [.{\highlight{\ibox{3} V}<1>\feattab{
%%               \highlight{\vform \type{bse}}<1>,\\
%%               \subj  \highlight{\ibox{1} \sliste{ NP[\type{nom}] }}<2,3>, \\ 
%%               \comps \highlight{\ibox{2} \sliste{ NP[\type{acc}] }}<2,3> }} lesen ]
%%         [.V\feattab{
%%               \vform \type{fin},\\
%%               \comps \highlight{\ibox{1} $\oplus$ \ibox{2}}<2> $\oplus$ \sliste{ \highlight{\ibox{3}}<1> } } wird ]
%% ]
%% \end{tikzpicture}

\begin{forest}
sm edges
[V\feattab{
              \vform \type{fin},\\
              \comps \ibox{1} $\oplus$ \ibox{2} } 
        [{\ibox{3} V}\feattab{
              \vform \type{bse},\\
              \subj  \ibox{1} \sliste{ NP[\type{nom}] }, \\ 
              \comps \ibox{2} \sliste{ NP[\type{acc}] } } [lesen] ]
        [V\feattab{
              \vform \type{fin},\\
              \comps \ibox{1} $\oplus$ \ibox{2} $\oplus$ \sliste{ \ibox{3} } } [wird] ] ]
\end{forest}


\begin{itemize}
\item Hilfsverb verlangt Infinitiv ohne \emph{zu} \iboxb{3}.
\pause
\item Subjekt \iboxb{1} und Komplemente \iboxb{2} werden übernommen.
\pause
\item \emph{lesen wird} hat dieselben Argumente wie \emph{liest}
\end{itemize}



}

\frame{
\frametitle{Komplexere Komplexe}


%% %\centerfit{%
%% \begin{tikzpicture}
%% \tikzset{level 1+/.style={level distance=4\baselineskip}}
%% \tikzset{level 2+/.style={level distance=5\baselineskip}}
%% \tikzset{frontier/.style={distance from root=14\baselineskip}}
%% \Tree[.V\feattab{
%%               \vform \type{fin},\\
%%               \comps \highlight{\ibox{1} $\oplus$ \ibox{2}}<3> } 
%%         [.{\ibox{4} V\feattab{
%%               \vform \type{bse},\\
%%               \subj  \ibox{1},\\
%%               \comps \highlight{\ibox{2}}<3> }} 
%%            [.{\highlight{\ibox{3} V}<1>\feattab{
%%               \highlight{\vform \type{bse}}<1>,\\
%%               \subj  \highlight{\ibox{1} \sliste{ NP[\type{nom}] }}<2,3>, \\ 
%%               \comps \highlight{\ibox{2} \sliste{ NP[\type{acc}] }}<2,3> }} lesen ]
%%            [.V\feattab{
%%               \vform \type{bse},\\
%%               \subj  \ibox{1},\\
%%               \comps \ibox{2} $\oplus$ \sliste{
%%                 \highlight{\ibox{3}}<1> } } können ] ]
%%         [.V\feattab{
%%               \vform \type{fin},\\
%%               \comps \highlight{\ibox{1} $\oplus$ \ibox{2}}<2> $\oplus$ \sliste{
%%                 \highlight{\ibox{4}}<1> } } wird ] 
%% ]
%% \end{tikzpicture}}



\centerline{\scalebox{0.9}{
\begin{forest}
sm edges
[V\feattab{
              \vform \type{fin},\\
              \comps \ibox{1} $\oplus$ \ibox{2} } 
        [{\ibox{4} V\feattab{
              \vform \type{bse},\\
              \subj  \ibox{1},\\
              \comps \ibox{2} }} 
           [{\ibox{3} V\feattab{
              \vform \type{bse},\\
              \subj  \ibox{1} \sliste{ NP[\type{nom}] }, \\ 
              \comps \ibox{2} \sliste{ NP[\type{acc}] } }} [lesen] ]
           [V\feattab{
              \vform \type{bse},\\
              \subj  \ibox{1},\\
              \comps \ibox{2} $\oplus$ \sliste{
                \ibox{3} } } [können] ] ]
        [V\feattab{
              \vform \type{fin},\\
              \comps \ibox{1} $\oplus$ \ibox{2} $\oplus$ \sliste{
                \ibox{4} } } [wird] ] 
]
\end{forest}}}

Das geht auch zu dritt, Hauptsache, einer übernimmt die Verantwortung



}


\frame{
\frametitle{Oder auch mal andersrum}





%\centerline{\scalebox{0.7}{
%% \centerfit{%
%% \begin{tikzpicture}
%% \tikzset{level 1+/.style={level distance=4\baselineskip}}
%% \tikzset{level 2+/.style={level distance=5\baselineskip}}
%% \tikzset{frontier/.style={distance from root=14\baselineskip}}
%% \Tree[.V\feattab{
%%               \vform \type{fin},\\
%%               \comps \highlight{\ibox{1} $\oplus$ \ibox{2}}<3> } 
%%         [.V\feattab{
%%               \vform \type{fin},\\
%%               \comps \highlight{\ibox{1} $\oplus$ \ibox{2}}<2> $\oplus$ \sliste{
%%                 \highlight{\ibox{4}}<1> } } wird ]
%%         [.{\ibox{4} V\feattab{
%%               \vform \type{bse},\\
%%               \subj  \ibox{1},\\
%%               \comps \highlight{\ibox{2}}<3> }} 
%%            [.{\highlight{\ibox{3} V}<1>\feattab{
%%               \highlight{\vform \type{bse}}<1>,\\
%%               \subj  \highlight{\ibox{1} \sliste{ NP[\type{nom}] }}<2,3>, \\ 
%%               \comps \highlight{\ibox{2} \sliste{ NP[\type{acc}] }}<2,3> }} lesen ]
%%            [.V\feattab{
%%               \vform \type{bse},\\
%%               \subj  \ibox{1},\\
%%               \comps \ibox{2} $\oplus$ \sliste{
%%                 \highlight{\ibox{3}}<1> } } können ] ] 
%% ]
%% \end{tikzpicture}}
%% %}

\centerline{%
\begin{forest}
sm edges
[V\feattab{
              \vform \type{fin},\\
              \comps \ibox{1} $\oplus$ \ibox{2} } 
        [V\feattab{
              \vform \type{fin},\\
              \comps \ibox{1} $\oplus$ \ibox{2} $\oplus$ \sliste{
                \ibox{4} } } [wird] ]
        [{\ibox{4} V\feattab{
              \vform \type{bse},\\
              \subj  \ibox{1},\\
              \comps \ibox{2} }} 
           [{\ibox{3} V\feattab{
              \vform \type{bse},\\
              \subj  \ibox{1} \sliste{ NP[\type{nom}] }, \\ 
              \comps \ibox{2} \sliste{ NP[\type{acc}] } }} [lesen] ]
           [V\feattab{
              \vform \type{bse},\\
              \subj  \ibox{1},\\
              \comps \ibox{2} $\oplus$ \sliste{
                \ibox{3} } } [können] ] ] 
]
\end{forest}}


}

\subsection{Keine Verbalkomplexe bei VO-Sprachen}

\frame{
\frametitle{English, Dänisch, \ldots}



\begin{itemize}
\item Normalerweise muss ein Argument vollständig sein,\\
      wenn es mit seinem Kopf kombiniert wird.
\pause
\item Verbalkomplexe sind anders: Wörter werden direkt verbunden.
\pause
\item Englisch und Dänisch haben nur die normale Regel, im Deutschen, Niederländischen gibt es
  zusätzlich die Verbalkomplexregel.

\pause
\item Die Hilfsverben betten in den SVO-Sprachen eine Verbphrase ein:
\ea
\gll Peter [has [read the book]].\\
     Peter \spacebr{}hat \spacebr{}gelesen das Buch\\
\z
\end{itemize}

}

\frame{
\frametitle{Verbalkomplexe ohne Verbalkomplexbildung?}

\begin{itemize}
\item Vorschläge, Hilfsverben auch mit einer VP zu kombinieren \citep{Wurmbrand2003b}:
\ea
dass keiner [[das Buch lesen] wird]
\z
\pause
\item Wie funktioniert dann Scrambling?
\ea
dass [das Buch]$_i$ keiner [[ \_$_i$ lesen] wird]
\z

\pause
\item Analysen, die Scrambling als Bewegung analysieren sind problematisch, da sie das Vorhandensein
  zusätzlicher Lesarten bei der Umstellung von NPen mit Quantoren vorhersagen (\citealp[\page 146]{Kiss2001a}; \citealp[Abschnitt~2.6]{Fanselow2001a}).

\end{itemize}

}

\frame{
\frametitle{Übungsaufgaben}


\begin{enumerate}
\item Skizzieren Sie die Analyse der Verbalkomplexe für die folgenden Beispiele:
\eal
\ex dass er darüber lachen wird
\ex dass er darüber wird lachen müssen
\ex dass er über diesen Witz wird haben lachen müssen
\zl
\item Suchen Sie in der Zeitung oder in Korpora (COSMAS, COW) zwei Verbalkomplexe mit mindestens
  drei Verben und analysieren Sie diese.

\item Suchen Sie in Korpora Verbalkomplex mit mehr als vier Verben. Dokumentieren Sie Ihr Vorgehen.
\end{enumerate}

}


%      <!-- Local IspellDict: de_DE -->



