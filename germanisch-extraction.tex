%% -*- coding:utf-8 -*-
\subsection{Verbzweitstellung und Voranstellung im Englischen}


\frame{
\frametitle{Extraktion}


\begin{itemize}
\item Auch in Sprachen mit relativ fester Konstituentenstellung ist es mitunter möglich, Konstituenten
umzustellen:
\eal
\ex This book, I read yesterday.
\ex Yesterday, I read this book.
\zl

\pause

\item Die germanischen (V2-)Sprachen stellen irgendeine Konstituente vor das finite Verb:
\eal
\ex \alert{Ich} habe das Buch gestern gelesen.
\ex \alert{Das Buch} habe ich gestern gelesen.
\ex \alert{Gestern} habe ich das Buch gelesen.
\ex \alert{Gelesen} habe ich das Buch gestern,\\
    gekauft hatte ich es aber schon vor einem Monat.
\ex \alert{Das Buch gelesen} habe ich gestern.
\zl

\end{itemize}

}


\frame{
\frametitle{Extraktion ist nicht satzgebunden}

\begin{itemize}
\item Extraktion kann über Satzgrenzen hinweggehen:
\eal
\ex Chris, David saw.
\ex Chris, we think that David saw.
\ex Chris, we think Anna claims that David saw.
\zl

\pause

\item Im Deutschen wohl eher in den süddeutschen Varietäten, aber:

\eal
\label{bsp-Fernabhaengigkeit}
\ex\label{bsp-um-zwei-millionen}
{}[Um zwei Millionen Mark]$_i$ soll er versucht haben,\\
{}[eine Versicherung \_$_i$ zu betrügen].\footnote{
         taz, 04.05.2001, S.\,20.
}
\ex
"`Wer$_i$, glaubt\iw{glauben} er, daß er \_$_i$ ist?"' erregte sich ein Politiker vom Nil.\footnote{
        Spiegel, 8/1999, S.\,18.
}
\ex\label{ex-wen-glaubst-du-dass}
Wen$_i$ glaubst du, daß ich \_$_i$ gesehen habe.\footnote{
    \citew[\page84]{Scherpenisse86a}.
    }
\ex {}[Gegen ihn]$_i$ falle es den Republikanern hingegen schwerer,\\
    {}[ [ Angriffe \_$_i$] zu lancieren].\footnote{
  taz, 08.02.2008, S.\,9.
}
\zl

\end{itemize}


}

\frame{
\frametitle{Weitergabe von Information im Baum}


%% \centerfit{\begin{tikzpicture}[
%% level 1+/.style={level distance=3\baselineskip},
%% frontier/.style={distance from root=12\baselineskip},
%% connect/.style={semithick,<->,color=green}]
%% %\draw (-3,-5) to[grid with coordinates] (4,0);
%% \Tree[.S
%%         [.\node (NP) {NP}; Chris ]
%%         [.\node (S/NP) {S/NP};
%%           [.NP David ] 
%%           [.\node (VP/NP) {VP/NP};  
%%             [.V saw ]
%%             [.\node (NP/NP) {NP/NP}; \trace{} ] ] ] ]
%% \draw[connect] (NP/NP.north east) [bend right] to (VP/NP.south east);
%% \draw[connect] (VP/NP.north east) [bend right] to (S/NP.south east);
%% \draw[connect] (S/NP.north east) [bend right] to (NP);
%% \end{tikzpicture}}

\medskip
\centerline{%
\begin{forest}
sm edges
[S
  [\gruen<3>{NP} [Chris] ]
  [S/\gruen<2>{NP} 
    [NP [David] ] 
    [VP/\gruen<2>{NP}  
      [V [saw] ]
      [NP/\gruen<1>{NP} [\trace] ] ] ] ]
%% todo to do
%% \draw[connect] (NP/NP.north east) [bend right] to (VP/NP.south east);
%% \draw[connect] (VP/NP.north east) [bend right] to (S/NP.south east);
%% \draw[connect] (S/NP.north east) [bend right] to (NP);
\end{forest}}

\begin{itemize}
\item Hinten fehlt das Objekt (Lücke, Spur): NP/NP
\pause
\item Information über das fehlende Objekt wird an VP- und S-Knoten repräsentiert.
\pause
\item Fehlende NP steht vorn. (der so genannte Füller)
\end{itemize}

}


\frame{
\frametitle{Weitergabe im Baum über größere Entfernungen}


%% \centerfit{\scalebox{.7}{%
%% \begin{tikzpicture}[
%% level 1+/.style={level distance=3\baselineskip},
%% frontier/.style={distance from root=21\baselineskip},
%% connect/.style={semithick,<->,color=green}]
%% %\draw (-3,-5) to[grid with coordinates] (4,0);
%% \Tree[.S
%%         [.\node (NP) {NP}; Chris ]
%%         [.\node (S/NP1) {S/NP};
%%           [.NP we ] 
%%           [.\node (VP/NP1) {VP/NP};  
%%             [.V think ]
%%               [.\node (CP/NP) {CP/NP};
%%                 [.C that ]
%%                 [.\node (S/NP) {S/NP};
%%                   [.NP David ] 
%%                   [.\node (VP/NP) {VP/NP};  
%%                     [.V saw ]
%%                     [.\node (NP/NP) {NP/NP}; \trace{} ] ] ] ] ] ] ]
%% \draw[connect] (NP/NP.north east)  [bend right] to (VP/NP.south east);
%% \draw[connect] (VP/NP.north east)  [bend right] to (S/NP.south east);
%% \draw[connect] (S/NP.north east)   [bend right] to (CP/NP.south east);
%% \draw[connect] (CP/NP.north east)  [bend right] to (VP/NP1.south east);
%% \draw[connect] (VP/NP1.north east) [bend right] to (S/NP1.south east);
%% \draw[connect] (S/NP1.north east)  [bend right] to (NP);

%% \end{tikzpicture}}}


\centerfit{\scalebox{.8}{%
\begin{forest}
sm edges
[S
  [\gruen{NP} [Chris] ]
  [S/\gruen{NP}
    [NP [we] ] 
    [VP/\gruen{NP}  
       [V [think] ]
       [CP/\gruen{NP}
         [C [that] ]
         [S/\gruen{NP}
            [NP [David] ] 
            [VP/\gruen{NP}  
               [V [saw] ]
               [NP/\gruen{NP} [\trace ] ] ] ] ] ] ] ]
%% \draw[connect] (NP/NP.north east)  [bend right] to (VP/NP.south east);
%% \draw[connect] (VP/NP.north east)  [bend right] to (S/NP.south east);
%% \draw[connect] (S/NP.north east)   [bend right] to (CP/NP.south east);
%% \draw[connect] (CP/NP.north east)  [bend right] to (VP/NP1.south east);
%% \draw[connect] (VP/NP1.north east) [bend right] to (S/NP1.south east);
%% \draw[connect] (S/NP1.north east)  [bend right] to (NP);
\end{forest}}}


}

\frame{
\frametitle{Schema für Füller-Kopf-Strukturen}

\vfill
\centerline{
\begin{forest}
[{H}
  [\ibox{1}]
  [H/\ibox{1}]]
\end{forest}
}

\vfill


}


\frame[shrink]{
\frametitle{Extraktion + Verbumstellung = V2: Deutsch (SOV)}



%% \centerfit{\begin{tikzpicture}[
%% level 1+/.style={level distance=3\baselineskip},
%% frontier/.style={distance from root=15\baselineskip},
%% connect/.style={semithick,<->,color=green}]
%% %\draw (-3,-5) to[grid with coordinates] (4,0);
%% \Tree[.S
%%         [.\node (NP) {NP}; \edge[roof]; {[das Buch]$_i$} ]
%%         [.\node (S/NP) {S/NP};
%%           [.{V \sliste{ S/\!/V }} 
%%             [.V liest$_k$ ] ]
%%            [.\node (S//V/NP) {S$/\!/$V/NP};
%%              [.\node (NP/NP) {NP/NP}; \trace$_i${} ]
%%              [.{V$'$$\!/\!/$V}
%%                [.NP Conny ]
%%                [.{V$\!/\!/$V} \_$_k$ ] ] ] ] ] ]
%% \draw[connect] (NP/NP) [bend right] to (S//V/NP.south east);
%% \draw[connect] (S//V/NP.north east) [bend right] to (S/NP.east);
%% \draw[connect] (S/NP.north east) [bend right] to (NP);
%% \end{tikzpicture}}


\centerfit{%
\begin{forest}
sm edges
[S
  [\gruen<2>{NP$_i$} [das Buch, roof] ]
  [S/\gruen<2>{NP}
     [V \sliste{ S\gruen<1>{$/\!/$V} } 
        [\gruen<1>{V} [liest$_j$] ] ]
     [S\gruen<1>{$/\!/$V}\!\gruen<2>{/NP}
        [NP\gruen<2>{/NP} [\trace$_i$] ]
        [V$'$\gruen<1>{$\!/\!/$V}
           [NP [Conny] ]
           [V\gruen<1>{$\!/\!/$V} [\_$_j$] ] ] ] ] ] ]
%% \draw[connect] (NP/NP) [bend right] to (S//V\!/NP.south east);
%% \draw[connect] (S//V\!/NP.north east) [bend right] to (S/NP.east);
%% \draw[connect] (S/NP.north east) [bend right] to (NP);
\end{forest}}


}




\frame{
\frametitle{Extraktion + Verbumstellung = V2: Dänisch (SVO)}


\centerfit{\scalebox{.9}{
\begin{forest}
sm edges
[S
   [NP$_i$ [bogen;Buch.\textsc{def} ] ]
      [S/NP
         [V \sliste{ S$/\!/$V }
           [V [læser$_j$;liest] ] ]
           [S$/\!/$V\!/NP
             [NP [Conny;Conny] ]
             [VP$\!/\!/$V$\!/$NP
               [V$\!/\!/$V  [\_$_j$] ]
               [NP/NP [\trace$_i$ ] ] ] ] ] ] 
%% \draw[connect] (NP/NP.north east) [bend right] to (V/V.south east);
%% \draw[connect] (V/V.north east) [bend right] to (S//V\!/NP.south east);
%% \draw[connect] (S//V\!/NP.north east) [bend right] to (S/NP.east);
%% \draw[connect] (S/NP.north east) [bend right] to (NP);
\end{forest}
}}

Das Deutsche unterscheidet sich vom Dänischen in der OV/VO-Stellung und der damit verbundenen
VP-Bildung, sonst ist bzgl.\ V2 alles gleich.

}





%% \frame{
%% \frametitle{Weitergabe von Information im Baum}

%% \centerfit{\begin{tikzpicture}[
%% level 1+/.style={level distance=3\baselineskip},
%% frontier/.style={distance from root=15\baselineskip},
%% connect/.style={semithick,<->,color=green}]
%% %\draw (-3,-5) to[grid with coordinates] (4,0);
%% \Tree[.S
%%         [.\node (NP) {NP}; wen ]
%%         [.\node (S/NP) {S/NP};
%%           [.{V \sliste{ S/\!/V }} 
%%             [.V glaubst$_k$ ] ]
%%            [.\node (S//V\!/NP) {S$/\!/$V\!/NP};
%%              [.\node (NP/NP) {NP/NP}; \trace{} ]
%%              [.{V$'$$\!/\!/$V}
%%                [.NP Conny ]
%%                [.{V$\!/\!/$V} \_$_k$ ] ] ] ] ] ]
%% \draw[connect] (NP/NP) [bend right] to (S//V\!/NP.south east);
%% \draw[connect] (S//V\!/NP.north east) [bend right] to (S/NP.east);
%% \draw[connect] (S/NP.north east) [bend right] to (NP);
%% \end{tikzpicture}}

%% }

\subsubsection{Vergleich Englisch, Dänisch, Deutsch}

\frame{
\frametitle{Vergleich Englisch, Dänisch, Deutsch}

Die Sätze sehen gleich aus, haben aber eine ganz andere Struktur:
\eal
\ex Conny reads a book.
\ex Conny læser en bog.
\ex Conny liest ein Buch.
\zl

}


\frame{
\frametitle{Strukturen für Englisch, Dänisch, Deutsch}


~\vfill
\scalebox{.8}{%
\begin{forest}
sm edges
[S
  [NP [Conny;Conny]]
  [VP
    [V [reads;liest]]
    [NP [a book;ein Buch,baseline,roof]]]]
\end{forest}}
\hfill
\visible<2->{\scalebox{.8}{%
\begin{forest}
sm edges
[S
   [NP$_i$ [Conny;Conny] ]
      [S/NP
         [V \sliste{ S$/\!/$V }
           [V [læser$_j$;liest] ] ]
           [S$/\!/$V\!/NP
             [NP/NP [\trace$_i$ ] ]
             [VP$\!/\!/$V
               [V$\!/\!/$V  [\_$_j$] ]
               [NP [en bog;ein Buch,baseline,roof] ] ] ] ] ] 
%% \draw[connect] (NP/NP.north east) [bend right] to (V/V.south east);
%% \draw[connect] (V/V.north east) [bend right] to (S//V\!/NP.south east);
%% \draw[connect] (S//V\!/NP.north east) [bend right] to (S/NP.east);
%% \draw[connect] (S/NP.north east) [bend right] to (NP);
\end{forest}}}
\hfill
\visible<3>{\scalebox{.8}{%
\begin{forest}
sm edges
[S
   [NP$_i$ [Conny] ]
      [S/NP
         [V \sliste{ S$/\!/$V }
           [V [liest$_j$] ] ]
           [S$/\!/$V\!/NP
             [NP/NP [\trace$_i$ ] ]
             [V$'$$\!/\!/$V
               [NP [ein Buch,baseline,roof] ] 
               [V$\!/\!/$V  [\_$_j$] ] ] ] ] ] 
\end{forest}}}

\vfill

}


\frame{
\frametitle{V2 in den germanischen Sprachen}



\begin{itemize}
\item Alle germanischen Sprachen (außer Englisch) haben V2-Sätze mit Verbvoranstellung +
  Voranstellung einer Konstituente.
\pause
\item Sätze immer gleich analysiert, auch wenn Subjekt das Vorfeld füllt.
\pause
\item V1-Sätze sind nicht unbedingt Fragen:
\eal
\ex Gibt er ihm das Buch?
\ex Gib mir das Buch!
\zl

\item V2-Sätze nicht unbedingt Aussagen:
\eal
\ex Wem gibt er das Buch?
\ex Jetzt gib ihm das Buch!
\ex Jetzt gibt er ihm das Buch.
\zl

\end{itemize}

\pause\pause\pause

}