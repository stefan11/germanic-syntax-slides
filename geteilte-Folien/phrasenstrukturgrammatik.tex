\subsection{Phrasenstrukturgrammatiken}


\subsubsection{Phrasenstrukturen}

\frame{
\frametitle{Phrasenstrukturen}

\smallframe
\hfill%
\begin{tabular}{@{}l@{\hspace{1cm}}l@{}}
\scalebox{.75}{%
\begin{forest}
sm edges
[S
  [NP [Aicke] ]
  [NP
    [Det [dem] ]
    [N [Affen] ] 
  ]
  [NP
    [Det [den] ]
    [N [Stock] ] 
  ]
  [V [gibt] ]
]
\end{forest}} &
\scalebox{.75111111}{%
\begin{forest}
sm edges
[V
  [NP [Aicke] ]
  [V
    [NP
      [Det [dem] ]
      [N [Affen] ] ]
    [V
      [NP
        [Det [den] ]
        [N [Stock] ] ]
      [V [gibt] ] ] ] ]
\end{forest}}
\\
\\[-0.4ex]
\begin{tabular}{@{~}l@{ }l@{}}
NP & $\to$ Det, N            \\
S  & $\to$ NP, NP, NP, V  \\
\end{tabular} & \begin{tabular}{@{~}l@{ }l@{}}
NP & $\to$ Det, N  \\
V  & $\to$ NP, V\\
\end{tabular}\\
\end{tabular}
\hfill\mbox{}

\medskip
Das Eigentliche sind die Ersetzungsregeln! Die Bäume sind nur die Visualisierung.\\
%
\pause%
%
Aus Platzgründen auch Klammerschreibweise:\\
{}[\sub{S} [\sub{NP} Aicke] [\sub{NP} [\sub{Det} dem] [\sub{N} Affen]]  [\sub{NP} [\sub{Det} den] [\sub{N} Stock]] [\sub{V} gibt]]

\handoutspace
}

\iftoggle{psgbegriffe}{
\subsubsection{Begriffe}


\frame{
\frametitle{Knoten (\emph{node})}

\vfill
\psset{xunit=5mm,yunit=5mm,nodesep=8pt}
\hfill
\begin{pspicture}(0,0)(14,7.4)
\rput(3,7){\rnode{xp}{A}}
\rput(1,4){\rnode{up}{B}}\rput(5,4){\rnode{xs1}{C}}
\rput(5,1){\rnode{vp}{D}}

\psset{angleA=-90,angleB=90,arm=0pt}
\ncdiag{xp}{up}\ncdiag{xp}{xs1}%
\ncdiag{xs1}{vp}\ncdiag{xs1}{xs2}%
\ncdiag{xs2}{wp}\ncdiag{xs2}{x}%
\ncdiag{wp}{yp}\ncdiag{wp}{ws}%

\pause

%\mode<beamer>{
\psset{linecolor=red}%radius=1em}
%}
%\pscircle(3,7){2ex}
\cnode[linewidth=1.5pt](3,7){1.7ex}{nodeA}
\pscircle[linewidth=1.5pt](1,4){1.7ex}\cnode[linewidth=1.5pt](5,4){1.7ex}{nodeC}
\cnode[linewidth=1.5pt](5,1){1.7ex}{nodeD}

\pause

\rput[l](8,7){\rnode{verz}{verzweigend}}
\rput[l](8,6){\rnode{nverz}{n}icht verzweigend}

%\psset{angleA=180,angleB=0,arm=0pt,arrows=->}
\only<3>{
\ncline{->}{verz}{nodeA}
}
\pause
\only<4>{
\ncline{->}{nverz}{nodeC}
}
%\psgrid
\end{pspicture}
\hfill\hfill\mbox{}
\vfill
}

\frame{

\frametitle{Mutter, Tochter und Schwester}

\vfill
\psset{xunit=5mm,yunit=5mm,nodesep=8pt}
\hspace{1cm}%
%\begin{tabular}{@{}l@{\hspace{1cm}}l@{}}
\begin{pspicture}(0,0)(7.4,7.4)
\rput(3,7){\rnode{xp}{A}}
\rput(1,4){\rnode{up}{B}}\rput(5,4){\rnode{xs1}{C}}
\rput(5,1){\rnode{vp}{D}}

\psset{angleA=-90,angleB=90,arm=0pt}
\ncdiag{xp}{up}\ncdiag{xp}{xs1}%
\ncdiag{xs1}{vp}\ncdiag{xs1}{xs2}%
\ncdiag{xs2}{wp}\ncdiag{xs2}{x}%
\ncdiag{wp}{yp}\ncdiag{wp}{ws}%

%\psgrid
\end{pspicture}
\hspace{1cm}\raisebox{3cm}{\begin{tabular}[t]{@{}l@{}}
A ist die Mutter von B und C\\
C ist die Mutter von D\\
B ist die Schwester von C\\
\end{tabular}}


Verhältnisse wie in Stammbäumen

\vfill

}

\iftoggle{einfsprachwiss-exclude}{
\frame{
\frametitle{Dominanz (\emph{dominance})}

\vfill
\psset{xunit=5mm,yunit=5mm,nodesep=8pt}
\hspace{1cm}
\begin{pspicture}(0,0)(7.4,7.4)
\rput(3,7){\rnode{xp}{A}}
\rput(1,4){\rnode{up}{B}}\rput(5,4){\rnode{xs1}{C}}
\rput(5,1){\rnode{vp}{D}}

\psset{angleA=-90,angleB=90,arm=0pt}
\ncdiag{xs1}{xs2}%
\ncdiag{xs2}{wp}\ncdiag{xs2}{x}%
\ncdiag{wp}{yp}\ncdiag{wp}{ws}%

\alt<2>{
\mode<beamer>{
\psset{linecolor=red}
}
\ncdiag{->}{xp}{up}\ncdiag{->}{xp}{xs1}
}{
\ncdiag{xp}{up}\ncdiag{xp}{xs1}%
}
\alt<2,4>{
\mode<beamer>{
\psset{linecolor=red}
}
\ncdiag{->}{xs1}{vp}
}{
\ncdiag{xs1}{vp}
}
%\psgrid
\end{pspicture}
\hspace{1cm}\raisebox{3cm}{\begin{tabular}[t]{@{}l@{}}
A dominiert \only<2->{B, C und D}\\
\only<3->{C dominiert} \only<4->{D} \\
\end{tabular}}

\bigskip

A dominiert B genau dann, wenn A höher im Baum steht und \\
wenn es eine ausschließlich abwärts führende Linie von A nach B gibt.

\pause\pause\pause

\vfill

}

\frame{

\frametitle{Unmittelbare Dominanz (\emph{immediate dominance})}

\psset{xunit=5mm,yunit=5mm,nodesep=8pt}
\hspace{1cm}
\begin{pspicture}(0,0)(7.4,7.4)
\rput(3,7){\rnode{A}{A}}
\rput(1,4){\rnode{B}{B}}\rput(5,4){\rnode{C}{C}}
\rput(5,1){\rnode{D}{D}}

\psset{angleA=-90,angleB=90,arm=0pt}
\ncdiag{C}{xs2}%
\ncdiag{xs2}{wp}\ncdiag{xs2}{x}%
\ncdiag{wp}{yp}\ncdiag{wp}{ws}%

\alt<2>{
\mode<beamer>{
\psset{linecolor=red}
}
\ncdiag{->}{A}{B}\ncdiag{->}{A}{C}
}{
\ncdiag{A}{B}\ncdiag{A}{C}%
}
\alt<4>{
\mode<beamer>{
\psset{linecolor=red}
}
\ncdiag{->}{C}{D}
}{
\psset{linecolor=black}
\ncdiag{C}{D}
}
%\psgrid
\end{pspicture}
\hspace{1cm}\raisebox{3cm}{\begin{tabular}[t]{@{}l@{}}
A dominiert unmittelbar \only<2->{B und C}\\
\only<3->{C dominiert unmittelbar} \only<4->{D} \\
\end{tabular}}

\bigskip

A dominiert unmittelbar B genau dann, wenn \\
A B dominiert und es keinen Knoten C zwischen A und B gibt.

\pause\pause\pause


}


\frame{
\frametitle{Präzedenz}

\begin{description}[<+->]
\item[Präzedenz (\textit{precedence})]~\\ A geht B voran, wenn A in einer Baumgrafik vor B steht und\\
     keiner der beiden Knoten den anderen dominiert. 
\item[Unmittelbare Präzedenz (\emph{immediate precedence})]~\\ Kein Element C zwischen A und B.
\end{description}

}
}%\end{einfsprachwiss-exclude}
}%psgbegriffe


\subsubsection{Eine Beispielgrammatik}


\frame[shrink=8]{
\frametitle{Beispielableitung bei Annahme flacher Strukturen}

\vfill

\bigskip
\parskip0pt
\begin{tabular}[t]{@{}l@{ }l}
\highlight{NP}<5,8> & \highlight{$\to$ Det N}<5,8>\\          
\highlight{S}<10>  & \highlight{$\to$ NP NP NP V}<10>
\end{tabular}\hspace{2cm}%
\begin{tabular}[t]{@{}l@{ }l}
\highlight{NP}<2> & \highlight{$\to$ Aicke}<2>\\
\highlight{Det}<3>  & \highlight{$\to$ dem}<3>\\
\highlight{Det}<6>  & \highlight{$\to$ den}<6>\\
\end{tabular}\hspace{8mm}
\begin{tabular}[t]{@{}l@{ }l}
\highlight{N}<4> & \highlight{$\to$ Affen}<4>\\
\highlight{N}<7> & \highlight{$\to$ Stock}<7>\\
\highlight{V}<9> & \highlight{$\to$ gibt}<9>\\
\end{tabular}
\vfill

\begin{tabular}{@{}llllll@{\hspace{2.5cm}}l}
Aicke            & dem          & Affen          & den          & Stock & gibt                \pause\\
\highlight{NP}<2> & dem          & Affen          & den          & Stock & gibt & \only<handout>{NP $\to$ Aicke}  \pause\\
NP            & \highlight{Det}<3> & Affen          & den          & Stock & gibt & \only<handout>{Det $\to$ das}  \pause\\
NP            & Det            & \highlight{N}<4>  & den          & Stock & gibt & \only<handout>{N $\to$ Buch} \pause\\
NP            &              & \highlight{NP}<5> & den          & Stock & gibt & \only<handout>{NP $\to$ Det N}\pause\\
NP            &              & NP            & \highlight{Det}<6> & Stock & gibt & \only<handout>{Det $\to$ den}  \pause\\
NP            &              & NP            & Det            & \highlight{N}<7>    & gibt & \only<handout>{N $\to$ Stock} \pause\\
NP            &              & NP            &              & \highlight{NP}<8>       & gibt & \only<handout>{NP $\to$ Det N}\pause\\
NP            &              & NP            &              & NP       & \highlight{V}<9>   & \only<handout>{V $\to$ gibt}  \pause\\
              &              &               &              &      & \highlight{S}<10>   & \only<handout>{S $\to$ NP NP NP V}\\
\end{tabular}

\vfill
}


\begin{frame}[fragile]
\frametitle{Do try this at home!}

Sie können solche Grammatiken selbst ausprobieren.
\begin{itemize}
\item Gehen Sie auf \url{https://swish.swi-prolog.org/}.
\item Klicken Sie "`Program"'.
\item Geben Sie folgendes ein:
\begin{verbatim}
s --> np, v, np, np.
np --> det, n.
np --> [Aicke].
det --> [dem].
det --> [den].
n --> [affen].
n --> [stock].
v --> [gibt].
\end{verbatim}
\item Geben Sie in die untere rechte Box folgendes ein: \texttt{s([Aicke,gibt,dem,affen,den,stock],[]).}
\item Wenn in der Box darüber "`true"' erscheint, feiern Sie!
\end{itemize}

\end{frame}

\frame{
\frametitle{Eine Generative Grammatik}

\begin{itemize}
\item Die Grammatik, die Sie eingegeben haben, kann Sätze erzeugen:
\pause
\item Man kann testen, welche Sätze die Grammatik generiert, indem man folgendes eingibt:
\texttt{s([X],[]),print(X),nl,fail.}

\pause
\item \texttt{s([X],[])} fordert Prolog auf, ein X zu finden, das ein "`s"' ist.
\pause
\item \texttt{print(X),nl} gibt das X und eine newline aus und
\pause
\item \texttt{fail} teilt Prolog mit, dass wir nicht zufrieden sind und dass es noch eine weitere
  Lösung suchen soll.
\pause
\item Es versucht weiter, bis es keine weiteren Lösungen mehr gibt und failt dann.
\pause
\item Einige Grammatiken generieren unendlich viele Xe. Dieser Prozess würde also nie terminieren
  (es sei denn, der Computer hat nicht genug Speicher \ldots).

\end{itemize}

}




\frame{

\frametitle{Von der Grammatik beschriebene Sätze}



\begin{itemize}
\item die Grammatik ist zu ungenau:\\
\begin{tabular}{@{}l@{ }l}
NP & $\to$ Det N\\
S  & $\to$ NP NP NP V\\
\end{tabular}
\eal
\ex[]{
Aicke dem Affen den Stock gibt
}
\ex[*]{
ich dem Affen den Stock gibt\\
\pause
(Subjekt"=Verb"=Kongruenz {\em ich\/}, {\em gibt\/})}
\pause
\ex[*]{
Aicke dem Affen dem Stock gibt\\\pause
(Kasusanforderungen des Verbs {\em gibt} verlangt Akkusativ)
}
\pause
\ex[*]{
er dem Affen das Stock gibt\\\pause
(Determinator"=Nomen"=Kongruenz {\em das}, {\em Stock})
}
\zl
\end{itemize}

}

% geht hier nicht, weil das von anderen eingebunden wird
%\exewidth{\exnrfont(12)}

\frame{

\frametitle{Subjekt"=Verb"=Kongruenz (I)}


\begin{itemize}
\item Übereinstimmung in Person (1, 2, 3) und Numerus (sg, pl)
\eal
\ex Ich schlafe. (1, sg)
\ex Du schläfst.  (2, sg)
\ex Er schläft.  (3, sg)
\ex Wir schlafen. (1, pl)
\ex Ihr schlaft.  (2, pl)
\ex Sie schlafen. (3,pl)
\zl
\item Wie drückt man das in Regeln aus?
\end{itemize}

}

\frame{
\frametitle{Subjekt"=Verb"=Kongruenz (II)}

\begin{itemize}
\item Verfeinerung der verwedenten Symbole\\
            aus S $\to$ NP NP NP V wird\\[2ex]
\begin{tabular}{@{}l@{ }l}
S  & $\to$ NP\_1\_sg NP NP V\_1\_sg\\
S  & $\to$ NP\_2\_sg NP NP V\_2\_sg\\
S  & $\to$ NP\_3\_sg NP NP V\_3\_sg\\
S  & $\to$ NP\_1\_pl NP NP V\_1\_pl\\
S  & $\to$ NP\_2\_pl NP NP V\_2\_pl\\
S  & $\to$ NP\_3\_pl NP NP V\_3\_pl\\
\end{tabular}

\item sechs Symbole für Nominalphrasen, sechs für Verben
\item sechs Regeln statt einer
\end{itemize}

}

\frame{

\frametitle{Kasuszuweisung durch das Verb}

\begin{itemize}
\item Kasus muß repräsentiert sein:
\begin{tabular}{@{}l@{ }l}
S  & $\to$ NP\_1\_sg\_nom NP\_dat NP\_acc V\_1\_sg\_ditransitiv\\
S  & $\to$ NP\_2\_sg\_nom NP\_dat NP\_acc V\_2\_sg\_ditransitiv\\
S  & $\to$ NP\_3\_sg\_nom NP\_dat NP\_acc V\_3\_sg\_ditransitiv\\
S  & $\to$ NP\_1\_pl\_nom NP\_dat NP\_acc V\_1\_pl\_ditransitiv\\
S  & $\to$ NP\_2\_pl\_nom NP\_dat NP\_acc V\_2\_pl\_ditransitiv\\
S  & $\to$ NP\_3\_pl\_nom NP\_dat NP\_acc V\_3\_pl\_ditransitiv\\
\end{tabular}
\item insgesamt 3 * 2 * 4 = 24 neue Kategorien für NP
\item 3 * 2 * x  Kategorien für V (x = Anzahl der Valenzmuster)
\end{itemize}

}


\frame[shrink=15]{

\frametitle{Determinator"=Nomen"=Kongruenz}

\begin{itemize}
\item Übereinstimmung in Genus (fem, mas, neu), Numerus (sg, pl) und\\
      Kasus (nom, gen, dat, acc)
\eal
\ex der Mann, die Frau, das Buch (Genus)
\ex das Buch, die Bücher (Numerus)
\ex des Buches, dem Buch (Kasus)
\zl
\pause
\item aus NP $\to$ Det N wird\\[2ex]
\resizebox{\linewidth}{!}{
\begin{tabular}{@{}l@{ }l@{\hspace{4mm}}l@{ }l}
NP\_3\_sg\_nom  & $\to$ Det\_fem\_sg\_nom N\_fem\_sg\_nom & NP\_gen  & $\to$ Det\_fem\_sg\_gen N\_fem\_sg\_gen\\
NP\_3\_sg\_nom  & $\to$ Det\_mas\_sg\_nom N\_mas\_sg\_nom & NP\_gen  & $\to$ Det\_mas\_sg\_gen N\_mas\_sg\_gen\\
NP\_3\_sg\_nom  & $\to$ Det\_neu\_sg\_nom N\_neu\_sg\_nom & NP\_gen  & $\to$ Det\_neu\_sg\_gen N\_neu\_sg\_gen\\
NP\_3\_pl\_nom  & $\to$ Det\_fem\_pl\_nom N\_fem\_pl\_nom & NP\_gen  & $\to$ Det\_fem\_pl\_gen N\_fem\_pl\_gen\\
NP\_3\_pl\_nom  & $\to$ Det\_mas\_pl\_nom N\_mas\_pl\_nom & NP\_gen  & $\to$ Det\_mas\_pl\_gen N\_mas\_pl\_gen\\
NP\_3\_pl\_nom  & $\to$ Det\_neu\_pl\_nom N\_neu\_pl\_nom & NP\_gen  & $\to$ Det\_neu\_pl\_gen N\_neu\_pl\_gen\\[2mm]


\ldots & \hspaceThis{$\to$} Dativ                                                             & \ldots & \hspaceThis{$\to$} Akkusativ\\[2mm]
\end{tabular}
}
\item 24 Symbole für Determinatoren, 24 Symbole für Nomen
\item 24 Regeln statt einer
\end{itemize}
}

\subsubsection{Erweiterung der PSG durch Merkmale}


\frame{

\frametitle{Probleme dieses Ansatzes}

\begin{itemize}
\item Gernalisierungen werden nicht erfaßt.
\item weder in Regeln noch in Kategoriesymbolen
      \begin{itemize}
      \item Wo kann eine NP oder NP\_nom stehen?\\
            Nicht wo kann eine NP\_3\_sg\_nom stehen?
      \item Gemeinsamkeiten der Regeln sind nicht offensichtlich.
      \end{itemize}
\pause
\item Lösung: Merkmale mit Werten und Identität von Werten\\
      Kategoriesymbol: NP Merkmal: Per, Num, Kas, \ldots\\

Wir erhalten \zb die Regeln:\\

\begin{tabular}{@{}l@{ }l}
NP(3,sg,nom)  & $\to$ Det(fem,sg,nom) N(fem,sg,nom)\\
NP(3,sg,nom)  & $\to$ Det(mas,sg,nom) N(mas,sg,nom)\\
\end{tabular}
\end{itemize}
}


\frame{
\frametitle{Merkmale und Regelschemata (I)}

\begin{itemize}
\item Regeln mit speziellen Werten zu Regelschemata verallgemeinern:

\medskip

\begin{tabular}{@{}l@{ }l@{ }l}
NP(\blau<3>{3},\blau<2>{Num},\blau<2>{Kas}) & $\to$ & Det(\gruen<2>{Gen},\blau<2>{Num},\blau<2>{Kas}) N(\gruen<2>{Gen},\blau<2>{Num},\blau<2>{Kas})\\
\end{tabular}
\pause
\item Gen-, Num- und Kas-Werte sind egal,\\
      Hauptsache sie stimmen überein (identische Werte)
\pause
\item Der Wert des Personenmerkmals (erste Stelle in NP(3,Num,Kas))\\
 ist durch die Regel festgelegt: 3.
\end{itemize}
}


\frame{
\frametitle{Merkmale und Regelschemata (II)}

\begin{itemize}
\item Regeln mit speziellen Werten zu Regelschemata verallgemeinern:

\medskip
\begin{tabular}{@{}l@{ }l@{ }l}
NP({3},{Num},{Kas}) & $\to$ & Det(Gen,{Num},{Kas}) N(Gen,{Num},{Kas})\\
S  & $\to$ & NP(\blau<1>{Per1},\blau<1>{Num1},\blau<3>{nom})\\
   &       & NP(Per2,Num2,\blau<3>{dat})\\
   &       & NP(Per3,Num3,\blau<3>{akk})\\
   &       & V(\blau<1>{Per1},\blau<1>{Num1})\\\\
\end{tabular}
\item Per1 und Num1 sind beim Verb und Subjekt gleich.
\pause
\item Bei anderen NPen sind die Werte egal.\\
      (Schreibweise für irrelevante Werte: `\_')
\pause
\item Die Kasus der NPen sind in der zweiten Regel festgelegt.
\end{itemize}

}

%% Kommt dann in Theorie anders, deshalb hier raus
%% \frame{

%% \small
%% \frametitle{Bündelung von Merkmalen}

%% \begin{itemize}
%% \item Kann es Regeln geben, in denen nur der Per-Wert oder nur der Num-Wert identisch sein muß?\\[2ex]

%% \begin{tabular}{@{}l@{ }l@{ }l}
%% S  & $\to$ & NP(Per1,Num1,nom)\\
%%    &       & NP(Per2,Num2,dat)\\
%%    &       & NP(Per3,Num3,akk)\\
%%    &       & V(Per1,Num1)\\\\
%% \end{tabular}
%% \pause
%% \item Gruppierung von Information $\to$ stärkere Generalisierung, stärkere Aussage\\[2ex]

%% \begin{tabular}{@{}l@{ }l@{ }l}
%% S  & $\to$ & NP(Agr1,nom)\\
%%    &       & NP(Agr2,dat)\\
%%    &       & NP(Agr3,akk)\\
%%    &       & V(Agr1)\\\\
%% \end{tabular}

%% wobei Agr ein Merkmal mit komplexen Wert ist: \zb agr(1,sg)
%% \end{itemize}


%% }


\iftoggle{hpsgvorlesung}{
\subsubsection{Abstraktion über Regeln: \texorpdfstring{\xbar}{X-Bar}-Theorie}

\frame{
\frametitle{Abstraktion über Regeln}

\xbar"=Theorie \citep{Jackendoff77a}:

\medskip
\oneline{\(
\begin{array}{@{}l@{\hspace{1cm}}l@{\hspace{1cm}}l}
\xbar\mbox{-Regel} & \mbox{mit Kategorien} & \mbox{Beispiel}\\[2mm]
\overline{\overline{\mbox{X}}} \rightarrow \overline{\overline{\mbox{Spezifikator}}}~~\xbar & \overline{\overline{\mbox{N}}} \rightarrow \overline{\overline{\mbox{DET}}}~~\overline{\mbox{N}} & \mbox{das [Bild von Maria]} \\
\xbar \rightarrow \xbar~~\overline{\overline{\mbox{Adjunkt}}}             & \overline{\mbox{N}} \rightarrow \overline{\mbox{N}}~~\overline{\overline{\mbox{REL\_SATZ}}} & \mbox{[Bild von Maria] [das alle kennen]}\\
\xbar \rightarrow \overline{\overline{\mbox{Adjunkt}}}~~\xbar             & \overline{\mbox{N}} \rightarrow \overline{\overline{\mbox{ADJ}}}~~\overline{\mbox{N}} & \mbox{schöne [Bild von Maria]}\\
\xbar \rightarrow \mbox{X}~~\overline{\overline{\mbox{Komplement}}}*               & \overline{\mbox{N}} \rightarrow \mbox{N}~~\overline{\overline{\mbox{P}}} & \mbox{Bild [von Maria]}\\\\
\end{array}
\)}

X steht für beliebige Kategorie, `*' für beliebig viele Wiederholungen

}

\frame{
\frametitle{\xbar-Theorie}

\xbar-Theorie wird in vielen verschiedenen Frameworks angenommen:\\
\begin{itemize}
\item Government \& Binding (GB): \citew*{Chomsky81a}
\item Lexical Functional Grammar (LFG): \citew{Bresnan82a-ed,Bresnan2001a}
\item Generalized Phrase Structure Grammar (GPSG):\\
      \citew*{GKPS85a}
\end{itemize}

}



\subsection{Hausaufgabe}

\frame{
\frametitle{Hausaufgabe}

\begin{enumerate}
\item Schreiben Sie eine Phrasenstrukturgrammatik, mit der man u.\,a.\ die Sätze in (\mex{1})
      analysieren kann, die die Wortfolgen in (\mex{2}) aber nicht zulässt.
      \eal
      \ex[]{
      Der Mann hilft der Frau.
      }
      \ex[]{
      Er gibt ihr das Buch.
      }
      \ex[]{
      Er wartet auf ein Wunder.
      }
%       \ex[]{
%       Er wartet neben dem Bushäuschen auf ein Wunder.
%       }
      \zl
      \eal
      \ex[*]{
        Der Mann hilft er.
      }
      \ex[*]{
        Er gibt ihr den Buch.
      }
      \zl
      Dabei sollen Sie nicht für jeden Satz einzeln eigene Regeln für NP usw.\ aufstellen, sondern gemeinsame Regeln für
      alle aufgeführten Sätze entwickeln.

      Sie können für Ihre Arbeit auch Prolog benutzen: \url{https://swish.swi-prolog.org} zur Syntax
      für die Grammatiken siehe \url{https://en.wikipedia.org/wiki/Definite_clause_grammar}.
\end{enumerate}

}

}




