%% -*- coding:utf-8 -*-

\subtitle{Passiv}

\section{Passiv}

\huberlintitlepage[22pt]

\settowidth\jamwidth {(Isländisch)}


\frame{
\frametitle{Literaturhinweis}



Zu diesem Abschnitt gibt es das Kapitel~6 in \citew{MuellerGermanic}.

Müller, Stefan, \citeyear{MuellerGermanic}. \emph{Germanic Syntax}. Berlin: Language Science
Press. In Vorbereitung. 



}


\subsection{Subjekte}

\frame{
\frametitle{Subjekte (Deutsch)}


\begin{itemize}
\item Was ist ein Subjekt?
\pause
\item Im Deutschen nicht-prädikative Nominalgruppen im Nominativ:
\eal
\ex Der Mann lacht.
\ex Der Mann hilft ihr.
\ex Der Mann gibt ihr ein Buch.
\zl
\pause
\item Was ist mit dem Genitiv und Dativ in (\mex{1})?
\eal
\ex Des Opfers wurde gedacht.
\ex Dem Mann wurde geholfen.
\zl
\pause
Diese werden im Deutschen nicht zu den Subjekten gezählt.

(Bitte lesen Sie \citew[Abschnitt~1.7]{MuellerGTBuch2}, wenn Ihnen das unklar ist.)

\end{itemize}




}


\frame{
\frametitle{Subjekte (Isländisch)}


\begin{itemize}
\item Abfolge für die SVO-Sprache Isländisch ist identisch \citep{ZMT85a}:
\eal
\ex 
\gll Þeim       var hjalpað.\\
     sie.\pl.\dat{} wurde geholfen\\
\ex 
\gll Hennar var saknað.\\
     sie.\sg.\gen{} wurde vermisst\\
\zl

An der Stellung in V2-Sätzen kann man grammatische Funktion nicht ablesen,\\
da die NPen auch vorangestellte Objekte sein könnten.

\pause
\item \citet*{ZMT85a} zeigen, dass es sinnvoll ist,\\
       diese Nicht-Nominative als Subjekte zu behandeln.
\begin{itemize}
\item Kontrollierbarkeit
\item Stellung
\item \ldots
\end{itemize}
\pause
\item Solche Nicht-Nominativ-Subjekte werden auch \alert{schräge Subjekte} genannt,\\
      auch \alert{quirky subjects} oder \alert{oblique subjects}.
\end{itemize}


}


\frame{
\frametitle{Subjekt: Stellung V2-Sätze}


\begin{itemize}
\item Subjekte folgen unmittelbar auf das finite Verb, wenn eine andere Konstituente vorangestellt
  wurde \citep*[Abschnitt~2.3]{ZMT85a}:

\eal
%% \ex[]{
%% \gll Refinn           skaut  Ólafur      með þessari byssu.\\
%%      den.Fuchs.\acc{} schoss Olaf.\nom{} mit diesem  Gewehr\\
%% }
% selbst erfunden, check
\ex[]{
\gll Meb þessari byssu  skaut  \gruen{Ólafur}      \rot{refinn}.\\
     mit diesem  Gewehr schoss Olaf.\nom{} den.Fuchs.\acc{}\\
}
\ex[*]{
\gll Meb þessari byssu  skaut  \rot{refinn}           \gruen{Ólafur}.\\
     mit diesem  Gewehr schoss den.Fuchs.\acc{} Olaf.\nom{}\\
}
\zl
\end{itemize}

}


\frame{
\frametitle{Subjekt: Stellung in V2-Sätzen}


\begin{itemize}

\item Auch in W-Fragen:
\settowidth\jamwidth {(W-Frage)}
\eal
\ex[]{
\gll Hvenær hafði \gruen{Sigga}        hjfilpað \rot{Haraldi}?\\
     wann   hat   Sigga.\nom{} geholfen Harold.\dat{}\\\jambox{(W-Frage)}
}
\ex[*]{
\gll Hvenær hafði \rot{Haraldi} \gruen{Sigga} hjfilpað?\\
     wann   hat   Harald.\dat{} Sigga.\nom{} geholfen\\
}
\zl
\pause
\item Dativobjekt kann vorangestellt werden, dann aber ganz nach vorn:
\ea[]{
\gll \rot{Haraldi}       hafði \gruen{Sigga}        aldrei hjfilpað.\\
     Harald.\dat{} hat   Sigga.\nom{} nie    geholfen\\\jambox{(V2-Satz)}
}
\z

\end{itemize}

}


\frame{
\frametitle{Subjekt: Stellung in V1-Sätzen}


\begin{itemize}

\item Auch in Entscheidungsfragen:
\eal
\ex[]{
\gll Hafði \gruen{Sigga} aldrei hjfilpað \rot{Haraldi}?\\
     hat   Sigga.\nom{}  nie    geholfen Harald.\dat\\
}
\ex[*]{
\gll Hafði \rot{Haraldi}       \gruen{Sigga}        aldrei hjfilpað?\\
     hat   Harald.\dat{} Sigga.\nom{} nie    geholfen\\
}
\zl

\end{itemize}

}

\exewidth{(135)}


\frame[shrink=5]{
\frametitle{Schräge Subjekte und Stellung}
%\smallexamples

\begin{itemize}
\item Geht auch bei bestimmten Dativen:

\eal
\ex[]{
\gll Hefur \gruen{henni}      alltaf þótt    \rot{Ólafur} leibinlegur?\\
     hat   sie.\dat{} immer  gedacht Olaf.\nom{} langweilig.\nom{}\\
\glt `Hat sie Olaf immer für langweilig gehalten?'
}
\ex[]{
\gll \rot{Ólafur} hefur \gruen{henni} alltaf þótt leibinlegur.\\
     Olaf.\nom{} hat sie.\dat{} immer gedacht langweilig.\nom{}\\
\glt `Den Olaf hat sie immer für langweilig gehalten.'
}
\ex[*]{
\gll Hefur \rot{Ólafur} \gruen{henni} alltaf þótt leibinlegur?\\
     hat Olaf.\nom{} sie.\dat{} immer gedacht langweilig.\nom\\
}
\zl

\pause
\item Das Gegenstück im Deutschen wäre:
\ea[??]{
Mich dünkt der Mann langweilig.
}
\z
\emph{dünkt} ist aber archaisch und wird -- wenn überhaupt -- nur mit \emph{dass}-Satz verwendet.
\pause
\item Aber:
\ea
Mir scheint der Mann langweilig.
\z

\end{itemize}

}

\frame{
\frametitle{Subjekte in Kontrollkonstruktionen}

\begin{itemize}
\item Subjekte können in Kontrollkonstruktionen oder bei sogenannter \emph{ärbiträrer Kontrolle} weggelassen werden:
\eal
\ex
\gll Ég  vonast til að fara heim.\\
     ich hoffe  für zu fahren heim\\
\glt `Ich hoffe, nach hause zu fahren.'
\ex
\gll Að fara heim snemma er óvenjulegt.\\
     zu fahren heim früher is ungewöhnlich\\
\glt `Früher nach hause zu fahren, ist ungewöhnlich.'
\zl

\end{itemize}

}


\frame{
\frametitle{Schräge Subjekte in Kontrollkonstruktionen}

\begin{itemize}
\item \emph{vantar} (`fehlen') verlangt zwei Akkusative:
\ea
\gll Mig        vantar peninga.\\
     ich.\acc{} fehlt  Geld.\acc\\
\z
\pause
\item Trotzdem kann das Verb unter \emph{vonast} (`hoffen') eingebettet werden:
\ea
\gll Ég  vonast til ab vanta ekki peninga.\\
     ich hoffe  für zu fehlen nicht Geld.\acc\\
\glt `Ich hoffe, dass mir nicht Geld fehlt.'
\z
\pause
\item Man vergleiche das mit Deutsch:
\ea[*]{
Ich hoffe, kein Geld zu fehlen.
}
\z


\end{itemize}

}


\frame{
\frametitle{Subjekt-Verb-Kongruenz?}

\begin{itemize}
\item Verben kongruieren mit dem Nominativelement.\\
      Wenn es keins gibt, ist das Verb dritte Person Singular (Neutrum).

\pause
\item In (\mex{1}a) keine Kongruenz:
\eal
\ex 
\gll Þeim       var hjalpað.\\
     sie.\blau{\pl}.\dat{} wurde geholfen\\
\ex 
\gll Hennar var saknað.\\
     sie.\blau{\sg}.\gen{} wurde vermisst\\
\zl

\pause
\item Der Dativ und der Genitiv sind aber dennoch Subjekte,\\
      wie wir gleich sehen werden.

\end{itemize}

}


\frame{
\frametitle{Schräge Subjekte im Passiv: Stellung}


\begin{itemize}
\item Dativ folgt Finitum in V1:

\ea
\gll Var \gruen{honum} aldrei hjfilpað af foreldrum sinum?\\
     war er.\dat{} nie geholfen von Eltern seinen\\
\glt `Wurde ihm nie von seinen Eltern geholfen?'
\z

\pause
\item Dativ folgt Finitum in V2:

\ea
\gll Í prófinu  var \gruen{honum} vist hjálpað.\\
     in Prüfung war er.\dat{} scheinbar geholfen\\
\glt `Anscheinend wurde ihm bei der Prüfung geholfen.'
\z
\end{itemize}

}




\frame{
\frametitle{Schräge Subjekte im Passiv: Kontrolle}


\begin{itemize}
\item Das Dativ-Subjekt kann unterdrückt werden:
\eal
\ex
\gll Ég vonast til að verba hjálpað.\\
     ich hoffe für zu werden  geholfen\\
\ex
\gll Að vera hjálpað i prófinu er óleyfilegt.\\
     zu werden geholfen in Prüfung ist unzulässig\\
\glt `Es ist verboten, sich in der Prüfung helfen zu lassen.'\\
(`Es ist verboten in der Prüfung geholfen zu bekommen.')
\zl

\pause
\item Man vergleiche:
\ea[*]{
Ich hoffe geholfen zu werden.
}
\z
\pause
\item Geht nur mit \emph{bekommen}-Passiv:
\ea[]{
Ich hoffe hier geholfen zu bekommen.\footnote{
\url{http://www.photovoltaikforum.com/sds-allgemein-ueber-solar-log-f38/solarlog-1000-mit-wifi-anschliesen-t96371.html}. 10.01.2014
}
}
\z


\end{itemize}

}



\subsection{Der Kasus von Argumenten: Struktureller und lexikalischer Kasus}
\label{sec-struk-lex-kas}
\label{sec-struc-lex-kas}

\frame{
\frametitle{Kasus und Kasusprinzipien}

\begin{itemize}
\item Welche Arten von Kasus gibt es?
\pause
\item Wie hängen Kasus vom syntaktischen Kontext ab?
\pause
\item Bisher wird Kasus in Valenzlisten festgelegt,\\
      wenn wir die Gesetzmäßigkeiten kennen, muss das nicht mehr sein.

Erfassen Generalisierungen und\\
brauchen nur einen Lexikoneintrag für das Verb \emph{lesen} in (\mex{1}):
\eal
\ex \blau{Er} möchte das Buch lesen.
\ex Ich sah \blau{ihn} das Buch lesen.
\zl

Kasus des Subjekts (und des Objekts) wird durch Prinzip geregelt.
\end{itemize}

}



\subsubsection{Struktureller Kasus}

\frame{
\frametitle{Struktureller und lexikalischer Kasus}

\begin{itemize}
\item Wenn Kasus von Argumenten von der syntaktischen Umgebung abhängt,\\
      spricht man von \blau{strukturellem Kasus}.\\
      Ansonsten haben die Argumente \blau{lexikalischen Kasus}.
\pause
\item Beispiele für strukturellen Kasus sind:
\eal
\ex \blau{Der Installateur} kommt.
\pause
\ex Der Mann läßt \blau{den Installateur} kommen.
\pause
\ex das Kommen \blau{des Installateurs}
\zl

\pause
\item In (\mex{0}) wird der Kasus des Subjekts von \emph{kommen} verschieden ausgedrückt, in
  (\mex{1}) der Kasus des Objekts von \emph{schlagen}:
\eal
\ex Judit schlägt \blau{den Weltmeister}.
\ex \blau{Der Weltmeister} wird geschlagen.
\zl
\end{itemize}
}

\subsubsection{Lexikalischer Kasus}

%\subsubsection{Genitiv}

\frame{
\frametitle{Lexikalische Kasus}

\begin{itemize}
\item vom Verb abhängiger Genitiv ist lexikalischer Kasus:\\
Bei Passivierung ändert sich der Kasus eines Genitivobjekts nicht.
\eal
\ex[]{
Wir gedenken \blau{der Opfer}.
}
\ex[]{
\blau{Der Opfer} wird gedacht.
}
\ex[*]{
\blau{Die Opfer} wird/werden gedacht.
}
\zl
\pause
(\mex{0}b) = unpersönliches Passiv, es gibt kein Subjekt.
\end{itemize}

}

%\subsubsection{Dativ}

\frame{
\frametitle{Der Dativ ein lexikalischer Kasus?}

\begin{itemize}
\item Genauso gibt es keine Veränderungen bei Dativobjekten:
\eal
\ex Der Mann hat \blau{ihm} geholfen.
\ex \blau{Ihm} wird geholfen.
\zl
\pause
\item Aber was ist mit (\mex{1})?
\eal
\ex Der Mann  hat   den Ball \blau{dem Jungen} geschenkt.
\ex \blau{Der Junge} bekam den Ball            geschenkt.
\zl
\pause
\item Die Einordnung des Dativs wird kontrovers diskutiert.\\
Drei Möglichkeiten für Dativargumente:
\begin{enumerate}[<+->]
\item Hypothese 1: Alle Dative sind lexikalisch.
\item Hypothese 2: Einige Dative sind lexikalisch, andere strukturell.
\item Hypothese 3: Alle Dative sind strukturell.
\end{enumerate}
\end{itemize}

}

\frame{
\frametitle{Hypothese 1: Alle Dative sind lexikalisch}

\begin{itemize}
\item Wenn man den Dativ als lexikalischen Kasus behandelt,
      muß man beim Dativpassiv eine Umwandlung von lex.\ in str.\ Kasus annehmen.
\pause
\item Haiders Beispiele in (\mex{1}) sind dann sofort erklärt \citeyearpar[S.\,20]{Haider86}:
\eal
\ex[]{\iw{streicheln}
Er       streichelt \blau{den Hund}.
}
\ex[]{
\blau{Der Hund} wurde gestreichelt.
}
\ex[]{
sein Streicheln \blau{des    Hundes}
}
\pause
\ex[]{\label{bsp-er-hilft-den-kindern}\iw{helfen}
Er hilft \blau{den Kindern}.
}
\ex[]{
\blau{Den Kindern} wurde geholfen.
}
\ex[]{
das Helfen \blau{der Kinder} (Kinder nur Agens)
}\label{das-helfen-der-Kinder}
\ex[*]{
sein Helfen \blau{der Kinder}
}\label{sein-helfen-der-Kinder}
\zl
\pause
\item Dativ kann nur pränominal ausgedrückt werden:
\ea
das \blau{Den-Kindern}-Helfen
\z

\end{itemize}
}

\frame{
\frametitle{Hypothese 2: Manche Dative sind strukturell, Bivalente Verben}

\begin{itemize}
\item Wenn man allein die Unterscheidung strukturell/lexikalisch hat,\\
      bekommt man bei bivalenten Verben ein Problem:
\eal
\ex Er hilft ihm.
\ex Er unterstützt ihn.
\zl
Die Information im Lexikoneintrag von \emph{helfen} 
und \emph{unterstützen} muß sich unterscheiden.
\pause
\item Bei ditransitiven Verben kann man Kasus aus allgemeinen
      Prinzipien ableiten (Nom, Dat, Acc), aber bei bivalenten
      geht das nicht.

$\to$ Dativ bei \emph{helfen} wird als lexikalisch eingeordnet, aber Dativ bei \emph{geben} als strukturell.

Vorhersage: Dativpassiv ist mit diesen Verben nicht möglich.

\end{itemize}
}

\frame{
\frametitle{Hypothese 2: Das Dativpassiv mit bivalenten Verben}

\savespace
\eal
\ex Er kriegte von vielen geholfen / gratuliert / applaudiert.
\ex Man kriegt täglich gedankt.
\zl

\pause
Die Beispiele in (\mex{1}) sind Korpusbelege:
\eal
\ex "`Da kriege ich geholfen."'\footnote{
Frankfurter Rundschau, 26.06.1998, S.\,7.%
}
\ex
% auch nach applaudiert geholfen + bekommen und kriegen gesucht 21.09.2003
Heute morgen bekam ich sogar schon gratuliert.\footnote{%
Brief von Irene G.\ an Ernst G.\ vom 10.04.1943, Feldpost-Archive mkb-fp-0270}
%Branich IG-Vorsitzender Friedel Schönel meinte deshalb, 
\ex
"`Klärle"' hätte es wirklich mehr als verdient, auch mal zu einem "`unrunden"' Geburtstag gratuliert zu bekommen.\footnote{
Mannheimer Morgen, 28.07.1999, Lokales; "`Klärle"' feiert heute Geburtstag.%
}
\ex
Mit dem alten Titel von Elvis Presley "`I can't help falling in love"' bekam Kassier Markus Reiß zum Geburtstag gratuliert, [\ldots]\footnote{
%der dann noch viel später bekannte: "Ich hab' immer noch Gänsehaut, das war der schönste Teil meines Geburtstages." Doch auch die anderen Abteilungen des Bürgervereins können auf ein erfolgreiches Jahr 1998 zurückblicken.
Mannheimer Morgen, 21.04.1999, Lokales; Motor des gesellschaftlichen Lebens.%
}
\zl

Also: Dativpassiv muss irgendwie anders gemacht werden, wenn Dativ bei zweistelligen Verben lexikalisch ist.

Dann kann man gleich Hypothese 1 annehmen: Alle Dative sind lexikalisch.

}

%% \subsubsection{Akkusativ}

%% \frame{
%% \frametitle{Akkusativ}

%% Neben der Möglichkeit des strukturellen Akkusativs gibt es auch lexikalische Akkusative:
%% \eal
%% \ex \blau{Ihn} dürstet.
%% \ex Der Vater lehrte seinen Sohn \blau{einen neuen Tritt}.
%% \zl

%% \ea
%% Die Söhne wurden einen neuen Tritt gelehrt.
%% }

%% \subsection{Adjektivumgebungen}

%% \frame{
%% \frametitle{Lexikalischer Kasus in Adjektivumgebungen}

%% Kasus von Objekten von Adjektiven kann sich nicht ändern.\\
%% Adjektive können Genitiv und Dativ zuweisen:
%% \eal
%% \ex Ich war mir \blau{dessen} sicher.
%% \ex Sie ist \blau{ihm} treu.
%% \zl
%% \pause
%% Die Zuweisung von Akkusativ ist ebenfalls möglich:
%% \eal
%% \ex Das ist \blau{diesen Preis} nicht wert.
%% \ex Der Student ist \blau{das Leben} im Wohnheim nicht gewohnt.\iw{gewohnt}\footnote{
%%         \citep*[S.\,312]{HB72a}
%%       }
%% \ex Du bist mir \blau{eine Erkl"arung} schuldig.\footnote{
%%         \citep*[S.\,620]{HFM81}
%%       }
%% \zl
%% Akkusativ ist bei Adjektivkomplementen aber selten \citep{Haider85b}.
%% }

%% \frame{
%% \frametitle{Struktureller Kasus in Adjektivumgebungen}


%% Kasus der Subjekte von Adjektiven hängt von der syntaktischen
%% Umgebung ab \citep{Wunderlich84}:
%% \eal
%% \ex \blau{Der Mond} wurde kleiner.\iw{klein}
%% \ex Er sah\iw{sehen} \blau{den Mond} kleiner werden.
%% \zl

%% }


%% \subsection{Semantische Kasus}
%% \label{sec-sem-kasus}
%% \is{Kasus!semantischer|(}

%% \frame{
%% \frametitle{Semantische Kasus}

%% \begin{itemize}
%% \item NPen können auch als Adjunkte auf"|treten \citep{Haider85b}:
%% \eal
%% \ex Sie hörten \blau{den ganzen Tag} dieselbe Schallplatte.
%% \ex Laßt \blau{mir} den Hund in Ruhe!
%% \ex \blau{Eines Tages} erschien ein Fremder.
%% \zl
%% \pause
%% \item auch der Urteilsdativ ({\it Dativ iudicantis})  \citep{Wegener85b}:

%% \eal
%% \ex Das ist \blau{mir} zu\iw{zu!Grad} schwer.
%% \ex Das ist \blau{dem Kind} zu langweilig / nicht interessant genug.\iw{genug!Grad}
%% \ex Du läufst \blau{der Oma} zu\iw{zu!Grad} schnell.
%% \ex Das Wasser ist \blau{dem Baby} warm genug.\iw{genug!Grad}
%% \zl
%% \end{itemize}
%% }

%% \frame[shrink=10]{
%% \frametitle{Zuweisung semantischer Kasus durch das Verb?}

%% \begin{itemize}
%% \item
%% Haider: Zuweisung durch Verb in (\mex{1}) nicht sinnvoll:
%% \ea
%% Sie hörten \blau{den ganzen Tag} dieselbe Schallplatte.
%% \z
%% Zeitangaben kommen auch in adjektivischen und nominalen Umgebungen vor:
%% % zitiert Toman83
%% \eal
%% \ex die Ereignisse \blau{letzten Sommer}
%% \ex der Flirt \blau{vorigen Dienstag}
%% \ex die \blau{diesen Sommer} sehr günstige Witterung
%% \ex die \blau{diesen Sommer} sehr teuren Urlaubsreisen
%% \zl
%% NPen mit strukturellem Kasus müssen in Nominalumgebungen
%% Genitiv sein. $\to$\\
%% In (\mex{0}) keine Zuweisung von strukturellem Kasus.

%% \pause
%% \item
%% Die Kasus in (\mex{0}) werden nicht aufgrund ihres Vorkommens in einer bestimmten
%% syntaktischen Struktur zugewiesen,\\
%% sondern sind vielmehr durch die Bedeutung des Nomens bestimmt.
%% \end{itemize}
%% }

%% \frame{
%% \frametitle{Akkusativ und Genitiv}

%% %\citep*{ZMT85a} -> semantische Kasusmarkierung
%% Der freie Akkusativ kommt bei Maß"-angaben\is{Maßangaben} (Zeitdauer und Zeitpunkt)
%% vor (\mex{1}) und Genitiv bei Lokalangaben oder Zeitangaben (\mex{2}).
%% \eal
%% \ex Sie studierte \blau{den ganze Abend}.
%% \ex \blau{Nächsten Monat}\iw{Monat} kommen wir.
%% \zl
%% \eal
%% \ex Ein Mann kam \blau{des Weges}.\iw{Weg}
%% \ex \blau{Eines Tages}\iw{Tag} sah ich sie wieder.
%% \zl
%% }




%% \subsection{Kongruenzkasus}

%% \frame{
%% \frametitle{Kongruenzkasus}

%% \begin{itemize}
%% \item Zwei Akkusative?
%% \eal
%% \ex Er nannte \blau{ihn} \rot{einen Experten}.
%% \ex \blau{Er} wurde \rot{ein Experte} genannt.
%% \zl
%% \pause
%% \item Wären das zwei unabhängige Akkusative,\\
%%       würde sich bei Passivierung nur einer ändern.

%% \pause
%% \item Kasus von \emph{einen Experten} wird \blau{Kongruenzkasus} genannt.\\
%% Die prädikative Phrase \emph{einen Experten} stimmt mit dem
%% Element,\\ über das prädiziert wird, im Kasus überein. 
%% \end{itemize}
%% }

%% \frame{
%% \frametitle{Kongruenzkasus mit Präpositionen}

%% Ähnliche Effekte kann man mit den Präpositionen \emph{als} und \emph{wie}
%% beobachten.
%% \eal
%% \ex \blau{Er} gilt als \rot{großer Künstler}.\footnote{
%%         \citew[S.\,203--204]{Heringer73a}.
%%       }
%% \ex Man läßt \blau{ihn} als \rot{großen Künstler} gelten\iw{gelten als}.
%% \zl
%% \pause
%% \eal
%% \ex Ich sehe \blau{ihn} als \rot{meinen Freund} an.\iw{ansehen}\footnote{
%%         \citew*[S.\,154]{SS88a}.
%% }
%% \ex \blau{Er} wird als \rot{mein Freund} angesehen.
%% \zl
%% }

%% \frame{
%% \frametitle{Kongruenzkasus mit Adjunkten}

%% Wie bei den prädikativen Argumenten gibt es auch Kongruenzkasus bei Adjunkten:
%% \eal
%% \ex Sie verhielt\iw{verhalten} \blau{sich} wie \blau{ihr Vater}.
%% \ex Ich behandelte\iw{behandeln} \blau{ihn} wie \blau{meinen Bruder}.
%% \ex Ich half\iw{helfen} \blau{ihm} wie \blau{einem Freund}.
%% \ex Ich erinnerte\iw{erinnern} mich \blau{dessen} wie \blau{eines fernen Alptraums}.
%% \zl

%% }

%% \frame{
%% \frametitle{Prädikation = Kasuskongruenz?}

%% \begin{itemize}
%% \item Kongruieren prädikative Phrasen immer mit dem Element,\\
%%       über das sie prädizieren?
%% \item Dies würde sofort auch Beispiele wie das in (\mex{1}) erklären:
%% \ea
%% Er wird ein großer Linguist.
%% \z
%% \pause
%% \item In AcI"=Konstruktionen müßten beide NPen im Akkusativ stehen.\\
%% Das ist nicht der Fall:
%% \eal
%% %\ex Laß ihn einen großen Linguisten werden.\label{bsp-lass-ihn-einen-grossen}
%% \ex Laß\iw{lassen|(} den wüsten Kerl [\ldots] meinetwegen ihr Komplize sein.\footnote{
%%         (\ref{bsp-lass-den-wuesten-kerl}) und (\ref{bsp-lass-mich}) sind aus dem \citet*[{\S}\,6925]{Duden66}.\iaf{Duden} %\citet*[{\S}\,1473]{Duden73}.\iaf{Duden}
%%         Die Quellen finden sich dort.
%%       }\label{bsp-lass-den-wuesten-kerl}
%% \ex Laß mich dein treuer Herold sein.\label{bsp-lass-mich}
%% \ex Baby, laß\iw{lassen|)} mich dein Tanzpartner sein.\footnote{
%%         Funny van Dannen, Benno-Ohnesorg-Theater, Berlin, Volksbühne, 11.10.1995
%%         }
%% \zl
%% \pause
%% \item
%% $\to$ Nominativ des Nicht-Subjekts in Kopulakonstruktionen ist\\
%%       ein lexikalischer Kasus \citep[S.\,54]{Thiersch78a}.
%% \end {itemize}
%% }

%% \subsection{Der Kasus nicht ausgedrückter Subjekte}
%% \label{sec-kasus-nicht-realisierter-subj}

%% \frame{
%% \frametitle{Der Kasus nicht ausgedrückter Subjekte (I)}
%% \savespace

%% \begin{itemize}
%% \item \citet*[Kapitel~6]{Hoehle83}:\\
%% Kasus nicht an der Oberfläche auf"|tretender Elemente bestimmbar.

%% {\em ein- nach d- ander-\/} kann sich auf mehrzahlige Konstituenten beziehen. 

%% Dabei muß Kasus und Genus mit der Bezugsphrase übereinstimmen.
%% \pause
%% \item In (\mex{1}) Bezug auf Subjekte bzw.\ Objekte:
%% \eal
%% \ex Die Türen sind eine nach der anderen kaputtgegangen.
%% \ex Einer nach dem anderen haben wir die Burschen runtergeputzt.
%% \ex Einen nach dem anderen haben wir die Burschen runtergeputzt.
%% \ex Ich ließ die Burschen einen nach dem anderen einsteigen.
%% \ex Uns wurde einer nach der anderen der Stuhl vor die Tür gesetzt.
%% \zl
%% \end{itemize}
%% }

%% \frame{
%% \frametitle{Der Kasus nicht ausgedrückter Subjekte (II)}
%% \savespace

%% In (\mex{1}) Bezug auf Dativ- bzw.\ Akkusativobjekte
%% eingebetteter Infinitive:

%% \eal
%% \ex Er hat uns gedroht, die Burschen demnächst einen nach dem anderen wegzuschicken.
%% \ex Er hat angekündigt, uns dann einer nach der anderen den Stuhl vor die Tür zu setzen.
%% \ex Es ist nötig, die Fenster, sobald es geht, eins nach dem anderen auszutauschen.
%% \zl

%% }

%% \frame{
%% \frametitle{Der Kasus nicht ausgedrückter Subjekte (III)}
%% \savespace

%% In (\mex{1}) Bezug auf Subjekt innerhalb der Infinitiv"=VP:
%% \eal
%% \ex Ich habe den Burschen geraten, im Abstand von wenigen Tagen einer nach dem anderen
%%       zu kündigen.
%% \ex Die Türen sind viel zu wertvoll, um eine nach der anderen verheizt zu werden.
%% \ex Wir sind es leid, eine nach der anderen den Stuhl vor die Tür gesetzt zu kriegen.
%% \ex Es wäre fatal für die Sklavenjäger, unter Kannibalen zu fallen und einer nach dem
%%       anderen verspeist zu werden.
%% \zl
%% {\em ein- nach d- ander-\/} im Nominativ $\to$\\
%% Das nicht realisierte Subjekt steht ebenfalls im Nominativ.

%% }


%% \frame{
%% \frametitle{Der Kasus nicht ausgedrückter Subjekte (IV)}

%% Dasselbe gilt für nicht realisierte Subjekte von adjektivischen Partizipien:
%% \eal
%% \ex die eines nach dem anderen einschlafenden Kinder
%% \ex die einer nach dem anderen durchstartenden Halbstarken
%% \ex die eine nach der anderen loskichernden Frauen
%% \zl
%% }

%% \frame{
%% \frametitle{Der Kasus nicht ausgedrückter Subjekte (V)}

%% Man muß also sicherstellen, daß auch nicht realisierte Subjekte Kasus zugewiesen bekommen.
%% Würde man diesen Kasus unterspezifiziert lassen, würden Sätze wie (\mex{1}) falsch analysiert werden.
%% \judgewidth{\#}
%% \ea[\#]{
%% Ich habe den Burschen geraten, im Abstand von wenigen Tagen einen nach dem anderen zu kündigen.
%% }
%% \z
%% In der zulässigen Lesart von (\mex{0}) ist die Phrase 
%% \emph{einen nach dem anderen} das Objekt von \emph{kündigen} und kann
%% sich nicht auf das Subjekt des Infinitivs, das referenzidentisch
%% mit \emph{den Burschen} ist, beziehen.

%% }





\subsection{Kasus von Argumenten}

\subsubsection{Das Kasusprinzip}

\frame{
\frametitle{Das Kasusprinzip (I)}

\begin{itemize}
\item Dativ wird als lexikalischer Kasus angesehen.
\pause
\item Alle Argumente werden in allen Sprachen in einer Liste repräsentiert.\\
      \textsc{argument-structure}-Liste bzw.\ \argst.
\pause
\item ditransitives Verb wie \word{geben} hat den \argstw: 
\ea
\sliste{ NP[\str], NP[\ldat], NP[\str] }
\z
\type{str} steht für strukturellen Kasus und \type{ldat} für lexikalischen Dativ.
\pause
\item Für SVO-Sprachen ist erstes Argument das Subjekt (\spr), die anderen \comps.

Bei den SOV-Sprachen sind bei finiten Verben alle \argst-Elemente in \comps.
\pause
\end{itemize}
}


\frame{
\frametitle{Das Kasusprinzip (II)}

\begin{itemize}

\item
Die Zuweisung struktureller Kasus wird durch das folgende Prinzip geregelt (\citealp{Prze99}; 
\citealp{Meurers99b}):


\begin{prinzip-break}[\hypertarget{case-p}{Kasusprinzip}]\is{Prinzip!Kasus-}
\label{case-p}
\begin{itemize}
\item In einer Liste, die sowohl das Subjekt als auch die Komplemente eines verbalen Kopfes
      enthält, bekommt das am weitesten links stehende Element mit strukturellem Kasus
      Nominativ\is{Kasus!Nominativ}, es sei denn es wird von einem übergeordneten Kopf angehoben.
\item Alle anderen nicht angehobenen Elemente der Liste, die strukturellen Kasus tragen, bekommen Akkusativ\is{Kasus!Akkusativ}.
\item In nominalen Umgebungen wird Elementen mit strukturellem Kasus Genitiv\is{Kasus!Genitiv} zugewiesen.
\end{itemize}
\end{prinzip-break}

\bigskip
\item Prinzip geht auf \citet*{YMJ87} zurück.
\end{itemize}
}

%% \frame{
%% \frametitle{Das Kasusprinzip (III)}

%% \begin{itemize}[<+->]
%% \item Prinzip ähnelt sehr stark dem von \citet*{YMJ87} und kann
%% damit auch die Kasussysteme verschiedener Sprachen erklären,\\
%% die von den genannten Autoren besprochen wurden,\\
%% insbesondere auch das komplizierte Kasussystem des Isländischen\il{Isländisch}.
%% \item
%% Ein wesentlicher Unterschied ist, daß das Prinzip~\ref{case-p} monoton ist,\\
%% \dash Kasus, die einmal zugewiesen wurden,\\
%% werden nicht von einem übergeordneten Prädikat überschrieben.
%% \end{itemize}

%% }

\subsubsection{Aktiv}

\frame{
\frametitle{Aktiv}

prototypische Valenzlisten:
\ea
\begin{tabular}[t]{@{}l@{~}l@{~}l}
a. & \emph{schläft}:     & \argst \sliste{ NP[\type{str}]$_i$ }\\
b. & \emph{unterstützt}: & \argst \sliste{ NP[\type{str}]$_i$, NP[\type{str}]$_j$ }\\
c. & \emph{hilft}:       & \argst \sliste{ NP[\type{str}]$_i$, NP[\type{ldat}]$_j$ }\\
d. & \emph{schenkt}:     & \argst \sliste{ NP[\type{str}]$_i$, NP[\type{ldat}]$_j$, NP[\type{str}]$_k$ }\\
\end{tabular}
\z
\pause
Das erste Element in der \argstl bekommt Nominativ.\\
Alle anderen mit strukturellem Kasus bekommen Akkusativ.

\pause
Für den Vergleich mit dem Passiv ist es sinnvoll,\\
die NPen mit kleinen Indizes zu versehen (i, j, k).

}

\subsubsection{Passiv}

\frame[shrink=5]{
\frametitle{Passiv}

\ea
\begin{tabular}[t]{@{}l@{~}l@{~}l}
a. & \emph{schläft}:     & \argst \sliste{ NP[\type{str}]$_i$ }\\
b. & \emph{unterstützt}: & \argst \sliste{ NP[\type{str}]$_i$, NP[\type{str}]$_j$ }\\
c. & \emph{hilft}:       & \argst \sliste{ NP[\type{str}]$_i$, NP[\type{ldat}]$_j$ }\\
d. & \emph{schenkt}:     & \argst \sliste{ NP[\type{str}]$_i$, NP[\type{ldat}]$_j$, NP[\type{str}]$_k$ }\\
\end{tabular}
\z

Bei Passivierung der Verben ergeben sich die folgenden \argst"=Listen:
\ea
\begin{tabular}[t]{@{}l@{~}l@{~}l}
a. & \emph{geschlafen wird}:  & \argst \sliste{ }\\
b. & \emph{unterstützt wird}: & \argst \sliste{ NP[\type{str}]$_j$ }\\
c. & \emph{geholfen wird}:    & \argst \sliste{ NP[\type{ldat}]$_j$ }\\
d. & \emph{geschenkt wird}:   & \argst \sliste{ NP[\type{ldat}]$_j$, NP[\type{str}]$_k$ }\\
\end{tabular}
\z
In (\mex{0}) steht jetzt eine andere NP an erster Stelle.\\
Erste NP mit strukturellen Kasus bekommt sie Nominativ.\\
Lexikalischer Kasus wie in (\mex{0}c--d) bleibt wie er ist,
nämlich lexikalisch spezifiziert.
}

%% \subsubsubsection{Dativpassiv}

%% \frame[shrink=15]{
%% \frametitle{Dativpassiv}

%% Bei der Kombination von \emph{geholfen} und
%% \emph{bekommen} bzw.\ von \emph{geschenkt} und \emph{bekommen} wird das Dativargument von 
%% \emph{geholfen} bzw.\ von \emph{geschenkt} zum ersten Argument gemacht und der lexikalische
%% Dativ beim eingebetteten Verb wird zu einem strukturellen Kasus beim Passiv"=Hilfsverb:
%% \ea
%% \begin{tabular}[t]{@{}l@{~}l@{~}l}
%% c. & \emph{hilft}:       & \argst \sliste{ NP[\type{str}]$_j$, NP[\type{ldat}]$_k$ }\\
%% d. & \emph{schenkt}:     & \argst \sliste{ NP[\type{str}]$_j$, NP[\type{str}]$_k$, NP[\type{ldat}]$_l$ }\\
%% \end{tabular}
%% \z
%% \ea
%% \begin{tabular}[t]{@{}l@{~}l@{~}l}
%% a. & \emph{geholfen bekommt}:    & \argst \sliste{ NP[\type{str}]$_k$ }\\
%% b. & \emph{geschenkt bekommt}:   & \argst \sliste{ NP[\type{str}]$_l$, NP[\type{str}]$_k$ }\\
%% \end{tabular}
%% \z
%% Details kommen im Kapitel über Passiv.

%% Kasusvergabe: Dadurch, daß das Dativargument an erster Stelle in der Valenzliste\\
%% von \emph{geholfen bekommen} bzw.\ von \emph{geschenkt bekommen} steht, kriegt es
%% Nominativ. 

%% Bei \emph{geschenkt bekommen} bekommt das zweite Element (das direkte Objekt) Akkusativ.

%% Die Umwandlung eines lexikalischen in einen strukturellen Kasus ist unschön,\\
%% es scheint zur Zeit jedoch keine bessere Alternative zu geben. 

%% }

%% \frame{
%% \frametitle{AcI-Konstruktionen (I)}
%% \smallframe

%% Bei der Analyse der AcI"=Konstruktion findet eine Argumentkomposition statt:\\
%% die Argumente des eingebetteten Verbs werden zu Argumenten des AcI"=Verbs:

%% \ea
%% \begin{tabular}[t]{@{}l@{~}l@{~}l}
%% a. & \emph{schläft}:     & \argst \sliste{ NP[\type{str}]$_j$ }\\
%% b. & \emph{unterstützt}: & \argst \sliste{ NP[\type{str}]$_j$, NP[\type{str}]$_k$ }\\
%% c. & \emph{hilft}:       & \argst \sliste{ NP[\type{str}]$_j$, NP[\type{ldat}]$_k$ }\\
%% d. & \emph{schenkt}:     & \argst \sliste{ NP[\type{str}]$_j$, NP[\type{str}]$_k$, NP[\type{ldat}]$_l$ }\\
%% \end{tabular}
%% \z
%% \ea
%% %{\small
%% \begin{tabular}[t]{@{}l@{~}l@{~}l@{}}
%% a. & \emph{schlafen läßt}:     & \argst \sliste{ NP[\str]$_i$, NP[\type{str}]$_j$ }\\
%% b. & \emph{unterstützen läßt}: & \argst \sliste{ NP[\str]$_i$, NP[\type{str}]$_j$, NP[\type{str}]$_k$ }\\
%% c. & \emph{helfen läßt}:       & \argst \sliste{ NP[\str]$_i$, NP[\type{str}]$_j$, NP[\type{ldat}]$_k$ }\\
%% d. & \emph{schenken läßt}:     & \argst \sliste{ NP[\str]$_i$, NP[\type{str}]$_j$, NP[\type{str}]$_k$, NP[\type{ldat}]$_l$ }\\
%% \end{tabular}
%% %}
%% \z

%% NP[\str]$_i$ steht dabei jeweils für das Subjekt des AcI-Verbs.\\ 
%% NP[\type{str}]$_j$, NP[\type{str}]$_k$ bzw.\ NP[\type{ldat}]$_l$ sind die Argumente des eingebetteten
%% Verbs. 
%% }

%% \frame{
%% \frametitle{AcI-Konstruktionen (II)}
%% \smallframe

%% \ea
%% %{\small
%% \begin{tabular}[t]{@{}l@{~}l@{~}l@{}}
%% a. & \emph{schlafen läßt}:     & \argst \sliste{ NP[\str]$_i$, NP[\type{str}]$_j$ }\\
%% b. & \emph{unterstützen läßt}: & \argst \sliste{ NP[\str]$_i$, NP[\type{str}]$_j$, NP[\type{str}]$_k$ }\\
%% c. & \emph{helfen läßt}:       & \argst \sliste{ NP[\str]$_i$, NP[\type{str}]$_j$, NP[\type{ldat}]$_k$ }\\
%% d. & \emph{schenken läßt}:     & \argst \sliste{ NP[\str]$_i$, NP[\type{str}]$_j$, NP[\type{str}]$_k$, NP[\type{ldat}]$_l$ }\\
%% \end{tabular}
%% %}
%% \z

%% Für die Kasusvergabe sind nur die Valenzlisten in (\mex{0}) relevant. 

%% Die Argumente in den Valenzlisten der eigentlichen Verben spielen für die Kasusvergabe keine Rolle, 
%% da das Kasusprinzip die Kasuszuweisung ausschließt,
%% wenn ein Element angehoben wird. 

%% Das erste Element in den Listen in (\mex{0}) bekommt immer Nominativ,\\
%% die restlichen Elemente mit strukturellem Kasus bekommen Akkusativ. 

%% Die logischen Subjekte der eingebetteten V werden also im Akkusativ realisiert.

%% }

%% \frame{
%% \frametitle{Adjektivsubjekte}


%% Die Kasuszuweisungen an das Subjekt von Adjektiven funktioniert analog. Die Kopula wird mit dem Adjektiv
%% verbunden, und es entsteht eine Valenzliste, die die Argumente des Adjektivs enthält (\mex{1}a).\\
%% Wird ein solcher Komplex noch unter ein AcI"=Verb wie \emph{sehen} eingebettet,\\
%% erhält man die Liste in (\mex{1}b):
%% \ea
%% \begin{tabular}[t]{@{}l@{~}l@{~}l}
%% a. & \emph{kleiner werden}:     & \argst \sliste{ NP[\str]$_j$ }\\
%% b. & \emph{kleiner werden sah}: & \argst \sliste{ NP[\str]$_i$, NP[\type{str}]$_j$ }\\
%% \end{tabular}
%% \z
%% Die Kasuszuweisung funktioniert analog zu den bereits diskutierten Fällen. In den verbalen Umgebungen
%% der Kopula bzw.\ des AcI"=Verbs bekommen die NPen mit strukturellem Kasus Nominativ bzw.\ Akkusativ.%
%% }


%% \subsection{Semantischer Kasus}
%% \frame{
%% \frametitle{Semantischer Kasus (I)}


%% Der Kasus von NPen wie \emph{den ganzen Tag} in (\mex{1}) ist von der syntaktischen Umgebung unabhängig.
%% \eal
%% \ex Sie arbeiten den ganzen Tag.
%% \ex Den ganzen Tag wird gearbeitet, [\ldots].\footnote{
%%   \url{http://www.philo-forum.de/philoforum/viewtopic.html?p=146060}. \urlchecked{12}{05}{2005}.
%% }
%% \zl
%% Daß die NP im Akkusativ steht, hängt mit ihrer Funktion zusammen. 

%% Unterschiedliche Lexikoneinträge für \emph{Tag} in (\mex{0}) und (\mex{1}):
%% \eal
%% \ex Ich liebe diesen Tag.
%% \ex Dieser Tag gefällt mir.
%% \zl

%% In (\mex{0}) liegen ganz gewöhnliche Argumente vor,\\
%% in (\mex{-1}) dagegen ein Adjunkt. 
%% }

%% \frame{
%% \frametitle{Semantischer Kasus (II)}

%% Adjunkte unterscheiden sich von Argumenten durch ihren \modw und durch ihern \contw.

%% Für (\mex{-1}) muß es unter \cont eine Dauer-Relation geben.

%% Zusammen mit dieser Information wird im Lexikoneintrag für das modifizierende Nomen der Kasus fest kodiert. 

%% Die morphologische Komponente kann dann für diesen Lexikoneintrag nur die Akkusativform erzeugen, 
%% da alle anderen Flexionsformen mit der bereits im Lexikoneintrag angegebenen Kasusinformation inkompatibel sind. 

%% Dadurch wird sichergestellt, daß Sätze wie (\mex{1}) nicht analysiert werden:
%% \eal
%% \ex[*]{
%% Er arbeitet der ganze Tag.
%% }
%% \ex[*]{
%% weil der ganze Tag gearbeitet wurde
%% }
%% \zl

%% }


\frame{
\frametitle{Vergleich Deutsch, Dänisch, Englisch, Isländisch}

Vergleich Deutsch, Dänisch, Englisch, Isländisch:

\begin{itemize}
\item Deutsch, Isländisch erlauben subjektlose Konstruktionen,\\
      Dänisch und Englisch nicht
\pause
\item Deutsch, Isländisch und Dänisch erlauben unpersönliches Passiv,\\
      Englisch nicht
\pause
\item Dänisch, Isländisch erlauben Promotion beider Objekte zum Subjekt,\\
      Deutsch und Englisch nicht
\pause
\item Dänisch und Isländisch haben ein morphologisches Passiv,\\
      Deutsch und Englisch nicht
\pause
\item Deutsch erlaubt das Fernpassiv, Dänisch hat das komplexe Passiv und Englisch und Dänisch haben
  das Reportive Passive

\end{itemize}

}


\subsection{Morphological and Analytic Forms}

\frame[shrink]{
\frametitle{Morphologische und analytische Formen im Dänischen}

\begin{itemize}
\item morphologisches Passiv: \suffix{s}-Suffix, Präsens- und Past-Varianten:
\eal
\ex[]{\label{ex-laeseract}
\gll Peter læser avisen.\\
     Peter liest Zeitung.{\sc def}\\
\glt `Peter liest die Zeitung.'}
\ex[]{\label{ex-laeses}
\gll Avisen              læses af Peter.\\
     Zeitung.{\sc def} liest.{\sc pres}.{\sc pass} von Peter\\
\glt `Die Zeitung wird von Peter gelesen.'}
\ex[]{\label{ex-laestes}
\gll Avisen            læs\gruen{t}es af Peter.\\
     Zeitung.{\sc def} lesen.{\sc past}.{\sc pass} von Peter\\
\glt `Die Zeitung wurde von Peter gelesen.'}
\zl

\pause
\item  analytische Form mit \emph{blive} + Partizip (\mex{1}) 

\ea
\gll Avisen            bliver læst af Peter.\\
     Zeitung.{\sc def} wird   gelesen von Peter\\
\glt `Die Zeitung wird von Peter gelesen.'
\z

%% \pause
%% \item

%% The morphological passive may also apply to infinitives:
%% \ea
%% \gll Avisen skal læses hver dag.\\
%%       newspaper.def must read.{\sc inf}.{\sc pass} every day\\
%% \glt `The newspaper must be read every day.'

%% \z
\end{itemize}

}


\frame{
\frametitle{Im Deutschen und Englischen nur analytische Formen}


\begin{itemize}
\item Englisch und Deutsch haben kein morphologisches Passiv:
\eal
\ex The paper was read.
\ex
Der Aufsatz wurde gelesen.
\zl    
\end{itemize}


}



\subsection{Persönliches und unpersönliches Passiv}


\frame{
\frametitle{Persönliches Passiv}

\begin{itemize}
\item Alle betrachteten Sprachen erlauben die Promotion einer Objekt-NP zum Subjekt.

\pause
\item Subjekt kann auch S oder VP sein:
\eal
\ex
\gll At regeringen træder tilbage, bliver påstået.\\
     dass Regierung.{\sc def} tritt zurück wird behauptet\\
\glt `Dass die Regierung zurücktritt, wird behauptet.'
\ex
\gll At reparere bilen, bliver forsøgt.\\
     zu reparieren Auto.{\sc def} wird versucht\\
\glt `Das Auto zu reparieren, wird versucht.'
\zl

%% \pause
%% (We do not assume that Ss or Vs are subjects in German \citep{Reis82}).
\end{itemize}

}


\frame{
\frametitle{Unpersönliche Passive im Deutschen und Isländischen}


\begin{itemize}
\item Deutsch, Dänisch und Isländisch haben unpersönliche Passive.

\pause
\item Deutsch einfach als subjektlose Konstruktion:
\ea
weil noch getanzt wurde
\z

\pause
\item Isländisch ebenfalls \citep[\page 264]{Thrainsson2007a-u}:
\eal
\ex 
\gll Oft var   talað      um   þennan mann.\\
     oft wurde gesprochen über diesen Mann.\acc.\sg.\mas\\\jambox{(Isländisch)}
\ex
\gll Aldrei hefur verið    sofið      í  þessu  rúmi.\\
     nie    hat   geworden geschlafen in diesem Bett.\dat\\
\glt `In diesem Bett ist nie geschalfen worden.'
\zl

\end{itemize}

}

\frame{
\frametitle{Unpersönliche Passive im Dänischen: Expletivum}


\begin{itemize}
\item Dänisch und Englisch brauchen Subjekt. Dänisch hat eine Lösung:

\eal
\ex 
\gll at \blau{der} bliver danset\\
     dass {\sc expl} wird getanzt\\
\glt `dass getanzt wurde'
\ex
\gll at \blau{der} danses\\
     dass {\sc expl} tanzen.{\sc pres}.{\sc pass}\\
\glt `dass getanzt wurde'
\zl

\pause
\item Im Deutschen ist ein expletives Subjekt ausgeschlossen:
\nocite{MOe2011a}
\ea[*]{
weil es noch getanzt wurde
}
\z
%% \eal
%% \ex[*]{ 
%% \gll Bliver danset.\\
%%      is danced\\
%% }
%% \ex[*]{
%% \gll Danses.\\
%%      dance.{\sc pass}\\
%% }
%% \zl


\end{itemize}

}



%% The examples in (\ref{ex-gearbeitet-wurde}) and (\ref{ex-bliver-arbejder}) show passives of
%% mono-valent verbs but of course bi-valent intransitive verbs like the German \emph{denken} (`think')
%% and Danish \emph{passe} (`take care of') also form impersonal passives:
%% \ea
%% \gll dass an die Männer gedacht wurde\\
%%      that {\sc prep} the men thought was\\
%% \glt `that one thought about the men'
%% \z
%% \eal
%% \label{ex-impersonal-passive-pp}
%% \ex
%% \gll Der passes på børnene.\\
%%      {\sc expl} take.care.of.{\sc pres}.{\sc pass} on children.{\sc def}\\
%% \glt `Somebody takes care of the children.'
%% \ex
%% \gll Der bliver passet  på børnene.\\
%%      {\sc expl} is taken.care.of on children.{\sc def}\\
%% \glt `Somebody takes care of the children.'
%% \zl


\subsection{Promotion des primären und sekundären Objekts}


\frame{
\frametitle{Primäres und sekundäres Objekt im Dt.\ und Engl.}

\smallexamples

\begin{itemize}
\item Deutsch und Englisch erlauben nur die Promotion eines Objekts

\eal
\ex[]{
weil der Mann \rot{dem} \rot{Jungen} \gruen{den} \gruen{Ball} schenkt
}
\ex[]{
weil \rot{dem} \rot{Jungen} \gruen{der} \gruen{Ball} geschenkt wurde
}
\ex[*]{
weil \rot{der} \rot{Junge} \gruen{den} \gruen{Ball} geschenkt wurde
}
\zl
\end{itemize}

}

\frame{
\frametitle{Primäres und sekundäres Objekt im Dt.\ und Engl.}

\begin{itemize}
\item Englisch: nur ein Objekt kann zum Subjekt werden:

\eal
\ex[]{
because the man gave \rot{the boy} \gruen{the ball}
}
\ex[]{
because \rot{the boy} was given \gruen{the ball}
}
\ex[*]{
because \gruen{the ball} was given \rot{the boy}
}
\zl

\pause
\item Effekt kann jedoch durch Verwendung eines anderen Valenzmusters oder \emph{get}-Passiv
  erreicht werden.
\eal
\ex because the man gave the ball \blau{to} the boy
\ex because the ball was given \blau{to} the boy
\zl
\end{itemize}

}

\frame{
\frametitle{Primäres und sekundäres Objekt im Dänischen}

\begin{itemize}
\item Im Dänischen können beide Objekte zum Subjekt werden:
\eal
\ex 
\gll fordi manden giver \rot{drengen} \gruen{bolden}\\ 
     weil Mann.{\sc def} gibt Junge.{\sc def} Ball.{\sc def}\\
\glt `weil der Mann dem Jungen den Ball gibt'
\ex\label{ex-boy-was-given-ball-danish}
\gll fordi \rot{drengen} bliver givet \gruen{bolden}\\ 
     weil Junge.{\sc def} wird gegeben Ball.{\sc def}\\
\glt `weil der Junge den Ball gegeben bekommt'
\ex\label{ex-ball-was-given-boy-danish}
\gll fordi \gruen{bolden} bliver givet \rot{drengen}\\ 
     weil Ball.{\sc def} wird gegeben Junge.{\sc def}\\
\glt `weil der Ball dem Jungen gegeben wird'
\zl
\pause
\item Aber das Dänische unterscheidet sich von Sprachen wie Moro \citep{AMM2013a} dadurch, dass die
  Objekte klar unterscieden werden. Zum Beispiel ist ihre Reihenfolge fest:
\ea[*]{
fordi manden giver \gruen{bolden} \rot{drengen}
}
\z

\end{itemize}

}


\frame[shrink=2]{
\frametitle{Primäres und sekundäres Objekt im Isländischen}


\begin{itemize}
\item \citet*[\page 460]{ZMT85a}:\\
      Das Dativobjekt kann zu schrägem Subjekt werden:

\ea
\gll Konunginum voru gefnar ambáttir.\\
     König.\dat{} wurden gegeben.\fem.\pl{} Sklavinnen.\nom.\fem.\pl\\\jambox{[S$_i$ Aux \_$_i$ V O]} 
\glt `Dem König wurden Sklavinnen gegeben.'
\z

Das Akkusativobjekt bekommt dann Nominativ.

\pause
\item Oder das Akkusativobjekt wird zum Subjekt:
\ea
\gll Ambáttin var gefin konunginum.\\
     Sklavin.\nom.\sg{}  wurde gegeben.\fem.\sg{} König.\dat\\\jambox{[S$_i$ Aux \_$_i$ V O]} 
\glt `Die Sklavin wurde dem König gegeben.'
\z

\pause
\item Nebenbemerkung: Verb kongruiert immer mit Nominativ.

\end{itemize}

}






\subsection{Designated Argument Reduction}


\frame{
\frametitle{Designated Argument Reduction}

\begin{itemize}
\item \citet{Haider86,HM94a,Mueller2003e}:\\
{\sc designated argument} ({\sc da}) das Subjekt transitiver und unergativischer Verben. (ein
``echtes'' Subjekt)

\pause
\item \daw unakkusativischer Verben ist die leere Liste.

\pause
\item Passiv = LR, die die \daliste von der Argumentstruktur des Eingabeverbs bzw.\ -stamms abzieht.


\ea
%\resizebox{\linewidth}{!}
\z

\end{itemize}

}

%% \frame{
%% \frametitle{Kasuszuweisung}

%% \begin{itemize}
%% \item 
%% Lexikalischer Kasus bleibt bei Passivierung unverändert:
%% \eal
%% \ex
%% weil der Mann ihm geholfen hat
%% \ex
%% weil ihm geholfen wurde
%% \zl

%% \end{itemize}

%% }

\frame{
\frametitle{Designated Argument Reduction}

\begin{itemize}
\item Partizipbildungsregel:
\ea
\label{lr-passive-prelim}
Lexikonregel für die Bildung des Partizips (vorläufig):
\ms[stem]{
head   & \ms[verb]{ da & \ibox{1}\\
                  }\\
arg-st & \ibox{1} $\oplus$ \ibox{2} \\
} $\mapsto$
\ms[word]{
arg-st & \ibox{2} \\
}
\z
\pause
\item 
Das designierte Argument wird blockiert.
\end{itemize}



}


\frame{
\frametitle{Designated Argument Reduction}

\begin{itemize}

\item \argstl des Partizips ist entweder leer oder beginnt mit dem Objekt der Aktivform:
\ea
\label{partizipien-hm}
%\resizebox{\linewidth}{!}
\z

\item Das erste Element der \argstl mit strukturellem Kasus bekommt Nominativ:

\ea
Der Aufsatz wurde gelesen.
\z
\end{itemize}

}


\frame{
\frametitle{Englisch: Promotion des ersten Objekts}

\begin{itemize}
\item Englisch: kein Dativ, struktureller Kasus für erstes Objekt,\\
      lexikalischer Akkusativ für zweites Objekt von \emph{give}
\ea\label{da-repr-hm-English}
%\resizebox{\linewidth}{!}{%
\begin{tabular}[t]{@{}l@{ }l@{ }l@{ }l@{ }l@{}}
  &                     & {\sc arg-st}\\[2mm]
b.&dance   (unerg):     & \liste{NP[\type{str}]}\\[2mm]
%c.&auf"|fallen (unacc): & \liste{}                         & \liste{NP[\type{str}], NP[\type{ldat}]}\\[2mm]
c.&read      (trans):   & \liste{NP[\type{str}], NP[\type{str}]}\\[2mm]
d.&give      (ditrans): & \liste{NP[\type{str}], NP[\type{str}], \blaubf{NP[{\it lacc\/}]}}\\[2mm]
e.&help      (trans):   & \liste{NP[\type{str}], \blaubf{NP[\type{str}]}}\\
\end{tabular}
%}
\z

\pause
\item Deutsch kann zweites Objekt (Akkusativ) zum Subjekt machen,\\
      Englisch das erste (das Objekt, das näher am Verb steht, OV vs.\ VO):
\eal
\ex dass \gruen{dem Jungen} \rot{der Ball} gegeben wurde
\ex because \rot{the boy} was given \gruen{the ball}
\zl

\end{itemize}
}

\frame{
\frametitle{Englisch: Persönliches Passiv mit \emph{help}}

\begin{itemize}
\item Englisch: kein Dativ, struktureller Kasus für erstes Objekt, lexikalischer Akkusativ für
  zweites Objekt von \emph{give}
\ea\label{da-repr-hm-English}
%\resizebox{\linewidth}{!}{%
\begin{tabular}[t]{@{}l@{ }l@{ }l@{ }l@{ }l@{}}
  &                     & {\sc arg-st}\\[2mm]
b.&dance   (unerg):     & \liste{NP[\type{str}]}\\[2mm]
%c.&auf"|fallen (unacc): & \liste{}                         & \liste{NP[\type{str}], NP[\type{ldat}]}\\[2mm]
c.&read      (trans):   & \liste{NP[\type{str}], NP[\type{str}]}\\[2mm]
d.&give      (ditrans): & \liste{NP[\type{str}], NP[\type{str}], \blaubf{NP[{\it lacc\/}]}}\\[2mm]
e.&help      (trans):   & \liste{NP[\type{str}], \blaubf{NP[\type{str}]}}\\
\end{tabular}
%}
\z

\item Deutsch hat ein unpersönliches Passiv für \emph{helfen},\\
      aber Englisch ein persönliches:

\eal
\ex weil ihm geholfen wurde
\ex because he was helped
\zl

\end{itemize}
}


%% \frame{
%% \frametitle{Isländisch}

%% \begin{itemize}
%% \item Dativ und Genitiv sind lexikalisch:



%% \end{itemize}

%% }

\subsection{Primäre und sekundäre Objekte}



\frame{
\frametitle{Dänisch: Promotion primäres und sekundäres Objekt}

\begin{itemize}
\item Dänisch ist wie Englisch: kein Dativ,\\
      erlaubt aber die Promotion beider Objekte von ditransitiven Verben:
\ea\label{da-repr-hm-Danish}
%\resizebox{\linewidth}{!}
\z

Dänisch hat zwei Objekte mit strukturellem Kasus,\\
Deutsch und Englisch nur eins.

\pause
\item Persönliches Passiv: Promotion eines Objekts mit strukturellem Kasus.

\end{itemize}

}

\frame[shrink=15]{
\frametitle{Verallgemeinerte Lexikonregel}


\begin{itemize}
\item Alt:

\ea
\label{lr-passive-prelim}
Lexikonregel für die Bildung des Partizips (vorläufig):\\
\ms[stem]{
head   & \ms[verb]{ da & \ibox{1}\\
                  }\\
arg-st & \ibox{1} $\oplus$ \ibox{2} \\
} $\mapsto$
\ms[word]{
arg-st & \ibox{2} \\
}
\z

Erstes Argument unterdrückt, zweites ist jetzt das erste.

\pause
\item \emph{promote} stellt die Liste \ibox{3} zur Verfügung, die entweder der Liste \ibox{2}
  entspricht oder falls \ibox{2} zwei NPen mit strukturellem Kasus enthält, zusätzlich auch noch
  eine Liste in der die Reihenfolge der beiden NPen vertauscht ist, \dash, die zweite NP mit
  strukturellem Kasus wird an erste Stelle gestellt.


\eas
Passiv-Lexikonregel für Dänisch, Deutsch, Englisch, Isländisch:\\
\ms[stem]{
head   & \ms[verb]{ da & \ibox{1}\\
                  }\\
arg-st & \ibox{1} $\oplus$ \ibox{2} \\
} $\mapsto$
\ms[word]{
arg-st & \ibox{3} \\
} $\wedge$ promote(\ibox{2}, \ibox{3})
\zs

\end{itemize}

}

%% \frame{
%% \frametitle{Doppelobjektkonstruktionen im Isländischen}

%% \begin{itemize}
%% \item Valenzspezifikation im Isländischen identisch mit der Deutschen:
%% \ea
%% \label{partizipien-hm}
%% %\resizebox{\linewidth}{!}{%
%% \begin{tabular}[t]{@{}l@{ }ll@{ }l@{}}
%%   &                   & {\sc arg-st}\\[2mm]
%% a.&tanzen    (unerg): & \liste{ NP[\type{str}]$_i$ }\\[2mm]
%% b.&lesen     (trans): & \liste{ NP[\type{str}]$_i$, NP[\type{str}]$_j$ }\\[2mm]
%% c.&geben   (ditrans): & \liste{ NP[\type{str}]$_i$, NP[\type{ldat}]$_j$ , NP[\type{str}]$_k$ }\\[2mm]
%% d.&helfen    (unerg): & \liste{ NP[\type{str}]$_i$, NP[\type{ldat}]$_j$ }\\
%% \end{tabular}
%% %}
%% \z
%% \pause
%% \item einziger Unterschied: Isländisch erlaubt das Mapping von NPen mit lexikalischem Kasus auf \spr.


%% \end{itemize}

%% }


%% \frame{
%% \frametitle{Doppelobjektkonstruktion im Deutschen/Isländischen}







%% }



\subsection{Unpersönliches Passiv}
\label{sec-impersonals}

\frame[shrink]{
\frametitle{Unpersönliches Passiv}

\begin{itemize}
\item Deutsch, Isländisch: Subjekt nicht obligatorisch

% Thrainsson 2007 264
      Dänisch: Einführung eines Expletivums bei der Abbildung von \argst auf \spr/\comps.

\pause
\item Englisch und Dänisch bilden die erste NP/VP/CP auf \spr ab und die restlichen Argumente auf \comps
und\\
Dänisch fügt ein Expletivum ein, wenn es keine anderen Elemente gibt,\\
die als Subjekt fungieren könnten.
\nocite{BB2007a}

\ea\label{da-repr-hm-Danish}
%\resizebox{\linewidth}{!}{%
\begin{tabular}[t]{@{}l@{ }l@{ }l@{ }l@{ }l@{~~~~~}l@{}}
  &                        & {\sc arg-st}                     & \spr   & \comps\\[2mm]
%a.&ankomme (unacc):       & \liste{}                         & \liste{NP[\type{str}]}\\[2mm]
a.&danset/-s   (unerg):     & \liste{}                        & \liste{ \blau{NP$_{expl}$} } & \liste{} \\[2mm]
%c.&auf"|fallen (unacc): & \liste{}                         & \liste{NP[\type{str}], NP[\type{ldat}]}\\[2mm]
b.&læst/-s      (trans):   &  \liste{NP[\type{str}]$_j$ }         & \liste{NP[\type{str}]$_j$ } & \eliste\\[2mm]
c.&givet/-s      (ditrans): & \liste{NP[\type{str}]$_j$, NP[\type{str}]$_k$ } & \liste{NP[\type{str}]$_j$ } & \liste{NP[\type{str}]$_k$ }\\[2mm]
  &                         & \liste{NP[\type{str}]$_k$, NP[\type{str}]$_j$ } & \liste{NP[\type{str}]$_k$ } & \liste{NP[\type{str}]$_j$ }\\[2mm]
d.&hjulpet/-s    (trans):   & \liste{NP[\type{str}]$_j$ }                      & \liste{NP[\type{str}]$_j$ } & \liste{}\\
\end{tabular}
%}
\z
\end{itemize}

}

%% \section{The Lexical Rules}

%% The following lexical rule accounts for the participle formation in German and English:
%% \ea
%% Lexical rule for the formation of the participle in English and German:\\
%% \ms[stem]{
%% head   & \ms[verb]{ da & \ibox{1}\\
%%                   }\\
%% arg-st & \ibox{1} $\oplus$ \ibox{2} \\
%% } $\mapsto$
%% \ms[word]{
%% arg-st & \ibox{2} \\
%% }
%% \z
%% This rule blocks the designated argument, if there is one. The rule is an inflectional rule, that
%% is, it maps a stem onto a word and adds the morphology of the particple.
%% The passive auxiliary requires a particple that has a referential designated argument and hence the
%% passivization of unaccusatives is excluded. See Section~\ref{sec-auxiliary} for details.

%% The lexical rule for the passives in Danish cannot be that simple since we have to take care of the
%% insertion of an expletive in the cases in which there is no argument that can be promoted to
%% subject. Hence we have to distinguish two cases: One for the impersonal passive with an inserted
%% expletive and one for the personal passive. (\mex{1}) shows the lexical rule for the personal
%% passive:
%% \eas
%% Lexical rule for the personal passive in Danish:\\
%% \ms{
%% head   & \ms[verb]{ da & \ibox{1}\\
%%                   }\\
%% arg-st & \ibox{1} $\oplus$ \ibox{2} {\rm ( \sliste{ NP $\vee$ S } $\oplus$ \etag )} \\
%% } $\mapsto$\\
%% \flushright\ms{
%% arg-st & \ibox{2} \\
%% }
%% \zs
%% This rule is similar to the one for English and German, but it requires that the list \ibox{2}
%% starts with an NP or an S. The output of the rule has \ibox{2} as the value of \argst. Since it was
%% required in the input that \ibox{2} starts with something that will be realized as the subject, it
%% is clear that we are talking about personal passives. (\etag stands for an arbitrary list).

%% The lexical rule for the impersonal passive places the reverse restriction on the list \ibox{2}:
%% \eas
%% Lexical rule for the impersonal passive in Danish:\\
%% \ms{
%% head   & \ms[verb]{ da & \ibox{1}\\
%%                   }\\
%% arg-st & \ibox{1} $\oplus$ \ibox{2}  \\
%% } $\wedge$ \ibox{2} $\neq$ ( \sliste{ NP $\vee$ S } $\oplus$ \etag ) $\mapsto$
%% \ms{
%% arg-st & \sliste{ NP\sub{der} } $\oplus$ \ibox{2} \\
%% }
%% \zs
%% In (\mex{0}) it is required that \ibox{2} does not start with an NP or an S and hence the result
%% will be an impersonal passive with \ibox{2} either being the empty list or a list that contiains an
%% oblique argument as for instance a PP (see (\ref{ex-impersonal-passive-pp})).

%% The output of the lexical rule has an \argstl that contains all elements of \ibox{2} but in addition
%% an expletive \emph{der} NP, which will be mapped onto the subject valence feature.

%% Both lexical rules are underspecified for the type of their input. There are subtypes of the lexical
%% rules that are given in (\mex{-1}) and (\mex{0}): These lexical rules can either apply to fully
%% inflected verbs\NOTE{ finite forms and infinitives? } and add the \suffix{s} suffix or they apply to
%% stems and add the participle morphology.

%% The generalization over all passive lexical rules is the following constraint:
%% \ea
%% Constraint that holds for all passive lexical rules in Danish, English, and German (preliminary):\\
%% \ms{
%% head   & \ms[verb]{ da & \ibox{1}\\
%%                   }\\
%% arg-st & \ibox{1} $\oplus$ \ibox{2} \\
%% } $\mapsto$
%% \ms{
%% arg-st & \etag $\oplus$ \ibox{2} \\
%% }
%% \z
%% The individual rules differ as to the value of \etag. In German and English \etag is always the
%% empty list. This is also true of personal passives in Danish. Only for impersonal passives \etag is
%% a list that contains the expletive NP. We believe that the representation in (\mex{0}) captures the
%% phenomenon of passive rather elegantly: Passive is the suppression of the most prominent argument.


\subsection{Isländisch: Schräge Subjekte}

\frame{
\frametitle{Isländisch}

\begin{itemize}
\item Kasusverteilung wie im Deutschen:
\ea\label{da-repr-hm-English}
%\resizebox{\linewidth}{!}{%
\begin{tabular}[t]{@{}l@{ }l@{ }l@{ }l@{ }l@{}}
  &                     & {\sc arg-st}\\[2mm]
b.&dance   (unerg):     & \liste{NP[\type{str}]}\\[2mm]
%c.&auf"|fallen (unacc): & \liste{}                         & \liste{NP[\type{str}], NP[\type{ldat}]}\\[2mm]
c.&read      (trans):   & \liste{NP[\type{str}], NP[\type{str}]}\\[2mm]
d.&give      (ditrans): & \liste{NP[\type{str}], \blaubf{NP[\type{ldat}]}, \blaubf{NP[\type{str}]}}\\[2mm]
e.&help      (trans):   & \liste{NP[\type{str}], \blaubf{NP[\type{ldat}]}}\\
\end{tabular}
%}
\z
\item Unpersönliches Passiv mit \emph{tanzen} gleich,\\
      aber \emph{helfen} bildet kein unpersönliches Passiv sondern ein persönliches.
\pause
\item \emph{geben} erlaubt zwei Varianten:\\
      Dativ wird zu schrägem Subjekt, Akkusativ wird zum Subjekt.

\end{itemize}


}


\frame{
\frametitlefit{Isländisch: Schräge Subjekte und Doppelobjektkonstruktionen}

\begin{itemize}
\item erste NP wird zum Subjekt, auch NPen mit lexikalischem Kasus\\\citep[\page 147--148]{Wechsler95a-u}

\ea\label{da-repr-hm-Danish}
%\resizebox{\linewidth}{!}{%
\begin{tabular}[t]{@{}l@{ }l@{ }l@{ }l@{ }l@{~~~~~}l@{}}
  &                        & {\sc arg-st}                     & \spr   & \comps\\[2mm]
%a.&ankomme (unacc):       & \liste{}                         & \liste{NP[\type{str}]}\\[2mm]
a.& danced    (unerg):     & \liste{}                        & \liste{ } & \liste{} \\[2mm]
%c.&auf"|fallen (unacc): & \liste{}                         & \liste{NP[\type{str}], NP[\type{ldat}]}\\[2mm]
b.&read       (trans):   &  \liste{NP[\type{str}]$_j$ }         & \liste{NP[\type{str}]$_j$ } & \eliste\\[2mm]
c.&given      (ditrans): & \liste{NP[\type{ldat}]$_j$, NP[\type{str}]$_k$ } & \liste{NP[\type{ldat}]$_j$ } & \liste{NP[\type{str}]$_k$ }\\[2mm]
  &                      & \liste{NP[\type{str}]$_k$, NP[\type{ldat}]$_j$ } & \liste{NP[\type{str}]$_k$ } & \liste{NP[\type{ldat}]$_j$ }\\[2mm]
d.&helped    (trans):   & \liste{NP[\type{ldat}]$_j$ }                  & \liste{NP[\type{ldat}]$_j$ } & \liste{}\\
\end{tabular}
%}
\z
\end{itemize}

}


%\section{Variation and Generalizations}

\subsection{Das analytische Passiv (Hilfsverb)}
\label{sec-auxiliary}



\frame{
\frametitle{Das Hilfsverb}

\begin{itemize}
\item Das Passivhilfsverb ist für alle behandelten Sprachen ähnlich:
\ea
Passivhilfsverb für Dänisch, Deutsch, Englisch:
\ms{
arg-st \ibox{1} $\oplus$ \ibox{2} $\oplus$  \liste{ \ms{ vform & ppp\\
                                                                        da & \sliste{ XP$_{ref}$ }\\
                                                                                      spr   & \ibox{1}\\
                                                                                      comps & \ibox{2}\\
                                                                                    } } 
}
\z

\pause
\item \daw schließt unakkusatische Verben und Wetterverben aus

\pause
\item Deutsch bildet Verbalkomplex: Argumente des Partizips (\ibox{1} and \ibox{2}) werden vom
  Passivhilfsverb angezogen \citep{HN89a}. 

\pause
\item Verbalkomplexschema erlaubt ungesättigte Nicht-Kopftochter.

\pause
\item Funktioniert auch für Sprachen, die keine Verbalkomplexe bilden:\\
\ibox{2} ist dann die leere Liste. 

%% \item Hence, we have explained how
%% Danish and English embed a VP and German forms a verbal complex although the lexical item of the
%% auxiliary does not require a VP complement.

\end{itemize}

}

\subsection{Das morphologische Passiv}




\frame{
\frametitle{Das morphologische Passiv}


%% We assume that the same lexical rule that accounts for the participle forms can be used for the
%% morphological passives in Danish, modulo differences in the realizations of affixes of course. For
%% the morphological passive it is assumed that the \da of the input to the lexical rule has to contain
%% a referential XP. As was discussed in the previous section, this excludes morphological passives of
%% unaccusatives and weather verbs. 

\begin{itemize}
\item Lexikonregel funktioniert auch für das morphologische Passiv. Es wird einfach ein \suffix{s} angehängt.
\end{itemize}

}


%% \section{Agent Expressions}

%% We follow \citet[Chapter~7]{Hoehle78a} and \citet[Section~5]{Mueller2003e} and treat the \emph{by}
%% phrases as adjuncts.

\subsection{Perfekt}



\frame{
\frametitle{Perfekt}

\begin{itemize}
\item Deutsch: Nur ein Partizip für Passiv und Perfekt \citep{Haider86}. 
\pause
\item Das designated argument wird blockiert, ist aber im Lexikonelement enthalten
\pause
\item Perfekthilfsverb deblockiert es.
\eal
\ex
Der Aufsatz wurde gelesen.
\ex
Er hat den Aufsatz gelesen.
\zl
\pause
\ms{
arg-st \ibox{1} $\oplus$ \ibox{2} $\oplus$ \ibox{3} $\oplus$  \liste{ \ms{ vform & ppp\\
                                                                        da & \ibox{1}\\
                                                                        spr   & \ibox{2}\\
                                                                        comps & \ibox{3}\\
                                                                       } } 
}


\end{itemize}

}

\frame{
\frametitle{Analyse als komplexes Prädikat für Dänisch und Englisch?}

\begin{itemize}
\item Bei einer Analyse mit Argumentdeblockierung müsste man Struktur in (\mex{1}a--b) annhemen:
\eal
\ex He [has given] the book to Mary.
\ex The book [was given] to Mary.
\ex He has [given the book to Mary].
\ex The book was [given to Mary].
\zl

Sonst wüssten wir zu spät vom deblockierten Subjekt, denn das Partizip würde ja nur -- wie in
(\mex{0}d) eine PP verlangen.


\end{itemize}

}


%% \frame{
%% \frametitle{Expletives}

%% \begin{itemize}
%% \item Expletives needed for passive only:

%% \eal
%% \ex[]{
%% \gll  at   der        bliver arbejdet\\
%%       that {\sc expl} is     worked\\
%% }
%% \ex[*]{
%% \gll  at   Peter har arbejdet der\\
%%       that Peter has worked   {\sc expl}\\
%% }
%% \ex[*]{
%% \gll  at   der        har arbejdet Peter\\
%%       that {\sc expl} has worked   Peter\\
%% }
%% \zl

%% \end{itemize}

%% }



%% \frame{
%% \frametitle{A Solution that Almost Works}

%% \begin{itemize}
%% \item Complex Passive: There has to be a way to distinguish between participles that can be used in both perfect and
%%   passive:\\
%% {\sc voice} feature. 

%% \begin{itemize}
%% \item Value is \type{passive} for those participles that cannot be used in perfect constructions.

%% \pause
%% \item Value is underspecified for participles that can be used in both perfect and passive

%% \pause
%% \item Perfect requires {\sc voice} value to be \type{active}.
%% \end{itemize}

%% \pause
%% \item Expletives: Perfect attracts args from \argstl rather than \spr/\comps.
%% \begin{itemize}
%% \item Since expletives are not on \argst, they will not get into the way.
%% \end{itemize}
%% \end{itemize}


%% }


\frame{
%\frametitle{But: (Partial) Fronting}
\frametitle{Problem: (Partial) Fronting}

\smallexamples

\begin{itemize}
\item \citet{Meurers99b} hat einen Trick gefunden, wie man die Kasuszuweisung in (\mex{1})
  analysieren kann:
\nocite{Meurers2000b,MdK2001a}
\eal
\ex 
Gelesen wurde der Aufsatz schon oft.
\ex 
Der Aufsatz gelesen wurde schon oft.
\ex
Den Aufsatz gelesen hat er schon oft.
\zl
\pause
\item Das funktioniert aber nicht für Dänisch/Englisch, denn hier haben wir nicht nur Kasus- sondern
  auch Positionsunterschiede:
\eal
\ex The book should have been given to Mary and\\
    {}[given to Mary] it was.
\ex He wanted to give the book to Mary and\\
    {}[given the book to Mary] he has.
\zl

Wenn sich keine ausgeklügelten Mechanismen für die Unterspezifikation verschiedener Mappings finden
lassen,\\ müssen wir wohl zwei verschiedene Partizipformen annehmen.
\end{itemize}


}








\subsection{Das Fernpassiv}
\label{sec-remote-passive-phen}

\frame{
\frametitle{Das Fernpassiv}


\begin{itemize}
\item \citet[S.\,175--176]{Hoehle78a}: in bestimmten Kontexten Objekte
von \emph{zu}-Infinitiven im Nominativ.

Die folgenden Sätze sind Beispiele für das sogenannte Fernpassiv:
\eal
\ex
daß er auch von mir zu überreden versucht wurde\footnote{
        \citew*[S.\,212]{Oppenrieder91a}.%
}
\ex
weil    der Wagen oft zu reparieren versucht wurde
\zl
\end{itemize}
}

\frame{
\frametitle{Das Fernpassiv}
\smallframe

Akkusativobjekte eingebetteter Verben können im Passiv zum Nominativ werden:
\eal
\ex Dabei darf jedoch nicht vergessen werden, daß in der Bundesrepublik, wo \blauit{ein Mittelweg} \blauit{zu gehen versucht wird}, 
die Situation der Neuen Musik allgemein und die Stellung der Komponistinnen im besonderen noch recht unbefriedigend ist.\footnote{
Mannheimer Morgen, 26.09.1989, Feuilleton; Ist's gut, so unter sich zu bleiben?
}
\pause
\ex Noch ist es nicht so lange her, da ertönten gerade aus dem Thurgau jeweils die lautesten Töne, 
    wenn im Wallis oder am Genfersee im Umfeld einer Schuldenpolitik mit den unglaublichsten Tricks 
    \blauit{der sportliche Abstieg} \blauit{zu verhindern versucht wurde}.\footnote{
St.\ Galler Tagblatt, 09.02.1999, Ressort: TB-RSP; HCT und das Prinzip Hoffnung.%
}
\pause
\ex Die Auf- und Absteigenden erzeugen ungewollt einen Ton,
        \blauit{der} bewusst nicht als lästig \blauit{zu eliminieren versucht wird}, 
    sondern zum Eigenklang des Hauses gehören soll, so wünschen es sich die Architekten.\footnote{
Züricher Tagesanzeiger, 01.11.1997, S.\,61.%
}
\zl

}

\frame{
\frametitle{Beispiele mit \word{beginnen}, \word{vergessen} und \word{wagen}}
\smallframe
\citet{Wurmbrand2003a}:
\eal
%% \ex
%% \blau{dieser} wurde bereits zu bauen begonnen.\footnote{
%%         \url{http://www.hollabrunn.noe.gv.at/mariathal/ortsvorsteher.html}, 28.07.2003.
%% }
\pause
\ex
\blau{der zweite Entwurf} wurde zu bauen begonnen,\footnote{
\url{http://www.waclawek.com/projekte/john/johnlang.html}, 28.07.2003.
}
\zl
\pause
\eal
\ex Anordnungen, die zu stornieren vergessen \blau{wurden}\footnote{
        \url{http://www.rlp-irma.de/Dateien/Jahresabschluss2002.pdf}, 28.07.2003.
}
\pause
\ex Aufträge [\ldots], die zu drucken vergessen worden \blau{sind}\footnote{
        \url{http://www.iitslips.de/news.html}, 28.07.2003.
}
\zl
\pause
\eal
%\ex Ist plötzlich übervoll von Emotionen und längst begrabenen Träumen, die nicht zu leben gewagt wurden\footnote{
% nicht auffindbar
\ex NUR Leere, oder doch noch Hoffnung, weil aus Nichts wieder Gefühle entstehen,
    die so vorher nicht mal zu träumen gewagt \blau{wurden}?\footnote{
        \url{http://www.ultimaquest.de/weisheiten_kapitel1.htm}, 28.07.2003.
}
\pause
\ex Dem Voodoozauber einer Verwünschung oder die gefaßte Entscheidung zu einer Trennung,
    die bis dato noch nicht auszusprechen gewagt \blau{wurden}\footnote{
        \url{http://www.wedding-no9.de/adventskalender/advent23_shawn_colvin.html}, 28.07.2003.
}
\zl
% Kasus bei PVP wie Haiders entziffern: Am leichtesten zu erklären fiel den 
% Experten dabei gestern der Kursverlust der Telekom, zu deren Schuldenproblem 
% eine neue Hiobsbotschaft kam.  (taz. 8./9. 9. 01 S. 9.)
%


}

\frame{
\frametitle{Fernpassiv und Verbalkomplexbildung (I)}

\begin{itemize}
\item Objekt eines Verbs, das unter ein Passivpartizip eingebettet ist,\\
wird zum Subjekt des Satzes:
\eal
\ex weil er den Wagen oft zu reparieren versucht hat
\ex weil der Wagen oft \blau<2>{zu reparieren versucht wurde}
\zl
\pause
\item Fernpassiv nur bei Verbalkomplexbildung möglich:
\eal
\ex[]{
weil oft versucht wurde, \blau{den Wagen zu reparieren}
}
\ex[*]{
weil oft versucht wurde, der Wagen zu reparieren
}
\pause
\ex[]{
\blau{Den Wagen zu reparieren} wurde oft versucht
}
\ex[*]{
Der Wagen zu reparieren wurde oft versucht
}
\zl
\end{itemize}

}

\frame{
\frametitle{Fernpassiv und Verbalkomplexbildung (II)}

\begin{itemize}

\item Erklärung: Fernpassiv = Passivierung des Prädikatskomplexes
\ea
weil    der Wagen     oft   [[zu reparieren versucht] wurde]
%
\z

\pause
\item In (\mex{1}a,c) liegen keine Verbalkomplexe vor. 
\eal
\ex[]{
weil oft versucht wurde, \blau<2>{den Wagen zu reparieren}
}
\ex[*]{
weil oft versucht wurde, der Wagen zu reparieren
}
\ex[]{
\blau<2>{Den Wagen zu reparieren} wurde oft versucht
}
\ex[*]{
Der Wagen zu reparieren wurde oft versucht
}
\zl

Objekt von \emph{zu reparieren} ist Teil der VP $\to$ bekommt Akkusativ
\pause

Die Passive in (\mex{0}a,c) sind unpersönliche Passive.

\end{itemize}

}

\frame[shrink=10]{

%% \centerline{\scalebox{1}{
%% %\centerfit{%
%% \begin{tikzpicture}
%% \tikzset{level 1+/.style={level distance=4\baselineskip}}
%% \tikzset{level 2+/.style={level distance=5\baselineskip}}
%% \tikzset{frontier/.style={distance from root=14\baselineskip}}
%% \Tree[.V\feattab{
%%               \vform \type{fin},\\
%%               \comps \ibox{1} } 
%%         [.{\ibox{3} V\feattab{
%%               \vform \type{ppp},\\
%%               \subj  \ibox{1},\\
%%               \comps \eliste }} 
%%            [.{\highlight{\ibox{2} V}<1>\feattab{
%%               \highlight{\vform \type{inf}}<1>,\\
%%               \subj  \highlight{\sliste{ NP[\str] }}<2,3>, \\ 
%%               \comps \highlight{\ibox{1} \sliste{ NP[\str] }}<2,3> }} {zu reparieren} ]
%%            [.V\feattab{
%%               \vform \type{ppp},\\
%%               \subj  \ibox{1},\\
%%               \comps \sliste{ \ibox{2} } } versucht ] ]
%%         [.V\feattab{
%%               \vform \type{fin},\\
%%               \comps \ibox{1} $\oplus$ \sliste{
%%                 \ibox{3} }} wurde ] 
%% ]
%% \end{tikzpicture}}
%% }

\centerline{\scalebox{1}{
%\centerfit{%
\begin{forest}
sm edges
[V\feattab{
              \vform \type{fin},\\
              \comps \ibox{1} } 
        [{\ibox{3} V\feattab{
              \vform \type{ppp},\\
              \subj  \ibox{1},\\
              \comps \eliste }} 
           [{\highlight{\ibox{2} V}<1>\feattab{
              \highlight{\vform \type{inf}}<1>,\\
              \subj  \highlight{\sliste{ NP[\str] }}<2,3>, \\ 
              \comps \highlight{\ibox{1} \sliste{ NP[\str] }}<2,3> }} [zu reparieren] ]
           [V\feattab{
              \vform \type{ppp},\\
              \subj  \ibox{1},\\
              \comps \sliste{ \ibox{2} } } [versucht] ] ]
        [V\feattab{
              \vform \type{fin},\\
              \comps \ibox{1} $\oplus$ \sliste{
                \ibox{3} }} [wurde] ] 
]
\end{forest}}
}


\begin{itemize}
\item \emph{versuchen} zieht Argumente von \emph{reparieren} an: \argstw \sliste{ NP[\str], NP[\str], V[\type{inf}] }
\pause
\item Passiv-LR unterdrückt erstes Argument: \emph{versucht} hat
\argstw \sliste{ NP[\str], V[\type{inf}] } 
\pause
\item \emph{zu reparieren versucht}: \argstw \sliste{ NP[\str] } und
\emph{zu reparieren versucht wurde} auch
\end{itemize}


}


\frame{
\frametitle{Fernpassiv mit Objektkontrollverben}

\begin{itemize}
\item Fernpassiv auch mit Objektkontrollverben möglich:
\eal
\ex
Keine Zeitung         wird ihr       zu lesen erlaubt.\footnote{
        Stefan Zweig. \emph{Marie Antoinette}. Leipzig: Insel-Verlag. 1932, S.\,515, 
        zitiert nach \citew[S.\,309]{Bech55a}. Siehe \citet[S.\,13]{Askedal88}.
}
\ex\iw{auskosten}
Der Erfolg        wurde uns      nicht auszukosten erlaubt.\footnote{
        \citew[S.\,110]{Haider86c}.%
}
\zl

\pause
\item Passiv der Konstruktion ohne Verbalkomplex ist ein unpersönliches Passiv:
\eas
Uns wurde erlaubt, den Erfolg auszukosten.
\zs
\pause
\item Generalisierung: In Passivkonstruktionen, in denen ein Verbalkomplex unter das Passivhilfsverb
eingebettet ist, wird das Subjekt unterdrückt und von den verbleibenden Argumenten
wird das erste Argument mit strukturellem Kasus zum Subjekt und bekommt Nominativ.%
\end{itemize}
}

\frame{
\frametitle{Fernpassiv mit Objektkontrollverben}

\ea
Keine Zeitung         wird ihr       zu lesen erlaubt.\footnote{
        Stefan Zweig. \emph{Marie Antoinette}. Leipzig: Insel-Verlag. 1932, S.\,515, 
        zitiert nach \citew[S.\,309]{Bech55a}. Siehe \citet[S.\,13]{Askedal88}.
}
\z

\oneline{%
\begin{tabular}{@{}l@{ }l@{}}
\emph{erlauben}: & \sliste{ NP[\str]$_i$, NP[\ldat]$_j$ } $\oplus$ \ibox{1} $\oplus$ \sliste{ V[\comps \ibox{1}] }\\

\emph{zu lesen erlauben}: & \sliste{ NP[\str]$_i$, NP[\ldat]$_j$, NP[\str]$_k$, V[\comps \sliste{ NP[\str]$_k$ }] }\\

\emph{zu lesen erlaubt wird}: & \sliste{ NP[\ldat]$_j$, NP[\str]$_k$, V[\comps \sliste{ NP[\str]$_k$ } ] }\\
\end{tabular}
}

\pause

Erste NP mit strukturellem Kasus ist Subjekt.

}


%% \frame{
%% \frametitle{Komplexes Passiv}

%% \begin{itemize}
%% \item Komplexes Passiv:

%% \ea
%% \gll at Bilen           blev forsøgt repareret\\
%%      dass Auto.{\sc def} wurde versucht repariert\\
%% \glt `dass versucht wurde, das Auto zu reparieren'
%% \z


%% \pause
%% \item Anhebung nur im Passiv.


%% \pause
%% \item \emph{forsøgt} (`versuchen') nimmt im Aktiv sonst kein Partizip:
%% \eal
%% \ex[]{
%% \gll at   Peter har  forsøgt \blaubf{at} \blaubf{reparere} bilen\\
%%      dass Peter hat versucht zu reparieren Auto.{\sc def}\\
%% \glt `dass Peter versucht hat, das Auto zu reparieren'
%% }
%% \ex[*]{
%% \gll at   Peter har  forsøgt \blaubf{repareret} bilen\\
%%      dass Peter hat  versucht  repariert Auto.{\sc def}\\
%% %\glt `that an attempt was made to repair the car'
%% }
%% \zl

%% %% \item Conclusion: We need special lexical items for passive participles.

%% %% \item analysis of the German passive and perfect can be maintained,\\
%% %% compatible with a more general analysis of the passive

%% \end{itemize}

%% }


\subsection{Zusammenfassung}


\frame{
\frametitle{Zusammenfassung}



\begin{itemize}
\item LRen für morphologische und analytische Passive
\pause
\item Das erste Element der \argstl wird unterdrückt
\pause
\item \emph{promote} promoviert eine NP mit strukturellem Kasus

\pause
\item Sprachen unterscheiden sich bzgl.\ der Kasus und der lexikalisch/strukturell Unterscheidung
\pause
\item Beim \argst-Mapping im Dänischen wird Expletivum eingesetzt.

\pause
\item SVO-Sprachen erfordern verschiedene Lexikonelement für Perfekt/Passiv-Partizipien, aber für
  Deutsch geht Analyse mit einer Partizipform.


\end{itemize}

}



\subsection*{Übungsaufgaben}

\frame{
\frametitle{Übungsaufgaben}

\begin{enumerate}
\item Welche der NPen in den folgenden Sätzen haben strukturellen, welche lexikalischen Kasus?
\eal
\ex Der Junge lacht.  
\ex Mich friert.    
\ex Er zerstört das Auto.
\ex Das dauert ein ganzes Jahr.
\ex Er hat nur einen Tag dafür gebraucht.
\ex Er denkt an den morgigen Tag.
\zl

\end{enumerate}

}





%      <!-- Local IspellDict: de_DE -->
