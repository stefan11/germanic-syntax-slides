\section{Organisatorisches}

\frame{
\frametitle{Organisatorisches}


\begin{itemize}
%% \item alle Teilnehmer bitte Mail in Maillisten eintragen\\
%%       zugänglich von \url{http://www.cl.uni-bremen.de/~stefan/Lehre/HPSG/}
%%   \begin{itemize}
%%   \item Mail-Liste für alle Linguistik-Studierenden
%%   \item Mail-Liste für alle Teilnehmer dieser Veranstaltung
%%   \end{itemize}
\item Bitte bei blackboard anmelden
\pause
\item Telefon und Sprechzeiten siehe: \url{https://hpsg.hu-berlin.de/~stefan/}
\pause
\item Beschwerden, Verbesserungsvorschläge:
      \begin{itemize}
      \item mündlich
      \item per Mail oder 
      \item anonym über das Web:\\
            \url{https://hpsg.hu-berlin.de/~stefan/Lehre/}
      \end{itemize}
\item Bitte unbedingt Mail-Regeln beachten!\\
\url{https://hpsg.hu-berlin.de/~stefan/Lehre/mailregeln.html}
\end{itemize}
}

\frame{
\frametitle{Materialien}


\begin{itemize}

\item Information zur Vorlesung:\\
\url{https://hpsg.hu-berlin.de/~stefan/Lehre/Germanisch/}

\item Wiederholung/Grundlagen: \citew[Kapitel~1--2]{MuellerGTBuch2}, \citew{Abramowski2024a}
\item Einführungsbuch zur HPSG: \citew{MuellerLehrbuch3}
    
\end{itemize}
}

\frame{
\frametitle{Vorgehen}


\begin{itemize}
\item Handouts ausdrucken, immer mitbringen und persönliche Anmerkungen einarbeiten
\item Veranstaltungen vorbereiten
\item Veranstaltungen unbedingt nacharbeiten!
      \begin{itemize}
      \item Kontrollfragen
      \item Übungsaufgaben
      \end{itemize}
\item Fragen!
\end{itemize}
}

%% \frame{
%% \frametitle{Leistungen}

%% \begin{itemize}
%% \item Aktive Teilnahme
%%       \begin{itemize}
%%       \item Klausur in letzter Veranstaltung %bei Vorlesung oder regelmäßige Abgabe der  Übungsaufgaben
%% %      \item Referat + Handout im Seminar
%%       \item In Vorlesung und Seminar: Abgabe von je drei Stichpunkten zur vorigen Veranstaltung und mindestens zwei Fragen
%%       \end{itemize}
%% \item %Klausur (90 Minuten) oder Hausarbeit in einer  (etwa 20 Seiten)
%%       Hausarbeit in einer der Veranstaltungen\\
%%       (\url{https://hpsg.hu-berlin.de/~stefan/Lehre/hausarbeiten.html})
%% \end{itemize}

%% Ideale Zeitaufteilung (für das gesamte Modul):

%% \begin{tabular}{ll}
%% Präsenzstudium         & 60 h \\
%% Vor- und Nachbereitung & 150 h (75h/17 = 4,41h!)\\
%% Prüfungsvorbereitung   & 240 h\\
%% \end{tabular}

%% }

\frame{
\frametitle{Leistungen für Vorlesung}

\begin{itemize}
\item Aktive Teilnahme
      \begin{itemize}
      \item Studiengang BA Germanistische Linguistik: Klausur
      \item für Teilnahmebescheinigung regelmäßige Abgabe der Übungsaufgaben in Moodle\\
            = jeweils zwei Fragen zum zu lesenden Text stellen
%      \item Referat + Handout im Seminar
      \end{itemize}
%\item %Klausur (90 Minuten) oder Hausarbeit in einer  (etwa 20 Seiten)
%      Hausarbeit in einer der Veranstaltungen\\
%      (\url{https://hpsg.hu-berlin.de/~stefan/Lehre/hausarbeiten.html})
\end{itemize}

Zeitaufteilung für Vorlesung laut Studienordnung 2014 BA Linguistik:

\begin{tabular}{ll}
Präsenzstudium         & 25 h \\
Vor- und Nachbereitung & 35 h (35h/15 = 2,33h!)\\
Prüfungsvorbereitung   & 90 h (90h/15 = 6h!!!)\\
\end{tabular}

Zeitaufteilung für Vorlesung laut Studienordnung 2014 BA Deutsch:

\begin{tabular}{ll}
Präsenzstudium         & 25 h \\
Vor- und Nachbereitung & 35 h (35h/15 = 2,33h!)\\
\end{tabular}


}



\input ../../plagiat.tex

\section{Ziele der Veranstaltung}


\frame{
\frametitle{Ziele}


% \begin{itemize}[<+->]
% \item Vermittlung grundlegender Vorstellungen über Grammatik
% \item Vorstellung zweier Grammatiktheorien und deren Herangehensweisen
% \item Vergleich der Sprachen Deutsch, Dänisch, Niederländisch
% \end{itemize}

\begin{itemize}
\item Überblick über die germanischen Sprachen
\item Detailliertere sprachvergleichende Diskussion ausgewählter syntaktischer Phänomene
\end{itemize}


}
