%%%%%%%%%%%%%%%%%%%%%%%%%%%%%%%%%%%%%%%%%%%%%%%%%%%%%%%%%
%%   $RCSfile: hpsg-slides.tex,v $
%%  $Revision: 1.3 $
%%      $Date: 2006/05/17 12:19:28 $
%%     Author: Stefan Mueller (DFKI)
%%    Purpose: 
%%   Language: LaTeX
%%%%%%%%%%%%%%%%%%%%%%%%%%%%%%%%%%%%%%%%%%%%%%%%%%%%%%%%%
%% $Log: hpsg-slides.tex,v $
%% Revision 1.3  2006/05/17 12:19:28  stefan
%% *** empty log message ***
%%
%% Revision 1.2  2004/08/14 15:44:45  stefan
%% konstituentenreihenfolge
%%
%% Revision 1.1  2004/06/21 19:14:48  stefan
%% alte Version vor LaTeX-Beamer
%%
%% Revision 1.1  2002/01/09 20:06:23  stefan
%% Initial revision
%%
%% Revision 1.1  2001/10/21 17:01:35  stefan
%% Initial revision
%%
%%%%%%%%%%%%%%%%%%%%%%%%%%%%%%%%%%%%%%%%%%%%%%%%%%%%%%%%%


\documentclass[xcolor=dvipsnames,10pt]{beamer}


\usepackage{hu-beamer-includes-pdflatex}%,beamer-movement}



% Checkmark
%\usepackage{tikz} % Examples and documentation: http://www.texample.net/tikz
%\usepackage{bbding}
\makeatletter
\newcommand{\dingfamily}{\fontencoding{U}\fontfamily{ding}\selectfont}
\newcommand{\@chooseSymbol}[1]{{\dingfamily\symbol{#1}}}
\newcommand{\Checkmark}{\@chooseSymbol{'041}}
\makeatother

%\fuberlinlogon{0.89cm}

% creates an unggly bibliography
%\usepackage[T1]{fontenc}


\usepackage{jambox}

\usepackage{epsfig}
\usepackage{tabularx}

\usepackage{tikz-grid}

%\renewcommand{\trace}{\raisebox{0.2ex}{\_}\rule{0cm}{0.7em}}


%\usepackage[T4,T1]{fontenc}

% die machen \! kaputt
%\usepackage{tipa,tipx,phonetic}

% \TH, \th, \dh, \DH
\usepackage{wasysym}

\let\TH=\Thorn
\let\th=\thorn

% Das ʒ muss aus tipa.sty muss auf textyogh umgebogen werden, da war \textezh drin.


\let\textezh=\textyogh
\let\textoopen=\textopeno
%\let\texteopen=\textopene gibts nicht
\let\texteopen=\textepsilon
\let\textgammalatinsmall=\textbabygamma
\let\ng=\engma
\let\textvhook=\textscriptv

%\DeclareUnicodeCharacter {720}{\textlengthmark }
%\DeclareUnicodeCharacter {635}{\textturnrrtail\,}

%% \nodemargin5pt%\treelinewidth2pt\arrowwidth6pt\arrowlength10pt
%% \selectlanguage{german}
%% \psset{nodesep=5pt} %,linewidth=0.8pt,arrowscale=2}
%% \psset{linewidth=0.5pt}

\usepackage[customcolors]{hf-tikz}

% for subnodes in trees
% is in hu-beamer-...
%% \usepackage{tcolorbox}
%% \tcbuselibrary{skins}
%% \newtcbox{\mybox}[1][]{empty,shrink tight,nobeforeafter,on line,before upper=\vphantom{gM},remember as=#1,top=2pt,bottom=2pt}


%\beamertemplateemptyfootbar

%\hypersetup{pdfauthor={Stefan Müller (unter Benutzung von Material von Peter Gallmann, FSU Jena)}}
%\hypersetup{pdftitle={Head-Driven Phrase Structure Grammar für das Deutsche}}


%\subject{Generative Grammatik für das Deutsche}


%\beamerdefaultoverlayspecification{<+->}

\let\mc=\multicolumn


\newcommand{\sigle}[1]{\tiny{#1}}



%\mode<beamer>{\beamertemplatebackfindforwardnavigationsymbolshorizontal}

\title{Besprechung der Evaluationsergebnisse}

%\includeonly{organisatorisches-germanisch-vl}
%\includeonly{germanisch-phaenomene}
%\includeonly{germanisch-valenz-scrambling}
%\includeonly{germanisch-verbalkomplex}
%\includeonly{germanisch-verbstellung}
%\includeonly{germanisch-extraction}
%\includeonly{germanisch-mehr-vf}
%\includeonly{germanisch-subjekt}
%\includeonly{germanisch-passiv}
%\includeonly{germanisch-expletives}

\begin{document} 
\hypersetup{bookmarksopen=false}

\huberlintitlepage

%\part{Einleitung}



%\beamertemplatecopyrightfootframenumber

\exewidth{(35)}


\frame{
\frametitle{Bluff}

\begin{itemize}
\item Anforderungen für die aktive Teilnahme:\\
      zwei Fragen (wenn keine Frage, dann "`keine Frage"' schreiben)
\item Nachfrage: "`Aber wir unterschreiben die Scheine doch selbst."'
\item Ich: "`Ja, trotzdem."'
\item Ich hatte nichts in der Hand und habe trotzdem etwas verlangt.\\
Das war Rumgeeier und Bluff.

\bigskip

\item Nach Gespräch mit Frau Schlachter und Absicherung bei Frau Kabelitz war ich der Annahme, ich könnte die Scheine
  selbst unterschreiben.

\item Missverständnis Fachschaft: Annahme, ich hätte Gespräch erfunden und so geblufft.

\item Dazu passend: Mail von mir, dass Sie nicht alle einzeln mit Frau Kabelitz bzw.\ Frau
  Schlachter reden sollten. Das war zu diesem Zeitpunkt schon mehrfach passiert (Zeugen). Auch mit Hubert
  Truckenbrot.

\item Ich wollte die Kolleginnen vor durch mich verursachter Mehrarbeit schützen.
\end{itemize}


}


\frame{
\frametitle{Die Niderschrift von der Smaragdenen Felswand}



Ferner hat der Zen-Meister Huan-djian von Yün-djü oder Wolkenheim bei der Unterweisung einmal gesagt:
"`Wenn der Löwe einen Elefanten reißt, so nimmt er seine ganze Kraft zusammen; und wenn er einen Hasen
reißt, so nimmt er seine ganze Kraft zusammen."' Nun war da ein Mönch der fragte: "`Nun möchte
ich nur wissen, was für eine Kraft das ist, die der Löwe ganz zusammennimmt."' Yün-djü antwortete:
"`Es ist die \alert{Kraft der Wahrheit ohne Falsch}."'\footnote{
aus Bi-Yän-Lu, Meister Yüan-wu's Niederschrift von der Smaragdenen Felswand, Ullstein Materialien (TB, 3 Bände in einem Band), Hanser Verlag, 3 Bände, gebunden)
}


}


\frame{
\frametitle{Bewertung BA-Deutsch}

\centerline{
\includegraphics[width=0.85\textwidth]{Evaluation/Bewertung-allgemein-BA-Deutsch}
}
}

\frame{
\frametitle{Bewertung Linguistik}

\centerline{
\includegraphics[width=0.85\textwidth]{Evaluation/Bewertung-allgemein-Linguistik}
}
}


\frame{
\frametitle{Positives BA-Deutsch}

\includegraphics[width=\textwidth]{Evaluation/Positives-BA-Deutsch}

}

\frame{
\frametitle{Positives Linguistik}

\includegraphics[width=\textwidth]{Evaluation/Positives-Linguistik}

}

\frame{
\frametitle{Verbesserungsvorschläge BA-Deutsch}

\includegraphics[width=\textwidth]{Evaluation/Verbesserungsvorschlaege-BA-Deutsch}

}

\frame{
\frametitle{Verbesserungsvorschläge BA-Deutsch (II)}

\includegraphics[width=\textwidth]{Evaluation/Verbesserungsvorschlaege-BA-Deutsch-2}

}
\frame{
\frametitle{Verbesserungsvorschläge Linguistik}

\includegraphics[width=0.85\textwidth]{Evaluation/Verbesserungsvorschlaege-Linguistik}

}

%% \frame{
%% \frametitle{Ärger}





%% }

\frame{
\frametitle{Sarkasmus und Ironie}

\begin{itemize}
\item Ich bitte um Entschuldigung.
\pause
\item Ich werde nie wieder ironisch sein!
\end{itemize}

}

\frame{
\frametitle{Aufreger Streik: Beteiligung an studentischen und universitäten Belangen}

\begin{itemize}
\item 

Mikro war an und lag auf dem Tisch.

\item Ich konnte nicht verstehen, wie jemand sich von einem Link mit Informationen zu einem
  aktuellen Vorgang an der Uni belästigt fühlen kann.

\item Ich habe überreagiert.

\end{itemize}


}

\frame{
\frametitle{Machtmißbrauch}

Das ist Machtmißbrauch:
\begin{itemize}
\item Noten gegen Sex
\item Veröffentlichung von Arbeitsergebnissen unter dem Namen des Vorgesetzten
\item Erzwungene Koautorenschaft der Arbeitsgruppenleiterin bzw. des Arbeitsgruppenleiters
\end{itemize}

\pause

Das ist kein Machtmißbrauch:
\begin{itemize}
\item Man unterschreibt einen Schein, wenn mit der Prüfungsordnung vereinbar.
\end{itemize}

\pause
Das ist Pech:
\begin{itemize}
\item Man fragt die falschen Leute.
\end{itemize}

\pause
PS: Ich unterschreibe nicht mal mit Titel, Ihr Stefan Müller.



}




\frame{
\frametitle{Das brauche ich als Lehrerin bzw. Lehrer nicht!}

\begin{itemize}
\item Stellen Sie sich andere Fächer vor:\\
      Mathematik, Physik, Chemie, Biologie, Informatik.

\begin{itemize}
\item Mathematik: Wieso soll ich mich mit Höherer Algebra und Differentialrechnung beschäftigen? Das
  kommt in der Schule eh nicht dran.

\item Informatik: Wieso soll ich mich mit Schaltkreisentwurf beschäftigen?\\ Werde ich in der Schule
  doch nicht machen.

\item Biologie: Wieso soll ich mich mit Kreuzspinne und Kreuzschnabel beschäftigen?\\ Das steht doch nicht im Lehrplan.

\end{itemize}

\pause
\item Wenn Sie nur unmittelbar Schulrelevantes lernen wollen,\\
      wäre die Uni eine Wiederholung Ihres Schulstoffes. Wie öde!

\pause
\item Es gibt Menschen, die sind wissbegierig und andere, die sind es nicht.

\item Wie wollen Sie andere, heranwachsende Menschen dazu bringen,\\
      nach Wissen zu gieren?
\end{itemize}

}

\frame{
\frametitle{Michael Hahn, früher Sindelfingen, jetzt Stanford}

%\vfill
\medskip
\includegraphics[width=0.76\textwidth]{Evaluation/Michael-Hahn-HPSG2013}

{\tiny Michael Hahn, 2013 auf der Internationalen Konferenz für HPSG. CC-BY-NC-ND Stefan Müller, LingPhot.}
\vfill
}


\frame{
\frametitle{Für LehrerInnen und WissenschaftlerInnen gleichermaßen}

\begin{itemize}[<+->]
\item Patrick Blackburn im Vortrag \emph{How to give a good talk}:\\
 Es muss klar sein, dass es sich nur um die Spitze des Eisbergs handelt.\\
Der Hauptteil befindet sich unter dem Wasser!

\item Kampfsportarten: Für einen sauberen Tritt Richtung Kopf müssen sie dehnbarer sein als es die
  Tritthöhe verlangt.
\end{itemize}

}




\frame{
\frametitle{Tochter}

\begin{itemize}
\item Fachschaft: Ihre zehnjährige Tochter ist in der Diskussion irrelevant.
\pause
\item Was sagt der Clown? "`Ich hab ja noch Reserve."'
\pause
\item Mein Sohn geht aufs Gymnasium.
\pause
\item Auch dort gibt es Subjekte und Objekte, Aktiv und Passiv.
\pause
\item Sneak Preview: In den kommenden Lehrplänen wird Valenz eine Rolle spielen (Marc Felfe).
\pause

\item Fachschaft besteht aus guten und engagierten StudentInnen,\\
 weshalb sie Anmerkungen bzgl. Kenntnisstand als Beleidigungen auffassen.

\pause
\item Aber: Ca. 2015 haben in der großen Einführungsvorlesung an der FU\\
      (450 TeilnehmerInnen) 80\,\% der Anwesenden für \emph{mir} als Subjekt gestimmt:
\ea
Mir graut vor der Prüfung.
\z
\pause
\item Im Jahr danach nach vorausgegangener Besprechung des Beispiels wieder 80\,\%.

\end{itemize}

}

\frame{
\frametitle{Ziele (ergänzte Folie in der Veranstaltung)}


% \begin{itemize}[<+->]
% \item Vermittlung grundlegender Vorstellungen über Grammatik
% \item Vorstellung zweier Grammatiktheorien und deren Herangehensweisen
% \item Vergleich der Sprachen Deutsch, Dänisch, Niederländisch
% \end{itemize}

\begin{itemize}
\item Überblick über die germanischen Sprachen
\item Detailliertere sprachvergleichende Diskussion ausgewählter syntaktischer Phänomene
\end{itemize}

Lehramtsrelevante Teilziele
\begin{itemize}
\item Wiederholung und Festigung grundlegender Begriffe:

\begin{itemize}
\item Wortarten
\item Kasus und andere morphosyntaktische Merkmale
\item Grammatische Funktionen
\item syntaktische Struktur des Deutschen und der Nachbarsprachen
      
      Hauptsätze/Nebensätze/Fragen

\item Valenz (Unterscheidung Argument/Adjunkt)
\item Aktiv/Passiv
\end{itemize}
\item Ansonsten: Blick über den Schul-Tellerrand
\end{itemize}


}


\frame{
\frametitle{Warum soll ich ein Buch auf Englisch lesen?}

\begin{itemize}
\item "`Es wäre besser, wenn deutsche syntaktische Phänomene auf Deutsch erklärt werden."'
\item Sie studieren an einem Institut für "`(Deutsche Sprache) und Linguistik"'. 
\item Das ist etwas Tolles! Es gibt Professuren für Allgemeine Sprachwissenschaft.
\item Wenn es darum geht, unsere kognitiven Fähigkeiten zu verstehen (was uns vom Affen
  unterscheidet), dann müssen wir Sprache allgemein betrachten und verstehen.
\item Deutsch ist als Wissenschaftssprache tot. (Ulrich Ammon)
\item Die deutsche Fachliteratur ist fünf Jahre hinterher und minderwertig.
\item Sie brauchen die Kenntnis der Fachsprache.
\item Englischsprachige Bücher, die offen entstehen (Open Science, Open Review, Open Access), werden
  besser als deutsche Bezahl-Bücher.

Sie haben mehr Nutzer, es gibt mehr Feedback.
\end{itemize}

}

\frame{
\frametitle{Andere Evaluationsergebnisse: Wir können auch anders!}

\centerline{
\includegraphics[width=0.85\textwidth]{Evaluation/Grammatiktheorie-2018}
}
}

\frame{
\frametitle{Andere Evaluationsergebnisse: Wir können auch anders!}


Evaluation HU WS2016 (Grundkurs Linguistik): 

\includegraphics[width=.9\textwidth]{Evaluation/netter-kompetenter-Dozent-GK2016WS}

\pause


Evaluation HU WS2016 (HPSG Master): 

\includegraphics[width=.9\textwidth]{Evaluation/nett-und-humorvoll}

}

\frame{
\frametitle{Andere Evaluationsergebnisse: Wir können auch anders!}

Evaluation Jena SS2018 (Grammatikimplementation Master):
\includegraphics[width=.9\textwidth]{Evaluation/2018-entspannte-Arbeitsatmosphaere}

}

\frame{
\frametitle{Wie geht es weiter?}

\begin{itemize}[<+->]
\item Fragen sind optional.
\item Fragen helfen mir zu verstehen, wo noch Erklärungen hinzugefügt werden können bzw.\ müssen.
\item Es wird zu jedem Thema einen Satz Multiple-Choice-Aufgaben zum Üben geben.

\item Es wird (noch mehr) Übungsaufgaben im Buch geben.
\item Das Buch wird auch ansonsten vervollständigt und überarbeitet.

\bigskip

\item Ich werde in alle Kommissionen zu neuen Studienordnungen gehen.
\item Kommen Sie auch! 
\item Engagieren Sie sich in der Fachschaft! 
\item Machen Sie in der universitären Selsbtverwaltung mit (Berufungskommissionen usw.)!

\end{itemize}


}

\frame{
\frametitle{Angebot}

\begin{itemize}
\item Lassen Sie uns miteinander reden!
\item Ich habe das der Fachschaft mehrfach angeboten.
\item In kleinerem oder größerem Kreis.
\item Hier oder da.
\end{itemize}


}


\frame{
\frametitle{Wie geht es weiter?}

\centerline{\fontsize{60}{48} \selectfont Ho!}

}




\appendix
% muß immer geladen werden, wegen Referenzen
\input{hu-literatur}


\end{document}


% Local variables:
% mode: lazy-lock
% End:




Fragen:

1. Kann die Definition für strukturellen und lexikalischen Kasus für andere Sprachen, auch außerhalb der germanischen Sprachen, übernommen werden oder müsste man sie von Grund auf neu definieren? 

2. Wie könnte man feststellen, ob ein Kasus einer Sprache, die noch nicht sehr gut beschrieben ist, ein struktureller oder lexikalischer Kasus ist? 

3. Welche Schritte sind notwendig um das Kasussystem einer Sprache zu definieren. Wie geht man vor? 

4. Wie geht man als Student damit um, wenn bezüglich eines Merkmals oder linguistischen Phänomens eine Kontroverse Meinung herrscht, wie z. B. bei der Einordnung des Dativs (lexikalisch/strukturell).




1.	Im Satz (85) ist ein Subjekt Haraldi oder Sigga?
Laut der Regel in Isländisch steht Subjekt direkt nach dem finiten Verb, egal in welchem Kasus (Haraldi in Dat.). Aber in diesem Satz gibt es auch eine Nominalgruppe  (Sigga).
2.	Wird Objektskasus nur in den Sätzen mit Passiv beobachtet?
  



1. Wenn die HPSG-Theorie eine lexikonbasierte Theorie ist und die
Syntaxstruktur vom Verb abhängen sollte, warum sollte man unterscheiden
zwischen strukturellem und lexikalischem Kasus?

2. Sind die semantische Kasus hier eine Subkategorie wie struktureller und
lexikalischer Kasus oder eine Tiefstruktur unter den syntaktischen Kasus?
Gibt es Position oder Beschreibung für semantische Kasus in Fillmores Case
Grammar wie agentive, objective, benefactive, locative, usw. in HPSG? 3.
Da ich Deutsch als Fremdsprache lernen, bedeutet die Grammatik für mich
die Regeln zu befolgen. Eine der Regeln ist, dass Objektive in der
folgenden Reihenfolge angeordnet werden: Akkusativpronomen < Dativpronomen
< Dativnomen < Akkusativnomen.
Also ist das Scrambling auf Seite 40 eine ungrammatische Ausnahme der
Regel oder ein häufiges Phänomen, oder ist die Regel nur eine
Standardvariante? Was ist das Ziel von HPSG und die andere
Grammatiktheorien speziell? Wie unterscheidet sich die Grammatik für
formale Beschreibung von Sprachlehre?




