%% -*- coding:utf-8 -*-
\section{Organisatorisches}

\frame{
\frametitle{Organisatorisches}


\begin{itemize}
%% \item alle Teilnehmer bitte Mail in Maillisten eintragen\\
%%       zugänglich von \url{http://www.cl.uni-bremen.de/~stefan/Lehre/HPSG/}
%%   \begin{itemize}
%%   \item Mail-Liste für alle Linguistik-Studierenden
%%   \item Mail-Liste für alle Teilnehmer dieser Veranstaltung
%%   \end{itemize}
\item Bitte bei moodle anmelden (gibt kein Passwort)
\pause
\item Telefon und Sprechzeiten siehe: \url{https://hpsg.hu-berlin.de/~stefan/}
\pause
\item Beschwerden, Verbesserungsvorschläge:
      \begin{itemize}
      \item mündlich
      \item per Mail oder 
      \item anonym über das Web:\\
            \url{https://hpsg.hu-berlin.de/~stefan/Lehre/}
      \end{itemize}
\item Bitte unbedingt Mail-Regeln beachten!\\
\url{https://hpsg.hu-berlin.de/~stefan/Lehre/mailregeln.html}
\end{itemize}
}

\frame{
\frametitle{Materialien}


\begin{itemize}

\item Information zur Vorlesung:\\
\url{https://hpsg.hu-berlin.de/~stefan/Lehre/Germanisch/}

\item Wiederholung/Grundlagen: \citew[Kapitel~1--2]{MuellerGTBuch2},\\
      \citew{GrundkursReader}
\item Einführungsbuch zur HPSG: \citew{MuellerLehrbuch3}
    
\end{itemize}
}

\frame{
\frametitle{Vorgehen}


\begin{itemize}
\item Handouts ausdrucken, immer mitbringen und persönliche Anmerkungen einarbeiten
\item Veranstaltungen vorbereiten
\item Veranstaltungen unbedingt nacharbeiten!
      \begin{itemize}
      \item Kontrollfragen
      \item Übungsaufgaben
      \end{itemize}
\item Fragen!
\end{itemize}
}


\section{Leistungen}
\frame{
\frametitle{Leistungen}

BA Ling: Modul 3: Grammatik II: Der Satz\\
BA Deutsch: Modul 6 Wort und Satz

\begin{itemize}
      \item Studiengang BA Germanistische Linguistik: Klausur in der letzten Woche im Vorlesungsraum
        bzw. dann im zweiten Prüfungszeitraum
      \item für alle (freiwillig): 
        \begin{itemize}
        \item kleine Tests zur Vertiefung
        \item zwei Fragen zum zu lesenden Text
        \end{itemize}
%Hausarbeit (\url{https://hpsg.hu-berlin.de/~stefan/Lehre/hausarbeiten.html})
\end{itemize}

Ideale Zeitaufteilung:

\begin{tabular}{@{}lr@{~}l}
Präsenzstudium Vorlesung  & 25 h \\
Vor- und Nachbereitung    & 35 h & (35/15 = 2 h 20 min für jede Sitzung)\\

Klausurvorbereitung       & 90 h & (90h/15 = 6h)
\end{tabular}

Das Modul entspricht 9 bzw.\ 8 Leistungspunkten.


}


%\input ../../plagiat.tex

\section{Ziele der Veranstaltung}


\frame{
\frametitle{Ziele}


% \begin{itemize}[<+->]
% \item Vermittlung grundlegender Vorstellungen über Grammatik
% \item Vorstellung zweier Grammatiktheorien und deren Herangehensweisen
% \item Vergleich der Sprachen Deutsch, Dänisch, Niederländisch
% \end{itemize}

\begin{itemize}
\item Überblick über die germanischen Sprachen
\item Detailliertere sprachvergleichende Diskussion ausgewählter syntaktischer Phänomene
\end{itemize}

Lehramtsrelevante Teilziele
\begin{itemize}
\item Wiederholung und Festigung grundlegender Begriffe:

\begin{itemize}
\item Wortarten
\item Kasus und andere morphosyntaktische Merkmale
\item Grammatische Funktionen
\item syntaktische Struktur des Deutschen und der Nachbarsprachen
      
      Hauptsätze/Nebensätze/Fragen

\item Valenz (Unterscheidung Argument/Adjunkt)
\item Aktiv/Passiv
\end{itemize}
\item Ansonsten: Blick über den Schul-Tellerrand
\end{itemize}


}
