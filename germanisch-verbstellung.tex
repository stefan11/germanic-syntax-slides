%% -*- coding:utf-8 -*-


\subtitle{Verbstellung: Verberst- und Verbzweitstellung in den V2-Sprachen}

\section{Verbstellung: Verberst- und Verbzweitstellung in den V2-Sprachen}

\huberlintitlepage[22pt]


\frame{
\frametitle{Literaturhinweis}


Zu diesem Abschnitt gibt es das Kapitel~6 in \citew{MuellerGermanic}.

Müller, Stefan, \citeyear{MuellerGermanic}. \emph{Germanic Syntax}. Berlin: Language Science
Press. In Vorbereitung. 



}

\subsection{Verberststellung}

\frame{
\frametitle{Lehrmeinung: Deutsch SPO}

\begin{itemize}
\item Behauptung: Deutsch ist Subjekt Prädikat Objekt
\pause
\item Das ist das häufigste Muster,\\
      wenn man nur Aussagesätze mit Subjekt, Prädikat und Objekt ansieht.
\pause

\item Es gilt aber schon nicht mehr für psychologische Prädikate:
\ea
Dem Mann gefallen die Bilder.
\z

\pause
\item Es gilt nicht für freien Text, in dem insbesondere Adverbialien vorkommen,\\ die die erste Stelle
  im Satz einnehmen können.

\pause
\item Deutsch ist eine SOV-Sprache und außerdem noch eine Verbzweitsprache (V2).
\pause
\item V2-Sprachen:\\
Beliebige Konstituenten können vor das finite Verb gestellt werden.

Alle germanischen Sprachen außer Englisch.
\end{itemize}

}

\frame{
\frametitle{Lehrmeinung: Deutsch SPO, nachgezählt}

\small

taz, 01.02.2013:

\rotit{Die Linke} fordert in dem Entwurf auch eine Vermögensteuer von fünf Prozent auf Privatvermögen
ab einer Million Euro, eine stärkere Besteuerung von Erbschaften und eine einmalige Vermögensabgabe
für Reiche. \gruenbf{Ab Jahreseinkommen von 65.000 Euro} soll ein Spitzensteuersatz von 53 Prozent gelten, das
Ehegattensplitting abgeschafft werden.

\rotit{SPD-Fraktionsvize Joachim Poß} kritisierte die Pläne als "`jenseits aller Vernunft und
Realitätstauglichkeit"'. \gruenbf{Mit solchen Vorschlägen} werde das wichtige Thema der Steuergerechtigkeit
diskreditiert. \gruenbf{Zwar} sei es notwendig, Spitzenverdiener stärker an der Finanzierung wichtiger
Zukunftsaufgaben zu beteiligen, "`aber mit Augenmaß und Vernunft"'. \gruenbf{Für eine Begrenzung von
Managergehältern} setzt sich auch die SPD ein.

\alt<beamer>{rot}{kursiv} = Subjekt = 2, \alt<beamer>{grün}{fett} = Nicht-Subjekt = 4

natürlich nicht repräsentativ \ldots

}

\frame[shrink=40]{
\frametitle{A9 soll Teststrecke werden}

taz: 27.01.2015

\gruenbf{Für selbstfahrende Autos} soll es in Deutschland nach Angaben von Bundesverkehrsminister
Alexander Dobrindt (CSU) bald eine Teststrecke geben. \gruenbf{Auf der Autobahn A9 in Bayern} \rotit{sei ein Pilotprojekt „Digitales
TestfeldAutobahn“ geplant}, wie aus einem Papier des Bundesverkehrsministeriums
hervorgeht. \gruenbf{Mit den ersten Maßnahmen für diese Teststrecke} solle schon in diesem Jahr begonnen
werden. \gruenbf{Mit dem Projekt} soll die Effizienz von Autobahnen generell
gesteigert werden. \gruenbf{„\rotit{Die Teststrecke} soll so digitalisiert und technisch ausgerüstet
werden, dass es dort zusätzliche Angebote der Kommunikation zwischen Straße und Fahrzeug
wie auch von Fahrzeug zu Fahrzeug geben wird“}, sagte Dobrindt zur Frankfurter Allgemeinen Zeitung.
\gruenbf{Auf der A9} sollten sowohl Autos mit Assistenzsystemen als
auch später vollautomatisierte Fahrzeuge fahren können. \gruenbf{Dort}
soll die Kommunikation nicht nur zwischen Testfahrzeugen,
sondern auch zwischen Sensoren an der Straße und den Autos
möglich sein, etwa zur Übermittlung von Daten zur Verkehrslage
oder zum Wetter. \gruenbf{\rotit{Das Vorhaben}
solle im Verkehrsministerium von einem runden Tisch mit Forschern
und Industrievertretern begleitet werden,} sagte Dobrindt. \rotit{Dieser} solle sich unter anderem
auch mit den komplizierten Haftungsfragen beschäftigen.
Also: \rotit{Wer} zahlt eigentlich, wenn ein automatisiertes Auto
einen Unfall baut?
\gruenbf{[\gruenbf{Mithilfe der Teststrecke}] solle
die deutsche Automobilindustrie auch beim digitalen Auto
„Weltspitze sein können“,} sagte der CSU-Minister. \rotit{Die deutschen
Hersteller} sollten die Entwicklung nicht Konzernen wie etwa
Google überlassen. \gruenbf{Derzeit} ist Deutschland noch
an das „Wiener Übereinkommen
für den Straßenverkehr“ gebunden,
das Autofahren ohne Fahrer
nicht zulässt. \gruenbf{Nur unter besonderen
Auflagen} sind Tests möglich.
\rotit{Die Grünen} halten die Pläne für
unnütz. \rotit{Grünen-Verkehrsexpertin
Valerie Wilms} sagte der Saarbrücker
Zeitung: „\rotit{Der Minister}
hat wichtigere Dinge zu erledigen,
als sich mit selbstfahrenden
Autos zu beschäftigen.“ \rotit{Die Technologie}
sei im Verkehrsbereich
nicht vordringlich, \gruenbf{auch} stehe sie
noch ganz am Anfang.
\gruenbf{Aus dem grün-rot regierten
Baden-Württemberg – mit dem
Konzernsitz von Daimler –} kamen
hingegen andere Töne. \gruenbf{\rotit{Was
in Bayern funktioniere,} müsse
auch in Baden-Württemberg
möglich sein,} sagte Wirtschaftsminister
Nils Schmid (SPD). \gruenbf{Von
den topografischen Gegebenheiten}
biete sich die Autobahn A81
an.

\alt<beamer>{rot}{kursiv} = Subjekt = 11, \alt<beamer>{grün}{fett} = Nicht-Subjekt = 16


natürlich nicht repräsentativ \ldots
}

\frame{
\frametitle{Subjekte in Korpora}

\begin{itemize}
\item \citet{HK2005a} 38.342 und 22.087 Bäume aus TüBa-D/S und Z
\pause
\item gesprochene und geschriebene Sprache (\verbmobil und taz)
\pause
\item 50,3\,\% und 52,1\,\% der Sätze enthielten das Subjekt im Vorfeld.
\pause
\item Annahme von SVO-Stellung würde also auch nicht helfen, denn man müsste erklären, wie das
  Subjekt nachgestellt und etwas anderes vorangestellt wird.
\end{itemize}


}


\frame{
\frametitlefit{Motivation der Verbletztstellung als Grundstellung: Partikeln}

\citew%[S.\,34--36]
{Bierwisch63a}: Sogenannte Verbzusätze oder Verbpartikel\\
bilden mit dem Verb eine enge Einheit.
\eal
\ex weil er morgen \alert{anfängt}
\ex Er \alert{fängt} morgen \alert{an}.
\zl
Diese Einheit ist nur in der Verbletzstellung zu sehen, was dafür spricht,\\
diese Stellung als Grundstellung anzusehen.
}

\frame{
\frametitle{Stellung von Idiomen}

\eal
\judgewidth{?*}
\ex[]{
dass niemand dem Mann \alert{den Garaus macht}
}
\ex[?*]{
dass dem Mann \alert{den Garaus} niemand \alert{macht}
}
\ex[]{
Niemand \alert{macht} ihm \alert{den Garaus}.
}
\zl

Idiomteile wollen nebeneinader stehen (\mex{0}a,b).

Umstellung des Verbs ist abgeleitete Stellung. Nur zur Markierung des Satztyps.


}

\frame{
\frametitle{Stellung in Nebensätzen}

Verben in infiniten Nebensätzen und in durch eine Konjunktion eingeleiteten
finiten Nebensätzen stehen immer am Ende\\
(von Ausklammerungen ins Nachfeld abgesehen):
\eal
\ex Der Clown versucht, Kurt-Martin die Ware \alert{zu geben}.
\ex dass der Clown Kurt-Martin die Ware \alert{gibt}
\zl
}

\frame{
\frametitle{Stellung der Verben in SVO und SOV-Sprachen}

\citet{Oersnes2009b}: 
\eal
\ex dass er ihn gesehen$_3$ haben$_2$ muss$_1$
\ex 
\gll at han må$_1$ have$_2$ set$_3$ ham\\
     dass er muss haben sehen ihn\\
\zl
\pause

Nur das finite Verb wird umgestellt, die anderen Verben bleiben hinten:
\eal
\ex Muss er ihn gesehen haben?
\ex 
\gll Må han have set ham?\\
     muss er haben sehen ihn\\
\zl


}

\frame[shrink]{
\frametitle{Skopus}

\citew[Abschnitt~2.3]{Netter92}:
Skopusbeziehungen der Adverbien hängt von ihrerer Reihenfolge ab (Präferenzregel?):\\
Links stehendes Adverb hat Skopus über folgendes Adverb und Verb.

\eal
\ex weil er [absichtlich [nicht lacht]]
\ex weil er [nicht [absichtlich lacht]]
\zl
\pause
Bei Verberststellung ändern sich die Skopusverhältnisse nicht.
\eal
\ex Er lacht absichtlich nicht.
\ex Er lacht nicht absichtlich.
\zl

\pause
Analyse:
\eal
\ex Er lacht$_i$ [absichtlich [nicht \_$_i$]].
\ex Er lacht$_i$  [nicht [absichtlich \_$_i$]].
\zl

Struktur ist in (\mex{0}) und (\mex{-2}) genau gleich.

}

\frame{
\frametitle{Mitunter nur SOV-Stellung möglich}



\citet{Haider97c}, \citet{Meinunger2001a}: Manche Verben lassen in Verbindung mit \emph{mehr als} nur Verbletztstellung zu:


\eal
\ex[]{
dass Hans seinen Profit letztes Jahr \alert{mehr als verdreifachte}
}
\ex[]{
Hans hat seinen Profit letztes Jahr \alert{mehr als verdreifacht}.
}
\ex[*]{
Hans \alert{verdreifachte} seinen Profit letztes Jahr \alert{mehr als}.
}
\zl

\pause

\citet{Hoehle91b}, \citet[\page 62]{Haider93a}: Über Rückbildung entstandene Verben können oft
nicht getrennt/umgestellt werden:
\eal
\ex[]{
weil sie das Stück heute \alert{uraufführen}%\\
%     because they the play today play.for.the.first.time\\
%\glt `because they premiered the play today'
}
\ex[*]{
Sie \alert{uraufführen} heute das Stück.%\\
%     they play.for.the.first.time  today the play\\
}
\ex[*]{
Sie \alert{führen} heute das Stück \alert{urauf}.%\\
%     they guide today the play  {\sc prefix}.{\sc part}\\
}
\zl


Zu einem Überblick siehe \citew{MuellerGermanHandbook}.
}


\frame{
\frametitle{Dänisch}


\begin{itemize}
\item \rot{Negation} verbindet sich mit der \blau<1>{VP}:
\ea
\gll  at   Jens \rot{ikke} [\sub{VP} {\blau<1>{læser}} \blau<1>{bogen}]\\
      dass Jens nicht      {}        liest          Buch.{\sc def}\\
\glt `dass Jens das Buch nicht liest'
%      that Jens not  {} reads   book.{\sc def}\\
%\glt `that Jens does not read the book'
\z

\pause
\item In V2-Sätzen wird das \blau{finite Verb} links von der \rot{Negation} realisiert:

\ea
\gll  Jens \blau{læser} \rot{ikke} bogen.\\
      Jens liest        nicht      Buch.{\sc def}\\
\glt `Jens liest das Buch nicht.'
%%       Jens reads   not  book.{\sc def}\\
%% \glt `Jens is not reading the book.'
\z
\pause
\item Das wird von vielen als Evidenz für Verbumstellung gesehen:
\ea
\gll  Jens \blau{læser}$_i$ \rot{ikke} [\sub{VP} \_$_i$ bogen].\\
      Jens liest            nicht      {}        {}     Buch.{\sc def}\\
%      Jens reads      not  {} {}    book.{\sc def}\\
\z
\end{itemize}
\nocite{KS2002a}
}

\frame{
\frametitle{Entscheidungsfragen wie im Deutschen V1-Stellung}


\eal
\ex
\gll at Jens læser bogen\\
     dass Jens liest Buch.{\sc def}\\
\glt `dass Jens das Buch liest'
\ex
\gll Læser Jens bogen?\\
     liest Jens Buch.{\sc def}\\
\glt `Liest Jens das Buch?'
\zl

\pause

Analyse:
\eal
\ex
\gll at Jens [\sub{VP} læser bogen]\\
     dass Jens {} liest Buch.{\sc def}\\
\glt `dass Jens das Buch liest'
\ex
\gll Læser$_i$ Jens [\sub{VP} \_$_i$ bogen]?\\
     liest Jens {}        {}     Buch.{\sc def}\\
\glt `Liest Jens das Buch?'
\zl


}


\frame{
\frametitle{Verbumstellung im Deutschen als Informationsweitergabe}

%% \begin{tikzpicture}
%% \tikzset{level 1+/.style={level distance=3\baselineskip}}
%% \tikzset{frontier/.style={distance from root=12\baselineskip}}
%% %\draw (-3,-5) to[grid with coordinates] (4,0);
%% \Tree[.S
%%         [.{V \sliste{ S/\!/V }} 
%%           [.V liest$_j$ ] ]
%%         [.{S$/\!/$V}
%%            [.NP Jens ]
%%            [.{V$'$$\!/\!/$V}
%%              [.NP \edge[roof]; {das Buch} ]
%%              [.{V$\!/\!/$V} \_$_j$ ] ] ] ]
%% \draw[semithick,<->,color=green] (3.1,-3.9) ..controls +(south east:.5) and +(south west:.5)..(2.7,-3.9);
%% \draw[semithick,<->,color=green] (3.5,-3.7) ..controls +(east:.5) and +(east:.5)..(2.8,-2.5);
%% \draw[semithick,<->,color=green] (2.8,-2.3) ..controls +(east:.5) and +(east:.5)..(1.7,-1.1);
%% \draw[semithick,<->,color=green] (1.5,-0.9) ..controls +(north:.5) and +(north:.5)..(-0.8,-0.9);
%% \draw[semithick,<->,color=green] (-0.7,-1.1) ..controls +(south east:.2) and +(north east:.5)..(-1.0,-2.4);
%% \end{tikzpicture}}

~\vfill
\centerfit{
\begin{forest}
sm edges
[S
  [{V \sliste{ S$/\!/$V }} 
    [V [liest$_j$] ] ]
       [{S$/\!/$V}
           [NP [Jens] ]
           [{V$'$$\!/\!/$V}
             [NP [das Buch, roof] ]
             [{\mybox[v1]{V}$\!/\!/$\mybox[v2]{V}} [\_$_j$] ] ] ] ] ]
%\draw[semithick,<->,color=green] (v1.south)--(v2.south);
%% \draw[semithick,<->,color=green] (3.1,-3.9) ..controls +(south east:.5) and +(south west:.5)..(2.7,-3.9);
%% \draw[semithick,<->,color=green] (3.5,-3.7) ..controls +(east:.5) and +(east:.5)..(2.8,-2.5);
%% \draw[semithick,<->,color=green] (2.8,-2.3) ..controls +(east:.5) and +(east:.5)..(1.7,-1.1);
%% \draw[semithick,<->,color=green] (1.5,-0.9) ..controls +(north:.5) and +(north:.5)..(-0.8,-0.9);
%% \draw[semithick,<->,color=green] (-0.7,-1.1) ..controls +(south east:.2) and +(north
       %% east:.5)..(-1.0,-2.4);
\end{forest}
}
\vfill
}


\frame{
\frametitle{Skopus}

~\vfill
\centerfit{\begin{forest}
sm edges
[S
        [{V \sliste{ S$/\!/$V }} 
          [V [lacht$_j$] ] ]
        [{S$/\!/$V}
           [NP [er] ]
           [{V$'$$\!/\!/$V}
             [Adv [nicht] ]
             [{V$'$$\!/\!/$V}
               [Adv [absichtlich] ]
               [{V$\!/\!/$V} [\_$_j$] ] ] ] ] ]
\end{forest}
}

\vfill

}


\frame{
\frametitle{Verbumstellung im Dänischen}

~\vfill
\centerfit{\begin{forest}
sm edges
[S
        [{V \sliste{ S$/\!/$V }} 
          [V [læser$_j$] ] ]
        [{S$/\!/$V}
           [NP [Jens] ]
           [{VP$\!/\!/$V}
             [{V$\!/\!/$V} [\_$_j$] ] 
             [NP [bogen] ] ] ] ]
\end{forest}}
\vfill


}


\frame[shrink]{
\frametitle{Verbumstellung im Dänischen mit Negation}

~\vfill
\centerfit{\begin{forest}
sm edges
[S
        [{V \sliste{ S$/\!/$V }} 
          [V [læser$_j$] ] ]
        [{S$/\!/$V}
           [NP [Jens] ]
           [{VP$\!/\!/$V}
             [Adv [ikke] ]
             [{VP$\!/\!/$V}
               [{V$\!/\!/$V} [\_$_j$] ] 
               [NP [bogen] ] ] ] ] ]
\end{forest}}
\vfill

}



\frame{
\frametitle{Übungsaufgaben}


\begin{enumerate}
\item Skizzieren Sie die Analyse für die folgenden Beispiele:
\eal
\ex dass er darüber lachen wird
\ex Wird er darüber lachen?
\zl

\ea
\gll Arbejder Bjarne ihærdigt  på bogen.\\
     arbeitet Bjarne ernsthaft an Buch.{\sc def}\\\jambox{(Dänisch)}
\glt `Arbeitet Bjarne ernsthaft an dem Buch?'
\z

\end{enumerate}

}
