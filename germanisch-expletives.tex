%% -*- coding:utf-8 -*-

\subtitle{Eingebettete Sätze, Expletiva und Satztypenmarkierung}

\section{Eingebettete Sätze, Expletiva und Satztypenmarkierung}

\huberlintitlepage[22pt]



\frame{
\frametitle{Literaturhinweis}



Zu diesem Abschnitt gibt es das Kapitel~8 in \citew{MuellerGermanic}.

Müller, Stefan, \citeyear{MuellerGermanic}. \emph{Germanic Syntax}. Berlin: Language Science
Press. In Vorbereitung. 

}



\subsection{Konjunktional eingeleitete Nebensätze}

\frame{
\frametitle{Deutsch: eingebettete Sätze sind VL}

\begin{itemize}
\item Deutsch, Niederländisch, \ldots: V-letzt:

\ea
Ich weiß, dass Aicke das Buch heute gelesen hat.
\z

\pause
Stellung der anderen Konstituenten ist frei:
\eal
\ex Ich weiß, dass das Buch Aicke heute gelesen hat.
\ex Ich weiß, dass das Buch heute Aicke gelesen hat.
\zl

\end{itemize}

}

\frame{
\frametitle{Englisch: eingebettete Sätze SVO}


\begin{itemize}
\item Englisch: eingebettete Sätze SVO
\ea[]{
I  know that Kim has read the book yesterday.
}
\z

%% \pause
%% Andere Stellungen sind nicht möglich:
%% \eal
%% \ex[*]{
%% I know that has Max read the book yesterday.
%% }
%% \ex[*]{
%% I know that yesterday Max has read the book.
%% }
%% \zl

\end{itemize}

}



\frame[shrink=5]{
\frametitle{Dänisch: eingebettete Sätze SVO oder V2}


\begin{itemize}
\item Dänisch: eingebettete Sätze SVO oder V2
\ea[]{
\gll Jeg  ved, at   Gert \rot{ikke}  har læst  bogen          {i dag}.\\
     ich weiß dass Gert nicht hat gelesen Buch.{\sc def} heute\\\jambox{(SVO)}
}
\z
Negation hilft, Verbstellung zu bestimmen:
\ea[]{
\gll Jeg  ved, at   Gert har \rot{ikke}  læst  bogen          {i dag}.\\
     ich weiß dass Gert hat nicht gelesen Buch.{\sc def} heute\\\jambox{(V2)}
}
\z

\pause
Andere Konstituenten in Initial-Stellungen sind möglich, \dash klares V2:
\eal
\ex[]{
\gll Jeg  ved, at   {i dag} har Gert ikke læst bogen.\\
     ich weiß dass heute   hat Gert nicht gelesen Buch.{\sc def}\\
}
\ex[]{
\gll Jeg ved, at   bogen          har Gert ikke  læst {i dag}.\\
     ich weiß dass Buch.{\sc def} hat Gert nicht gelesen heute\\
}
\zl

%% \ex[*]{
%% \gll Jeg  ved, at   {i dag} Gert har læst bogen.\\
%%      ich weiß dass heute   Gert hat gelesen Buch.{\sc def}\\
%% }
%%
%% \ex[*]{
%% \gll Jeg  ved, at   bogen          Gert har læst {i dag}.\\
%%      ich weiß dass Buch.{\sc def} Gert hat gelesen heute\\
%% }
%% \zl
%%
\end{itemize}

}


\frame{
\frametitle{Jiddisch, Isländisch: eingebettete Sätze sind V2}

% Isländisch: Wikipedia (en)

\begin{itemize}
\item Jiddish: eingebettete Sätze sind V2 \citep[]{Diesing90a}:
\eal
\ex
\gll Ikh meyn  az   haynt hot Max geleyent dos bukh.\footnotemark\\
     ich   denke dass heute hat Max gelesen   das Buch\\
\footnotetext{\citew[\page 58]{Diesing90a}.}
\glt `Ich denke, dass Max heute das Buch gelesen hat.'

\ex% check!
\gll Ikh meyn  az   dos bukh hot Max geleyent.\\
     ich denke dass das Buch hat Max gelesen\\

\zl

\pause
Isländisch:
\ea 
\gll Engum         datt í hug,  að   vert  væri að reyna til     að kynnast honum.\footnotemark\\
     no.one.\DAT{} fell to mind that worth was  to try   \PREP{} to know    him\\%\icelandic
\footnotetext{\citew[\page 75]{Maling90a-u}.}
\glt `It didn't occur to anyone that it was worth trying to get to know him.'
\z



\end{itemize}

}






\subsection{Sätze mit Komplementierer}

\frame{
\frametitle{Deutsch: Komplementierer + Verbletzt, ohne Inversion}

\vfill
\centerfit{\begin{forest}
sm edges
[CP
       [C [dass] ]
       [S
        [{NP[\type{nom}]} [niemand] ]
        [V$'$
          [{NP[\type{acc}]} [ihn] ]
          [V [kennt] ]
           ] ] ]
\end{forest}}

\vfill
Komplementierer verlangt nicht umgestelltes Verb

\vfill

}


\frame{
\frametitle{Englisch: Komplementierer + SVO}


\vfill


\centerfit{\begin{forest}
sm edges
[CP
       [C [that] ]
       [S
        [{NP[\type{nom}]} [nobody] ]
        [VP
          [V  [knows] ]
          [{NP[\type{acc}]} [him] ] ] ] ]
\end{forest}}

\vfill

Komplementierer verlangt nicht umgestelltes Verb

\vfill

}


\frame{
\frametitle{Jiddisch: Komplementierer + V2}



\centerfit{\scalebox{.7}{\begin{forest}
sm edges
[CP
        [C [az;dass] ]
        [S
          [{Adv$_i$} [haynt;heute] ]
          [{S/Adv}
            [{V \sliste{ S/$\!$/V }} 
              [V [hot$_k$;hat] ] ]
            [{S$/\!/$V/Adv}
              [NP [Max;Max] ]
              [{VP$\!/\!/$V/Adv}
                [{V$\!/\!/$V}  [\_$_k$] ]
                [VP/Adv 
                  [VP
                    [V [geleyent;gelesen] ]
                    [NP [dos bukh;das Buch,roof] ] ]
                  [Adv/Adv [\_$_i$] ] ] ] ] ] ] ]
%% \draw[connect] (Adv/Adv.north east)   [bend right] to (VP/Adv.south east);
%% \draw[connect] (VP/Adv.north east)    [bend right] to (V/V/Adv.south east);
%% \draw[connect] (V/V/Adv.north east)   [bend right] to (S//V/Adv.south east);
%% \draw[connect] (S//V/Adv.north east)  [bend right] to (S/Adv.east);
%% \draw[connect] (S/Adv.north east)     [bend right] to (Adv);
\end{forest}}}



% \ea
% \gll Ikh meyn  az   haynt hot Max geleyent dos bukh.\footnotemark\\
%      ich   denke dass heute hat Max gelesen   das Buch\\
% \footnotetext{\citew[p.\,58]{Diesing90a}.}
% \glt `Ich denke, dass Max heute das Buch gelesen hat.'
% \z

}



\subsection{Interrogativnebensätze}


\frame{
\frametitle{Deutsch: Interrogativnebensätze \emph{w} + VL}

\begin{itemize}
\item Deutsch, Niederländisch, \ldots: \emph{w} + V-letzt:

\eal
\ex Ich weiß, wer heute das Buch gelesen hat.
\ex Ich weiß, was Aicke heute gelesen hat.
\zl

Interrogativnebensätze beginnen mit einer \emph{w}-Phrase.

\pause
\item Die \emph{w}-Phrase kann von weit her kommen:
\ea
Ich weiß nicht, [\gruen{über welches Thema}]$_i$ sie versprochen hat,\\
{}[[einen Vortrag \_$_i$] zu halten].
\z

\pause
\item Stellung der anderen Konstituenten ist frei:
\eal
\ex Ich weiß, was keiner diesem Eichhörnchen geben würde.
\ex Ich weiß, was diesem Eichhörnchen keiner geben würde.
\zl


\end{itemize}

}



\frame{

\frametitle{Dänisch, Englisch: Interrogativnebensätze  \emph{w} + SVO}


\begin{itemize}
\item Dänisch: Interrogativnebensätze sind \emph{w} + SVO

\eal
\ex
\gll Jeg ved, hvad Gert har givet ham.\\
     ich weiß was Gert  hat gegeben ihm\\
\glt `Ich weiß, was Gert ihm gegeben hat.'
\ex
\gll Jeg ved, hvem Gert har givet   bogen.\\
     ich weiß wem  Gert hat gegeben Buch.{\sc def}\\
\glt `Ich weiß, wem Gert das Buch gegeben hat.'
\zl

\end{itemize}

}

\frame{
\frametitle{Jiddish: Interrogativnebensätze \emph{w} + V2}

\begin{itemize}
\item Jiddish: Interrogativnebensätze \emph{w} + V2 \citep[Abschnitte~4.1, 4.2]{Diesing90a}

%% \ea
%% %\ex
%% \label{vosmaks}
%% \gll Ikh veys nit   [vos Max hot gegesn].\footnotemark\\
%%      ich weiß nicht \hspaceThis{[}was Max hat gegessen\\
%% \footnotetext{\citew[S.\,68]{Diesing90a}.}
%% \glt `Ich weiß nicht, was Max gegessen hat.'

%% %% \ex%check
%% %% \gll Ikh veys nit   [vos              hot Max gegesn].\footnotemark\\
%% %%      ich weiß nicht \hspaceThis{[}was heute hat Max gegessen\\
%% %% \footnotetext{\citew[S.\,68]{Diesing90a}.}
%% %% \glt `Ich weiß nicht, was Max heute gegessen hat.'


\ea
\gll Ir veyst efsher [avu            do    voynt Roznblat   der goldshmid]?\footnotemark\\
     Sie wissen vielleicht  \spacebr{}wo da wohnt Roznblat der Goldschmied\\
\glt `Wissen Sie vielleicht, wo Roznblat der Goldschmied wohnt?' 
\footnotetext{
\citew[S.\,65]{Diesing90a}. Zitiert aus Olsvanger, \emph{Royte Pomerantsn}, 1949
}
\z
\end{itemize}

}


\frame{
\frametitle{Die Analyse: Voranstellung der \emph{w}-Phrase}

\begin{itemize}
\item Analyse: \emph{w}-Phrase ist extrahiert (\slasch). Wie Vorfeldbesetzung.

Ansonsten Verbletzt und restliche Syntax ganz parallel.

\ea
Ich weiß, [\gruen{worüber}]$_i$ [ \trace$_i$ sie spricht].
\z


\hfill
\begin{forest}
sm edges
[S
  [PP [worüber]]
  [S/PP
    [PP/PP [\trace]]
    [V\rlap{$'$}
      [NP [sie]]
      [V [spricht]]]]]
\end{forest}
\hfill
\begin{forest}
sm edges
[{S[\slasch \sliste{ }]}
  [PP\ibox{1} [worüber]]
  [{S[\slasch \sliste{ \ibox{1} }]}
    [{PP\ibox{1}[\slasch \sliste{ \ibox{1} }]} [\trace]]
    [V\rlap{$'$}
      [NP [sie]]
      [V [spricht]]]]]
\end{forest}
\hfill\mbox{}

\end{itemize}



}


\frame{
\frametitle{Die \emph{w}-Phrase}

\begin{itemize}
\item Kleines Detail: Die vorangestellte \emph{w}-Phrase kann ein einzelnes Fragepronomen oder auch
  komplex sein:
\eal
\ex Ich weiß, [\gruen{worüber}]$_i$ [ \trace$_i$ sie spricht].
\ex Ich weiß, [\gruen{über welches Thema}]$_i$ [ \trace$_i$ sie spricht].
\zl
\pause
\item derselbe Trick wie mit \slasch:\\
      Information über \emph{w}-Element wird im Baum nach oben gegeben. \textsc{que}-Merkmal.

\end{itemize}

}

\frame{
\frametitle{\textsc{que}-Perkolation}

\vfill
\centerline{\begin{forest}
sm edges
[{S[\textsc{que} \sliste{ }, \slasch \sliste{ }]}
  [{PP\ibox{1}[\textsc{que} \sliste{ \ibox{2} }]} 
     [P [über]]
     [{NP[\textsc{que} \sliste{ \ibox{2} }]}
       [{Det[\textsc{que} \sliste{ \ibox{2} }]} [welches]]
       [N [Thema]]]]
  [{S[\slasch \sliste{ \ibox{1} }]}
    [{PP\ibox{1}[\slasch \sliste{ \ibox{1} }]} [\trace]]
    [V\rlap{$'$}
      [NP [sie]]
      [V [spricht]]]]]
\end{forest}}

\vfill
Information über \emph{w}-Wort \iboxb{2} wird nach oben gereicht.

Semantische Information, damit man weiß, wonach gefragt wurde.
\vfill


}



\frame{
\frametitle{Schema für Interrogtivnebensätze}

\vfill
\centerline{
\begin{forest}
[{H[\textsc{que} \sliste{ }, \slasch \sliste{ }]}
  [\ibox{1}{[\textsc{que} \sliste{ \etag{} }]}]
  [H{[\slasch \sliste{ \ibox{1} }]} ]]
\end{forest}
}

\vfill

Schema ähnelt dem Füller-Kopf-Schema.

Zusätzlich wird Element in \textsc{que} gefordert.

Für das Deutsche:\\
V2-Sätze haben Verb in Initialstellung,
Interrogativnebensätze in Finalstellung.

\vfill


}



\subsection{Expletiva zur Satztypmarkierung}

\frame{
\frametitle{Expletiva zur Satztypmarkierung}

\begin{itemize}
\item Germanische Sprachen benutzen Expletiva, um Satztypen kenntlich zu machen,
      falls keine andere Konstituente die entsprechende Position füllt.
\pause
\item Deutsch V2-Hauptsätze

\eal
\ex Drei Reiter ritten zum Tor hinaus.
\ex \rot{Es} ritten drei Reiter zum Tor hinaus.
\zl
\end{itemize}


}

\frame{
\frametitle{Dänisch: \emph{w}-Sätze mit extrahiertem Subjekt}

\begin{itemize}
\item Dänisch: \emph{w} + SVO\\
      Bei Subjektextraktion muss Extraktion explizit kenntlich gemacht werden:
\eal
\ex[]{
\gll Politiet ved ikke, \gruen{hvem} \rot{der}   havde placeret bomben.\\
     Polizei.{\sc def} weiß nicht wer {\sc expl} hat plaziert Bombe.{\sc def}\\
\glt `Die Polizei weiß nicht, wer eigentlich die Bombe plaziert hat.'
}
\ex[*]{
\gll Politiet ved ikke, \gruen{hvem} havde placeret bomben.\\
     Polizei.{\sc def} weiß nicht wer hat plaziert Bombe.{\sc def}\\
}
\zl

\pause

Expletivum macht Extraktion sichtbar:
\eal
\ex[*]{ 
\gll [\gruen{hvem}$_i$     [\trace$_i$ havde placeret bomben]]\\
     \spacebr{}wer {}          hat   plaziert Bombe.\textsc{def}\\\danish
}
\ex[]{
\gll [\gruen{hvem}$_i$     [\rot{der}                    havde \trace$_i$ placeret bomben]]\\
     \spacebr{}who \spacebr{}\textsc{expl} hat   {}         plaziert Bombe.\textsc{def}\\
}
\zl
\pause
\item Expletivum ist in der Subjektposition


\end{itemize}

}

\frame{
\frametitle{Jiddish: \emph{w}-Sätze mit extrahiertem Subjekt}

\begin{itemize}
\item Jiddish: Interrogativnebensätze \emph{w} + V2\\
%% \ea
%% \label{vosmaks}
%% \gll Ikh veys nit   [vos Max hot gegesn].\footnotemark\\
%%      ich weiß nicht \hspaceThis{[}was Max hat gegessen\\
%% \footnotetext{\citew[S.\,68]{Diesing90a}.}
%% \glt `Ich weiß nicht, was Max gegessen hat.'
%% \z
%% \pause
%% \item 
  Wenn %das Subjekt extrahiert wird und 
kein anderes Element ins Vorfeld soll, muss dort ein \emph{es} stehen:

\eal
\ex[]{
\gll ikh hob  zi  gefregt \gruen{ver} \rot{es}   iz beser  far ir\\
     ich habe sie gefragt wer         {\sc expl} ist besser für sie\\
\glt `Ich habe sie gefragt, wer besser für sie ist.'}
\ex[]{
\gll ikh hob  im  gefregt \gruen{vemen} \rot{es}   kenen ale dayne khaverim\\
     ich habe ihn gefragt wen           {\sc expl} kennen alle deine Freunde\\
\glt `Ich habe ihn gefragt, wen alle seine Freunde kennen.'}
\zl

\end{itemize}

}



\subsection{Positionionale Expletiva}

\frame{
\frametitle{Positionale Expletiva}

\begin{itemize}
\item Deutsch, Dänisch, Jiddish, \ldots{} erlauben Expletiva vor dem finiten Verb:

\ea[]{
Es ritten drei Reiter zum Tor hinaus.
}
\z



\end{itemize}

}


\subsection{Lexikonregel zur Einführung der Expletiva}

\frame{
\frametitle{Lexikonregel zur Einführung der Expletiva}


\ea
\label{positional-expl-lr}
\ms{
head & verb\\
arg-st & \ibox{1}\\
} $\mapsto$
\ms{
head & verb\\
arg-st & \sliste{ NP[\type{lnom}]$_{\textit{expl}}$ } $\oplus$ \ibox{1}\\
}
\z
\begin{itemize}
\item Lexikonregel hängt vorn ein Expletivum mit lexikalischem Nominativ an die \argstl \citep{MOe2011a}.

\pause

\item Da der Kasus lexikalisch ist, bleiben alle anderen Kasuszuweisungen unberührt.
\pause
\item Von der Position ist das Expletivum in den SVO-Sprachen ein Subjekt,\\
      weil es das erste Element ist.

\pause
\item Kongruenz mit der ersten NP mit strukturellem Kasus\\
      (richtig für isländische Subjekte/Objekte und klassische deutsche Subjekte). 


\pause
\item Insgesamt lizenzieren unsere Grammatiken jetzt zu viel:
\begin{itemize}
\item Deutsch, Jiddish: Expletiva im Mittelfeld
\item Dänisch: Immer noch Extraktion aus der Subjektposition
\end{itemize}
\end{itemize}



}


\subsection{Interrogativnebensätze mit \emph{w}-Subjekt}

\frame{
\frametitle{Deutsch: \emph{w} + SOV}

\vfill

\centerfit{\begin{forest}
sm edges
[S
       [{NP[\snom]} [wer] ]
       [S/NP 
         [NP/NP [\trace] ]
         [V$'$
           [{NP[\sacc]}  [das Buch,roof] ]
           [V [liest] ] ] ] ]
\end{forest}}

}


\frame{
\frametitle{Danish: \emph{w}-Subjekt + Expl + VO}

\ea
Lexikoneintrag für \emph{læser} `lesen' mit expletivem Subjekt:
\ms{
spr    & \sliste{ NP[\type{lnom}]\upshape \sub{\type{expl}} }\smallskip\\
comps  & \sliste{ NP[\type{str}], NP[\type{str}] } \smallskip\\
arg-st & \sliste{ NP[\type{lnom}]\upshape \sub{\type{expl}}, NP[\type{str}], NP[\type{str}] } 
}
\z


}

\frame{
\frametitle{Danish: \emph{w}-Subjekt + Expl + VO}

\vfill

\centerfit{\scalebox{0.9}{\begin{forest}
sm edges
[S
       [{NP[\snom]} [hvem;wer] ]
       [{S/NP[\snom]}
         [{NP[\lnom]} [der;\textsc{expl}] ]
         [{VP/NP[\snom]}
           [{V$'$/NP[\snom]}
             [V [læser;liest] ]
             [{NP[\snom]/NP[\snom]} [\trace] ] ]
           [{NP[\sacc]} [bogen;Buch.\textsc{def} ] ] ] ] ]
\end{forest}}}



}


\frame{
\frametitle{Jiddish: \emph{w}-Subjekt + V2 mit Expletivum im VF}


\vfill

\centerfit{\scalebox{0.65}{\begin{forest}
sm edges
[S
       [{NP[\snom]} [ver$_i$;wer] ]
       [{S/NP[\snom]}
         [{NP[\lnom]} [es$_j$;\textsc{expl}] ]
         [{S/NP[\snom]/NP[\lnom]}
           [{V \sliste{S//V}}
             [V [leyent$_k$;liest] ] ]
           [{S//V/NP[\snom]/NP[\lnom]}
             [{NP[\lnom]/NP[\lnom]} [\trace$_j$] ] 
             [{VP//V/NP[\snom]}
               [{V$'$//V/NP[\snom]} 
                 [V//V [\trace$_k$] ]
                 [{NP[\snom]/NP[\snom]} [\trace$_i$] ] ]
               [{NP[\sacc]} [dos bukh;the book] ] ] ] ] ] ]
\end{forest}}}


}


\frame{
\frametitle{Expletiva im Deutschen und Jiddischen}


\begin{itemize}
\item Expletiva können im Deutschen nicht im Mittelfeld stehen:
\eal
\ex[]{
Es arbeiten noch drei Männer.
}
\ex[*]{
dass es noch drei Männer arbeiten
}
\zl
\pause
\item Beschränkung, die verlangt, dass diese Elemente extrahiert sein müssen:

\ea
\ms{
arg-st & \sliste{ NP\ibox{1}[\slasch \sliste{ \ibox{1} }] } $\oplus$ \etag\\
}
\z

\pause
\item Aber Extraktion darf nicht über Satzgrenzen hinweg erfolgen:
\ea[*]{
Es$_i$ glaube ich, dass \_$_i$ noch drei Männer arbeiten.
}
\z
\pause
\item Voranstellung von Expletivpronomina über Satzgrenzen muss ohnehin ausgeschlossen werden:
\ea[*]{
Es$_i$ glaube ich, dass \_$_i$ regnet.
}
\z


\end{itemize}

}

\subsection{Zusammenfassung}

\frame{
\frametitle{Zusammenfassung}

\begin{itemize}[<+->]
\item Analyse der Nebensätze mit Komplementierer: SVO oder V2 %= \textsc{inv}$+$/$-$
\item Analyse der Interrogativnebenätze über Fernabhängigkeiten:\\
      \emph{w}-Phrase + V2/SVO/VL
\item \emph{w}-Phrase kann komplex sein und \emph{w}-Wort tief eingebettet.
\item Derselbe Trick: Perkolation von Merkmalen (\textsc{que}).
\item Expletiva können Vorfeld oder Subjektstelle füllen,\\
      um Struktur sichtbar/durchschaubarer zu machen.
\item Einführung des Expletivums als erstes Element auf der \argstl.
\end{itemize}

\pause\pause\pause
}