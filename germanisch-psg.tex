%% -*- coding:utf-8 -*-

\subtitle{Phrasenstrukturgrammatiken und \xbar-Theorie}

\section{Phrasenstrukturgrammatiken und \xbar-Theorie}


\huberlintitlepage[22pt]


\outline{

\begin{itemize}
\item {Überblick über die germanischen Sprachen}
\item Phänomene
\item \alert{Phrasenstrukturgrammatiken und \xbart}
\item Valenz, Argumentanordnung und Adjunkte
\item Verbalkomplexbildung in den SOV-Sprachen
\item Verbstellung: Verberst- und Verbzweitstellung
\item Passiv
\item Eingebettete Sätze
\end{itemize}

}


\frame{
\frametitle{Literaturhinweis}


Zu Phrasenstrukturgrammatiken und \xbart lesen Sie bitte: \citew[Kapitel~3]{MuellerGermanic}.


\begin{refsection}

\nocite{MuellerGermanic}

\printbibliography[heading=none,notkeyword=this]

\end{refsection}


}

\input{geteilte-Folien/phrasenstrukturgrammatik}
\input{geteilte-Folien/xbar-Theorie}

