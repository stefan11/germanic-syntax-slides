%% -*- coding:utf-8 -*-
\author{Stefan Müller}

\subtitle{Valenz, Argumentanordnung und Adjunktstellung}

\section{Valenz, Argumentanordnung und Adjunktstellung}


\huberlintitlepage[22pt]


\subsection{Valenz}


\frame{
\frametitle{Literaturhinweis}



Zu diesem Abschnitt gibt es das Kapitel~3 in \citew{MuellerGermanic}.

Müller, Stefan, \citeyear{MuellerGermanic}. \emph{Germanic Syntax}. Berlin: Language Science
Press. In Vorbereitung. 



}


\frame{
\frametitle{Grundannahmen: Valenz}

\vfill
\centerfit{
\begin{forest}
sm edges
[{S \eliste}
  [{NP[\type{nom}]} [niemand] ]
  [{V$'$\sliste{ NP[\type{nom}] } }
    [{NP[\type{acc}]} [ihn] ]
    [{V \sliste{ NP[\type{nom}], NP[\type{acc}]}} [kennt]] ] ]
\end{forest}}

\vfill
\pause
\begin{itemize}
\item Valenzanforderung ist in einer Liste repräsentiert
\pause
\item Ein Element der Liste wird mit dem Kopf kombiniert.\\
      Liste mit restlichen Elementen wird nach oben gegeben.
\end{itemize}

}


\subsection{Scrambling}

\frame{
\frametitle{Scrambling: Konstituentenstellung im Deutschen}

\vfill
\centerfit{
\begin{forest}
sm edges
[{S \eliste}
   [{NP[\type{acc}]} [ihn] ]
   [{V$'$\sliste{ NP[\type{acc}] } }
      [{NP[\type{nom}]} [niemand] ]
      [{V \sliste{ NP[\type{nom}], NP[\type{acc}]}} [kennt] ] ] ]
\end{forest}}
\vfill
\pause
\begin{itemize}
\item Ein beliebiges Element der Liste kann mit Kopf kombiniert werden.\\
      $\to$ auch Abfolge Acc $<$ Nom analysierbar.\\
      Liste mit restlichen Elementen wird nach oben gegeben.
\end{itemize}

}



\subsection{SVO: Dänisch/Englisch}

\frame{
\frametitle{Dänisch, Englisch, \ldots}

\vfill

\centerfit{\begin{forest}
sm edges
[{V[\spr \eliste, \comps \eliste]}, name=S
   [{NP[\type{nom}]} [nobody] ]
   [V\feattab{
      \spr \sliste{ NP[\type{nom}] }, \comps \sliste{} }, name=VP
     [V\feattab{
         \spr \sliste{ NP[\type{nom}] },\\
         \comps \sliste{ NP[\type{acc}]}} [knows] ]
        [{NP[\type{acc}]} [him] ] ] ]
\node [right=4cm] at (S)
    {
        = S
    };
\node [right=4cm] at (VP)
    {
        = VP
    };
\end{forest}}


\vfill

\begin{itemize}
\item Englisch ist eine SVO-Sprache:\\
      Komplemente rechts des Verbs, Subjekt links
%\item Komplemente können nicht einfach umgestellt werden.
\item Komplemente bilden mit dem Verb zusammen eine Phrase (VP = \comps \sliste{}).

      Diese wird mit dem Subjekt kombiniert.
\end{itemize}

\vfill

}


\frame{
\frametitle{Kein Scrambling}

\begin{itemize}
\item Dänisch, Englisch:\\
      Elemente aus Valenzliste müssen von links nach rechts abgebunden werden.

\pause
\item Deutsch, Niederländisch:\\
      Elemente können in beliebiger Reihenfolge abgebunden werden.

%\pause
%\item 
\end{itemize}

}
%% \frame{
%% \frametitle{Regeln: Englisch und Deutsch}

%% \begin{itemize}
%% \item Englisch:
%% \ea
%% H[\comps \ibox{B}] $\to$ H[\comps \ibox{B} $\oplus$ \sliste{ \ibox{1} } ] ~~~\ibox{1}
%% \z

%% `$\oplus$' zerlegt Liste in zwei Teillisten.

%% \pause

%% \item Deutsch:
%% \ea
%% H[\comps \ibox{A} $\oplus$ \ibox{B}] $\to$ H[\comps \ibox{A} $\oplus$ \sliste{ \ibox{1} } $\oplus$ \ibox{B} ] ~~~\ibox{1}
%% \z

%% \pause

%% \item Das Englische unterscheidet sich vom Deutschen dadurch,\\
%%       dass \ibox{A} die leere Liste ist.

%% $\to$ Englisch ist restriktiver.

%% \end{itemize}


%% }


\frame{
\frametitle{Deutsch}



\centerfit{\begin{forest}
sm edges
[{V[\spr \eliste, \comps \eliste]}, name=S
        [{NP[\type{nom}]} [niemand;nobody] ]
        [{V\feattab{
              \spr \sliste{ }, \comps \sliste{ NP[\type{nom}] } }}, name = Vs
          [{NP[\type{acc}]} [ihn;him] ] 
          [V\feattab{
              \spr \sliste{  },\\
              \comps \sliste{ NP[\type{nom}], NP[\type{acc}]}} [kennt;knows] ]
] ]
\node [right=4cm] at (S)
    {
        = S
    };
\node [right=4cm] at (Vs)
    {
        = V$'$
    };
\end{forest}}

Das Subjekt ist bei finiten Verben in der \compsl \citep{Pollard90a,Kiss95a}.

Abkürzungen: \begin{tabular}[t]{@{}l@{ = }l}
             S  & [\spr \eliste, \comps \eliste]\\
             VP & [\spr \sliste{ NP[\type{nom}] }, \comps \sliste{}]\\
             V$'$ & alle anderen V-Projektionen (außer Verbalkomplexen)\\
             \end{tabular}

}



\frame{
\frametitle{Übungsaufgaben}


\begin{enumerate}
\item Geben Sie die Valenzlisten für folgende Wörter an:
\eal
\ex lachen
\ex essen
\ex übergießen
\ex bezichtigen
\ex er
\ex der
\zl
\item Zeichnen Sie die Bäume für folgende Beispiele:
\eal
\ex weil der Mann ihm ein Buch schenkt
\ex because the man gave the book to him
\ex
\gll at Bjarne læste bogen\\
     dass Bjarne las Buch.{\sc def}\\
\glt `dass Bjarne das Buch las'
\zl
\end{enumerate}







} 
